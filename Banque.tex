% !TeX TXS-program:pdflatex = pdflatex -synctex=1 -recorder -interaction=nonstopmode --shell-escape %.tex
\documentclass[12pt,a4paper]{exam}%Documentation de la classe exam : https://mirrors.ircam.fr/pub/CTAN/macros/latex/contrib/exam/examdoc.pdf
%\usepackage[abspath]{currfile}
\usepackage[utf8]{inputenc}
\usepackage[T1]{fontenc}
\usepackage[french]{babel} 
%\usepackage{amsmath,amssymb}
\usepackage[explicit]{titlesec}
\usepackage{graphicx}
%\usepackage{booktabs}
\usepackage{enumitem}
\usepackage{multirow}
\usepackage{lastpage}
\usepackage{macrolulu2}
\usepackage{tikz}
\usetikzlibrary{calc,shadows}
\usepackage{hyperref}

%\newcommand{\cheminexo}{}
%\newcommand{\cheminexo}{"Z:/Travail/Cours finis/Python/0001 cours perso/2024 2025/L3/Exercices/"}
%\newcommand{\chemincode}{"Z:/Travail/Cours finis/Python/0001 cours perso/2024 2025/L3/Exercices/code/"}
\newcommand{\chemincode}{code/}
%\newcommand{\chemincode}{\currfileabsdir /code/}

\renewcommand{\questionshook}{%
    %\setlength{\leftmargin}{0pt}%
    \setlength{\labelwidth}{-\labelsep}%
    \setlength{\itemsep}{0.5cm}} 
    
    
%En-tête et pied de page
\pagestyle{headandfoot}
\header{\includegraphics[scale=0.035]{python}}{\small\textbf{Banque de questions}}{\includegraphics[scale=0.035]{python}}

%\headrule
\footrule
\footer{}{\scriptsize Page \thepage{} sur \pageref{LastPage}}{}

\titleformat{\section}[display]
  {\normalfont}
  {}
  {0pt}
  {%
    \begin{tikzpicture}
    	\node[minimum width=\textwidth,anchor=west] (B) at (0,0) {};
    	\node[font=\bfseries\large,
    			rounded corners,
    			draw=orange,
    			fill=orange!10,
    			inner sep=8pt,
    			line width=2pt,
    			anchor=west] (A) at (0,0) {\thesection \quad #1};
    	\draw[orange] (A.east) -- (B.east);
    	\foreach \x in {60,45,30,15}
    		{\fill[orange!\x] ([xshift=- \x pt]B.east) circle [radius=10pt];}
    \end{tikzpicture}%
  }
\titlespacing{\section}{0pt}{0pt}{0.5cm}



%\qformat{\textbf{Exercice \thequestion~:}\hfill}
\qformat{\begin{tikzpicture}
    	\node[minimum width=\textwidth,anchor=west] (B) at (0,0) {};
    	\node[font=\bfseries,
    			rounded corners,
    			draw=orange,
    			fill=orange!10,
    			inner sep=6pt,
    			line width=2pt,
    			anchor=west] (A) at (0,0) {Question \thequestion \quad};
    	\draw[orange] (A.east) -- (B.east);
    	\foreach \x in {60,45,30,15}
    		{\fill[orange!\x, drop shadow] ([xshift=- \x pt]B.east) circle [radius=5pt];}
    \end{tikzpicture}
        \vrule depth 1em width 0pt}
%\extrawidth{1cm}
\renewcommand{\solutiontitle}{\noindent\textbf{Code python :}\par\noindent}
%\newenvironment{\reqresult}[2][1]
\newcommand{\img}[2][1]
{
\begin{figure}[H]
	\begin{center}
		\includegraphics[scale=\#1]{\chemincode \#2}
	\end{center}
\end{figure}
}
%baseline={([yshift=-1em] current bounding box.north)}
%baseline=(label.base)
\DeclareRobustCommand{\circled}[1]{%
	\tikz[baseline={([yshift=-1em] current bounding box.north)}]{\node[scale=0.4,circle,
		black, circular drop shadow={shadow xshift=0.2ex, yshift=0.1ex},font=\bfseries,fill= orange!30, draw=orange,inner sep=2pt,minimum size=2.5em] (label) {#1};}}

\setlist[enumerate,1]{
	label =\circled{\Large\arabic*},
	leftmargin=2cm, rightmargin=1cm,
	itemsep=5pt
}
\newcommand*{\MyBall}{\tikz \draw [baseline, ball color=orange!70, draw=orange!20!white] circle (2pt);}
%
\setlist[itemize,1]{label ={\MyBall},leftmargin=2cm, rightmargin=1cm}

%\definecolor{monjaune}{RGB}{255,196,12}
%\definecolor{monjaune}{named}{orange!10}
\colorlet{monjaune}{orange!10}
\definecolor{monmauve}{RGB}{87,23,69}
\definecolor{monnoir}{RGB}{51,51,51}

\newlength{\xE}
\setlength{\xE}{2cm}
\newlength{\xA}
\setlength{\xA}{4cm}
\newlength{\xC}
\setlength{\xC}{13cm}
\newlength{\yE}
\setlength{\yE}{18cm}
\newlength{\yA}
\setlength{\yA}{9cm}
\newlength{\yC}
\setlength{\yC}{6cm}
\newlength{\carreE}
\setlength{\carreE}{5cm}
\newlength{\carreA}
\setlength{\carreA}{12cm}
\newlength{\carreC}
\setlength{\carreC}{6cm}


\printanswers					%Commenter cette ligne pour cacher les solutions.

\newcommand{\nomfichier}{pascorrige.py}

\begin{document}
\thispagestyle{empty}
		\begin{tikzpicture}[
			remember picture,overlay,
			inner sep=0pt,
			outer sep=0pt,
			]
		\coordinate (A) at ([xshift=\xA,yshift=-\yA]current page.north west);
		\coordinate (B) at ([xshift=\xA+\carreA,yshift=-(\yA+\carreA)]current page.north west);
		\coordinate (AA) at ([xshift=\carreA-3.5cm,yshift=0]A);
		\coordinate (BB) at ([xshift=0,yshift=\carreA-3.5cm]B);
		\coordinate (AAA) at ([xshift=0,yshift=-\carreA+3.5cm]A);
		\coordinate (BBB) at ([xshift=-\carreA+3.5cm,yshift=0]B);
		\fill[monjaune,draw=orange,line width= 2pt] (A) -- (AA) -- (BB) -- (B) -- (BBB) -- (AAA) -- cycle;
		
		\coordinate (E) at ([xshift=\xE,yshift=-\yE]current page.north west);
		\coordinate (F) at ([xshift=\xE+\carreE,yshift=-(\yE+\carreE)]current page.north west);
		\coordinate (EE) at ([xshift=\carreE-3cm,yshift=0]E);
		\coordinate (FF) at ([xshift=0,yshift=\carreE-3cm]F);
		\coordinate (EEE) at ([xshift=0,yshift=-\carreE]E);
		\fill[monnoir] (E) -- (EE) -- (FF) -- (F) -- (EEE)  -- cycle;
		
		\coordinate (C) at ([xshift=\xC,yshift=-\yC]current page.north west);
		\coordinate (D) at ([xshift=\xC+\carreC,yshift=-(\yC+\carreC)]current page.north west);
		\coordinate (CC) at ([xshift=0,yshift=-\carreC+3cm]C);
		\coordinate (DD) at ([xshift=-\carreC+3cm,yshift=0]D);
		\coordinate (CCC) at ([xshift=\carreC,yshift=0]C);
		\fill[monnoir] (C) -- (CC) -- (DD) -- (D) -- (CCC)  -- cycle;
		
		%*****bas
		\coordinate (G) at ([yshift=2.5cm]current page.south west);
		\coordinate (H) at ([xshift=4cm,yshift=2.5cm]current page.south);
		\coordinate (I) at ([xshift=3cm,yshift=0.5cm]current page.south);
		\coordinate (J) at ([yshift=0.5cm]current page.south west);
		\coordinate (K) at ([xshift=5cm,yshift=2.5cm]current page.south);
		\coordinate (L) at ([yshift=2.5cm]current page.south east);
		\coordinate (M) at ([yshift=0.5cm]current page.south east);
		\coordinate (N) at ([xshift=4cm,yshift=0.5cm]current page.south);
		\fill[monnoir] (G) -- (H) -- (I) -- (J) -- cycle;
		\fill[orange] (K) -- (L) -- (M) -- (N) -- cycle;		
										
		%*****haut
		\coordinate (GG) at ([yshift=-1cm]current page.north west);
		\coordinate (HH) at ([xshift=-3cm,yshift=-1cm]current page.north);
		\coordinate (II) at ([xshift=-4cm,yshift=-0.5cm]current page.north);
		\coordinate (JJ) at ([yshift=-0.5cm]current page.north west);
		\coordinate (KK) at ([xshift=-3cm,yshift=-1cm]current page.north);
		\coordinate (LL) at ([yshift=-1cm]current page.north east);
		\coordinate (MM) at ([yshift=-0.5cm]current page.north east);
		\coordinate (NN) at ([xshift=-4cm,yshift=-0.5cm]current page.north);
		\fill[monnoir] (GG) -- (HH) -- (II) -- (JJ) -- cycle;
		\fill[orange] (KK) -- (LL) -- (MM) -- (NN) -- cycle;				

		\node [scale=0.7,
				align=center,
				font=\sffamily\bfseries\huge,
				text width=10cm
				] at ([xshift=6cm,yshift=-6cm]A) {
			\includegraphics[scale=0.15]{python}\par
			\vspace{1cm}Python \par
			Banque de questions
			};		
		
		\node [scale=0.6,
				align=center,
				font=\sffamily\bfseries\huge,
				text width=10cm,
				orange
				] at ([xshift=6.5cm,yshift=-1cm]G) {
			Université de TOURS
			};							
		\node [scale=0.5,
				align=center,
				font=\sffamily\bfseries\huge,
				text width=5cm,
				orange
				] at ([xshift=3.5cm,yshift=-2.3cm]C) {
			Frédéric \par LURET
			};	
		\node [scale=0.5,
				align=center,
				font=\sffamily\bfseries,
				text width=10cm,
				align= left,
				anchor = west,
				orange
				] at ([xshift=0.2cm,yshift=0.3cm]EEE) {
			Version du \today
			};					
	\end{tikzpicture}
\clearpage
%\section{Titre}
%\shadedsolutions
\unframedsolutions
%\SolutionEmphasis{\itshape\small}
\tableofcontents
%*********************************************
\section{Site 1}
En rapport avec les séries 100+\par
\href{https://gist.github.com/KirosG/f265f136bd97bd669632fa0f2f2721b4}{lien vers le site d'origine 1}\par
\href{https://github.com/darkprinx/break-the-ice-with-python}{lien vers le site d'origine 2}
\begin{questions}


%Question 1
\renewcommand{\chemincode}{../../code/}

\question
Écrivez un programme qui trouve tous les nombres multiples de 7 mais pas de 5,
entre 2000 et 3200 (les deux inclus).
Les nombres obtenus doivent être imprimés dans une séquence séparée par des virgules sur une seule ligne.
\par
\textbf{Indices : }Utilisez la méthode range(début, fin)
\renewcommand{\nomfichier}{q001.py}
\begin{solution}
    \pythonfile{\chemincode \nomfichier}[][\nomfichier]
\end{solution}

\renewcommand{\nomfichier}{q001-01.py}
\begin{solution}
    \pythonfile{\chemincode \nomfichier}[][\nomfichier]
\end{solution}

%Question 2
\question
Écrivez un programme qui peut calculer la factorielle d'un nombre donné.

Supposons que l'entrée suivante soit fournie au programme:\par

8\newline
Ensuite, la sortie doit être:\newline
40320

\renewcommand{\nomfichier}{q002.py}
\begin{solution}
    \pythonfile{\chemincode \nomfichier}[][\nomfichier]
\end{solution}

\renewcommand{\nomfichier}{q002-01.py}
\begin{solution}
    \pythonfile{\chemincode \nomfichier}[][\nomfichier]
\end{solution}

\renewcommand{\nomfichier}{q002-02.py}
\begin{solution}
    \pythonfile{\chemincode \nomfichier}[][\nomfichier]
\end{solution}

\renewcommand{\nomfichier}{q002-03.py}
\begin{solution}
    \pythonfile{\chemincode \nomfichier}[][\nomfichier]
\end{solution}

\renewcommand{\nomfichier}{q002-04.py}
\begin{solution}
    \pythonfile{\chemincode \nomfichier}[][\nomfichier]
\end{solution}
%Question 3
\question
Avec un nombre entier \textbf{n} donné, écrivez un programme pour générer un dictionnaire qui contient \textbf{(i, i*i)} tel que \textbf{i} est un nombre entier entre \textbf{1 et n} (les deux inclus). et ensuite le programme doit imprimer le dictionnaire.

Supposons que l'entrée suivante soit fournie au programme :\newline
8\newline
La sortie devrait alors être :\newline
\{1: 1, 2: 4, 3: 9, 4: 16, 5: 25, 6: 36, 7: 49, 8: 64\}

\renewcommand{\nomfichier}{q003.py}
\begin{solution}
    \pythonfile{\chemincode \nomfichier}[][\nomfichier]
\end{solution}
\renewcommand{\nomfichier}{q003-01.py}
\begin{solution}
    \pythonfile{\chemincode \nomfichier}[][\nomfichier]
\end{solution}

%Question 4
\question
Écrire un programme qui accepte une séquence de nombres séparés par des virgules à partir de la console et qui génère une liste et un tuple contenant chaque nombre.

Supposons que l'entrée suivante soit fournie au programme :\newline
34,67,55,33,12,98\newline
Ensuite, la sortie doit être:\newline
['34', '67', '55', '33', '12', '98']\newline
('34', '67', '55', '33', '12', '98')
\renewcommand{\nomfichier}{q004.py}
\begin{solution}
    \pythonfile{\chemincode \nomfichier}[][\nomfichier]
\end{solution}
\renewcommand{\nomfichier}{q004-01.py}
\begin{solution}
    \pythonfile{\chemincode \nomfichier}[][\nomfichier]
\end{solution}

%Question 5
\question
Question POO
%Définir une classe qui possède au moins deux méthodes :\newline
%\textbf{getString} : pour obtenir une chaîne de caractères à partir de l'entrée de la console\newline
%\textbf{printString} : pour imprimer la chaîne en majuscules.\newline
%Veuillez également inclure une fonction de test simple pour tester les méthodes de la classe.
%\par
%\textbf{Indices : }Utilisez la méthode \_\_init\_\_ pour construire certains paramètres
%\renewcommand{\nomfichier}{q005.py}
%\begin{solution}
%    \pythonfile{\chemincode \nomfichier}[][\nomfichier]
%\end{solution}

%Question 6
\question
Écrivez un programme qui calcule et imprime la valeur selon la formule donnée :

Q = Racine carrée de [(2 * C * D)/H]\newline

Voici les valeurs fixes de C et H :\newline
C est 50. H est égal à 30.\newline
D est la variable dont les valeurs doivent être introduites dans votre programme dans une séquence séparée par des virgules.\newline
Exemple\newline
Supposons que le programme reçoive la séquence d'entrée suivante, séparée par des virgules :\newline
100,150,180\newline
La sortie du programme devrait être :\newline
18,22,24
\par
\textbf{Indices : }Si la sortie reçue est sous forme décimale, elle doit être arrondi à sa valeur la plus proche (par exemple, si la sortie reçue est de 26,0, elle doit être imprimée comme 26)
\renewcommand{\nomfichier}{q006.py}
\begin{solution}
    \pythonfile{\chemincode \nomfichier}[][\nomfichier]
\end{solution}

\renewcommand{\nomfichier}{q006-01.py}
\begin{solution}
    \pythonfile{\chemincode \nomfichier}[][\nomfichier]
\end{solution}

\renewcommand{\nomfichier}{q006-02.py}
\begin{solution}
    \pythonfile{\chemincode \nomfichier}[][\nomfichier]
\end{solution}

\renewcommand{\nomfichier}{q006-03.py}
\begin{solution}
    \pythonfile{\chemincode \nomfichier}[][\nomfichier]
\end{solution}

\renewcommand{\nomfichier}{q006-04.py}
\begin{solution}
    \pythonfile{\chemincode \nomfichier}[][\nomfichier]
\end{solution}
%Question 7
\question
Écrivez un programme qui prend 2 chiffres, X,Y en entrée et génère un tableau à 2 dimensions. La valeur de l'élément dans la i-ième ligne et la j-ième colonne du tableau doit être i*j.\newline
Remarque : i = 0,1.., X-1 ; j = 0,1,¡Y-1.\newline
Exemple\newline
Supposons que les entrées suivantes soient données au programme :\newline
3,5\newline
La sortie du programme devrait alors être la suivante :\newline
[[0, 0, 0, 0, 0], [0, 1, 2, 3, 4], [0, 2, 4, 6, 8]]\newline
Puis un affichage sous la forme d'un tableau :\newline
0 0 0 0 0\newline
0 1 2 3 4\newline
0 2 4 6 8
\renewcommand{\nomfichier}{q007.py}
\begin{solution}
    \pythonfile{\chemincode \nomfichier}[][\nomfichier]
\end{solution}
\renewcommand{\nomfichier}{q007-01.py}
\begin{solution}
    \pythonfile{\chemincode \nomfichier}[][\nomfichier]
\end{solution}
\renewcommand{\nomfichier}{q007-02.py}
\begin{solution}
    \pythonfile{\chemincode \nomfichier}[][\nomfichier]
\end{solution}
\renewcommand{\nomfichier}{q007-03.py}
\begin{solution}
    \pythonfile{\chemincode \nomfichier}[][\nomfichier]
\end{solution}

%Question 8
\question
Écrivez un programme qui accepte une séquence de mots séparée par des virgules en entrée et imprime les mots dans une séquence séparée par des virgules après les avoir triés de manière alphabétique.\newline
Supposons que l'entrée suivante soit fournie au programme:\newline
sans, bonjour, sac, monde\newline
Ensuite, la sortie doit être:\newline
Sac, bonjour, sans, monde

\renewcommand{\nomfichier}{q008.py}
\begin{solution}
    \pythonfile{\chemincode \nomfichier}[][\nomfichier]
\end{solution}

%Question 9
\question
Écrivez un programme qui accepte une séquence de lignes en entrée et imprime les lignes après avoir mis en majuscules tous les caractères de la phrase. La saisie d'une ligne vide lance votre traitement.\newline
Supposons que l'entrée suivante soit fournie au programme :\newline
Bonjour le monde\newline
C'est en forgeant qu'on devient forgeron\newline
La sortie devrait alors être :\newline
BONJOUR AU MONDE\newline
C'EST EN FORGEANT QU'ON DEVIENT FORGERON
\renewcommand{\nomfichier}{q009.py}
\begin{solution}
    \pythonfile{\chemincode \nomfichier}[][\nomfichier]
\end{solution}

\renewcommand{\nomfichier}{q009-01.py}
\begin{solution}
    \pythonfile{\chemincode \nomfichier}[][\nomfichier]
\end{solution}

%Question 10
\question
Écrivez un programme qui accepte une séquence de mots séparés dans l'espace en entrée et imprime les mots après avoir retiré tous les mots en double et les tris de manière alphanumérique.\newline
Supposons que l'entrée suivante soit fournie au programme:\newline
Bonjour le monde et la pratique rend à nouveau le monde parfait et bonjour\newline
La sortie doit être:\newline
Bonjour bonjour et la le monde nouveau parfait pratique rend à
\par
\textbf{Indices : }
Nous utilisons le conteneur \textbf{set} pour supprimer automatiquement les données dupliqués.
\renewcommand{\nomfichier}{q010.py}
\begin{solution}
    \pythonfile{\chemincode \nomfichier}[][\nomfichier]
\end{solution}
\renewcommand{\nomfichier}{q010-01.py}
\begin{solution}
    \pythonfile{\chemincode \nomfichier}[][\nomfichier]
\end{solution}
\renewcommand{\nomfichier}{q010-02.py}
\begin{solution}
    \pythonfile{\chemincode \nomfichier}[][\nomfichier]
\end{solution}

%Question 11
\question
Écrivez un programme qui accepte une séquence de nombres binaires à 4 chiffres séparés par des virgules comme entrée, puis vérifiez s'ils sont divisibles par 5 ou non.Les nombres divisibles par 5 doivent être imprimés dans une séquence séparée par des virgules.\newline
Exemple:\newline
0100,0011,1010,1001\newline
Qui correspondent respectivement à 4, 3, 10 et 9.\newline
Alors la sortie doit être:\newline
1010
\renewcommand{\nomfichier}{q011.py}
\begin{solution}
    \pythonfile{\chemincode \nomfichier}[][\nomfichier]
\end{solution}

\renewcommand{\nomfichier}{q011-01.py}
\begin{solution}
    \pythonfile{\chemincode \nomfichier}[][\nomfichier]
\end{solution}

\renewcommand{\nomfichier}{q011-02.py}
\begin{solution}
    \pythonfile{\chemincode \nomfichier}[][\nomfichier]
\end{solution}
\renewcommand{\nomfichier}{q011-03.py}
\begin{solution}
    \pythonfile{\chemincode \nomfichier}[][\nomfichier]
\end{solution}

%Question 12
\question
Écrivez un programme, qui trouvera tous les chiffres entre 1000 et 3000 (tous deux inclus) pour lesquels chaque chiffre du nombre est pair.
Les nombres obtenus doivent être imprimés dans une séquence séparée par des virgules sur une seule ligne.

\renewcommand{\nomfichier}{q012.py}
\begin{solution}
    \pythonfile{\chemincode \nomfichier}[][\nomfichier]
\end{solution}
\renewcommand{\nomfichier}{q012-01.py}
\begin{solution}
    \pythonfile{\chemincode \nomfichier}[][\nomfichier]
\end{solution}
\renewcommand{\nomfichier}{q012-02.py}
\begin{solution}
    \pythonfile{\chemincode \nomfichier}[][\nomfichier]
\end{solution}
\renewcommand{\nomfichier}{q012-03.py}
\begin{solution}
    \pythonfile{\chemincode \nomfichier}[][\nomfichier]
\end{solution}
\renewcommand{\nomfichier}{q012-04.py}
\begin{solution}
    \pythonfile{\chemincode \nomfichier}[][\nomfichier]
\end{solution}
%Question 13
\question
Écrivez un programme qui accepte une phrase et qui calcule le nombre de lettres et de chiffres.\newline
Supposons que l'entrée suivante soit fournie au programme:\newline
Bonjour le monde!123\newline
Ensuite, la sortie doit être:\newline
Lettres 14\newline
Chiffres 3
\renewcommand{\nomfichier}{q013.py}
\begin{solution}
    \pythonfile{\chemincode \nomfichier}[][\nomfichier]
\end{solution}
\renewcommand{\nomfichier}{q013-01.py}
\begin{solution}
    \pythonfile{\chemincode \nomfichier}[][\nomfichier]
\end{solution}

\renewcommand{\nomfichier}{q013-02.py}
\begin{solution}
    \pythonfile{\chemincode \nomfichier}[][\nomfichier]
\end{solution}

%Question 14
\question
Écrivez un programme qui accepte une phrase et calculez le nombre de lettres en majuscules et de lettres minuscules.\newline
Supposons que l'entrée suivante soit fournie au programme:\newline
BonJour le Monde!\newline
Ensuite, la sortie doit être:\newline
Majuscules 3\newline
Minuscules 11
\renewcommand{\nomfichier}{q014.py}
\begin{solution}
    \pythonfile{\chemincode \nomfichier}[][\nomfichier]
\end{solution}
\renewcommand{\nomfichier}{q014-01.py}
\begin{solution}
    \pythonfile{\chemincode \nomfichier}[][\nomfichier]
\end{solution}
\renewcommand{\nomfichier}{q014-02.py}
\begin{solution}
    \pythonfile{\chemincode \nomfichier}[][\nomfichier]
\end{solution}
\renewcommand{\nomfichier}{q014-03.py}
\begin{solution}
    \pythonfile{\chemincode \nomfichier}[][\nomfichier]
\end{solution}
\renewcommand{\nomfichier}{q014-04.py}
\begin{solution}
    \pythonfile{\chemincode \nomfichier}[][\nomfichier]
\end{solution}
%Question 15
\question
Écrivez un programme qui calcule la valeur d'un a + aa + aaa + aaaa avec un chiffre donné comme valeur de a.\newline
Supposons que l'entrée suivante soit fournie au programme:\newline
9\newline
Ensuite, la sortie doit être:\newline
Le résultat de : 9 + 99 + 999 + 9999\newline
est : 11106

\renewcommand{\nomfichier}{q015.py}
\begin{solution}
    \pythonfile{\chemincode \nomfichier}[][\nomfichier]
\end{solution}

\renewcommand{\nomfichier}{q015-01.py}
\begin{solution}
    \pythonfile{\chemincode \nomfichier}[][\nomfichier]
\end{solution}
\renewcommand{\nomfichier}{q015-02.py}
\begin{solution}
    \pythonfile{\chemincode \nomfichier}[][\nomfichier]
\end{solution}

%Question 16
\question
Utilisez une compréhension de liste pour élever au carré chaque nombre impair d'une liste. La liste est introduite par une séquence de nombres séparés par des virgules.\newline
Supposons que l'entrée suivante soit fournie au programme :\newline
1,2,3,4,5,6,7,8,9\newline
La sortie devrait alors être :\newline
1,9,25,49,81

\renewcommand{\nomfichier}{q016.py}
\begin{solution}
    \pythonfile{\chemincode \nomfichier}[][\nomfichier]
\end{solution}

%Question 17
\question
Écrivez un programme qui calcule le montant net d'un compte bancaire basé sur un journal de transaction à partir de l'entrée de la console

Le format de journal des transactions est affiché comme suit:\newline
D 100\newline
W 200\newline

D signifie dépôt et w retrait.\newline
Supposons que l'entrée suivante soit fournie au programme:\newline
D 300\newline
D 300\newline
W 200\newline
D 100\newline
Ensuite, la sortie doit être:\newline
500
\renewcommand{\nomfichier}{q017.py}
\begin{solution}
    \pythonfile{\chemincode \nomfichier}[][\nomfichier]
\end{solution}
\renewcommand{\nomfichier}{q017-01.py}
\begin{solution}
    \pythonfile{\chemincode \nomfichier}[][\nomfichier]
\end{solution}
\renewcommand{\nomfichier}{q017-02.py}
\begin{solution}
    \pythonfile{\chemincode \nomfichier}[][\nomfichier]
\end{solution}

%Question 18
\question
Un site Web oblige les utilisateurs à saisir le nom d'utilisateur et le mot de passe pour s'inscrire.Écrivez un programme pour vérifier la validité de la saisie du mot de passe par les utilisateurs.\newline
Voici les critères de vérification du mot de passe:
\begin{enumerate}
	\item Au moins 1 lettre entre [a-z]
	\item Au moins 1 nombre entre [0-9]
	\item Au moins 1 lettre entre [A-Z]
	\item Au moins 1 personnage de [\$ \# @]
	\item Longueur minimal : 6
	\item Longueur maximale : 12
	\item Ne doit pas contenir d'espace
\end{enumerate}
Votre programme doit accepter une séquence de mots de passe séparés par des virgules et les vérifiera conformément aux critères ci-dessus.Les mots de passe qui correspondent aux critères doivent être imprimés, chacun séparé par une virgule.\newline
Exemple\newline
Si les mots de passe suivants sont donnés en entrée au programme:\newline
ABd1234@1,a F1\#,2w3E*,2We3345\newline
Ensuite, la sortie du programme doit être:\newline
AbD1234@1

\renewcommand{\nomfichier}{q018.py}
\begin{solution}
    \pythonfile{\chemincode \nomfichier}[][\nomfichier]
\end{solution}
\renewcommand{\nomfichier}{q018-01.py}
\begin{solution}
    \pythonfile{\chemincode \nomfichier}[][\nomfichier]
\end{solution}
\renewcommand{\nomfichier}{q018-02.py}
\begin{solution}
    \pythonfile{\chemincode \nomfichier}[][\nomfichier]
\end{solution}
\renewcommand{\nomfichier}{q018-03.py}
\begin{solution}
    \pythonfile{\chemincode \nomfichier}[][\nomfichier]
\end{solution}
\renewcommand{\nomfichier}{q018-04.py}
\begin{solution}
    \pythonfile{\chemincode \nomfichier}[][\nomfichier]
\end{solution}
\renewcommand{\nomfichier}{q018-05.py}
\begin{solution}
    \pythonfile{\chemincode \nomfichier}[][\nomfichier]
\end{solution}

%Question 19
\question
Vous devez rédiger un programme pour trier les tuples (nom, âge, hauteur) par ordre croissant où le nom est une chaîne, l'âge et la taille sont des entiers.Les tuples sont entrés par console.\newline

Les critères de tri sont:\newline
\begin{enumerate}
\item Trier basé sur le nom;
\item puis trier en fonction de l'âge;
\item Puis triez par la taille.
\end{enumerate}

Si les tuples suivants sont donnés comme entrée au programme:\newline
Tom,19,80\newline
John,20,90\newline
Jony,17,91\newline
Jony,17,93\newline
Json,21,85
Ensuite, la sortie du programme doit être:\newline
[('John', '20', '90'), ('Jony', '17', '91'), ('Jony', '17', '93'), ('Json', '21', '85'), ('Tom', '19', '80')]

\renewcommand{\nomfichier}{q019.py}
\begin{solution}
    \pythonfile{\chemincode \nomfichier}[][\nomfichier]
\end{solution}

\renewcommand{\nomfichier}{q019-01.py}
\begin{solution}
    \pythonfile{\chemincode \nomfichier}[][\nomfichier]
\end{solution}

%Question 20
\question
Question POO
%Définissez une classe avec un générateur qui peut itérer les nombres, qui sont divisibles par 7, entre une plage donnée 0 et n.
%Par exemple l'entrée suivante :\newline
%18\newline
%donne la sortie :\newline
%0\newline
%7\newline
%14\newline
%
%\renewcommand{\nomfichier}{q020.py}
%\begin{solution}
%    \pythonfile{\chemincode \nomfichier}[][\nomfichier]
%\end{solution}
%\renewcommand{\nomfichier}{q020-01.py}
%\begin{solution}
%    \pythonfile{\chemincode \nomfichier}[][\nomfichier]
%\end{solution}
%\renewcommand{\nomfichier}{q020-02.py}
%\begin{solution}
%    \pythonfile{\chemincode \nomfichier}[][\nomfichier]
%\end{solution}
%Question 21
\question
Un robot se déplace dans un avion à partir du point d'origine (0,0).Le robot peut se déplacer vers le haut, le bas, la gauche et la droite.\newline
La trace du mouvement du robot est indiquée comme suit:\newline
UP 5\newline
DOWN 3\newline
LEFT 3\newline
RIGHT 2

Les nombres qui suivent la direction sont des pas.\newline
Veuillez écrire un programme pour calculer la distance entre la position actuelle après une séquence de mouvements et le point d'origine. Si la distance est un flottant, il suffit d'imprimer l'entier le plus proche.\newline
Exemple:\newline
Si les tuples suivants sont donnés comme entrée au programme:
UP 5\newline
DOWN 3\newline
LEFT 3\newline
RIGHT 2\newline
Ensuite, la sortie du programme doit être:\newline
2

\renewcommand{\nomfichier}{q021.py}
\begin{solution}
    \pythonfile{\chemincode \nomfichier}[][\nomfichier]
\end{solution}

\renewcommand{\nomfichier}{q021-01.py}
\begin{solution}
    \pythonfile{\chemincode \nomfichier}[][\nomfichier]
\end{solution}

%Question 22
\question
Écrivez un programme pour calculer la fréquence des mots à partir de l'entrée.La sortie doit sortir après le tri de la clé de manière alphanumérique.\newline
Supposons que l'entrée suivante soit fournie au programme:\newline
Nouveau sur Python ou choisir entre Python 2 et Python 3 ? Lisez Python 2 ou Python 3.\newline
Ensuite, la sortie doit être:\newline
2:2\newline
3:1\newline
3.:1\newline
?:1\newline
Lisez:1\newline
Nouveau:1\newline
Python:5\newline
choisir:1\newline
entre:1\newline
et:1\newline
ou:2\newline
sur:1

\renewcommand{\nomfichier}{q022.py}
\begin{solution}
    \pythonfile{\chemincode \nomfichier}[][\nomfichier]
\end{solution}
\renewcommand{\nomfichier}{q022-01.py}
\begin{solution}
    \pythonfile{\chemincode \nomfichier}[][\nomfichier]
\end{solution}
\renewcommand{\nomfichier}{q022-02.py}
\begin{solution}
    \pythonfile{\chemincode \nomfichier}[][\nomfichier]
\end{solution}
\renewcommand{\nomfichier}{q022-03.py}
\begin{solution}
    \pythonfile{\chemincode \nomfichier}[][\nomfichier]
\end{solution}
\renewcommand{\nomfichier}{q022-04.py}
\begin{solution}
    \pythonfile{\chemincode \nomfichier}[][\nomfichier]
\end{solution}




%Question 23
\question
Écrire une fonction qui peut calculer la valeur carrée d'un nombre.

\renewcommand{\nomfichier}{q023.py}
\begin{solution}
    \pythonfile{\chemincode \nomfichier}[][\nomfichier]
\end{solution}

%Question 24
\question
Python possède de nombreuses fonctions intégrées, il a une fonction de documentation intégrée pour toutes ses fonctions.
Veuillez écrire un programme pour imprimer la documentation des fonctions suivantes :
\begin{itemize}
\item abs()
\item int()
\item input ()
\end{itemize}
Puis écrire une fonction qui peut calculer la valeur carrée d'un nombre
et lui ajouter une documentation.

\renewcommand{\nomfichier}{q024.py}
\begin{solution}
    \pythonfile{\chemincode \nomfichier}[][q024.py]
\end{solution}

%Question 25
\question
Question POO
%    Définir une classe qui a un paramètre de classe et un même paramètre d'instance.
%    
%\par
%\textbf{Indices : }
%\begin{itemize}
%\item Pour définir un paramètre d'instance, il faut l'ajouter dans la méthode \_\_init\_\_.
%\item Vous pouvez initialiser un objet avec un paramètre de construction ou en définir la valeur ultérieurement.
%\end{itemize}
%\renewcommand{\nomfichier}{q025.py}
%\begin{solution}
%    \pythonfile{\chemincode \nomfichier}[][\nomfichier]
%\end{solution}
%\renewcommand{\nomfichier}{q025-01.py}
%\begin{solution}
%    \pythonfile{\chemincode \nomfichier}[][\nomfichier]
%\end{solution}
%\renewcommand{\nomfichier}{q025-02.py}
%\begin{solution}
%    \pythonfile{\chemincode \nomfichier}[][\nomfichier]
%\end{solution}

%Question 26
\question
Définissez une fonction qui peut calculer la somme de deux nombres.
\par
\textbf{Indices : }Définissez une fonction avec deux nombres comme arguments.Vous pouvez calculer la somme dans la fonction et renvoyer la valeur.
\renewcommand{\nomfichier}{q026.py}
\begin{solution}
    \pythonfile{\chemincode \nomfichier}[][\nomfichier]
\end{solution}

%Question 27
\question
Définissez une fonction qui peut convertir un entier en une chaîne et l'imprimer dans la console.
\par
\textbf{Indices : }Utilisez STR () pour convertir un nombre en chaîne.
\renewcommand{\nomfichier}{q027.py}
\begin{solution}
    \pythonfile{\chemincode \nomfichier}[][\nomfichier]
\end{solution}

%Question 28
\question
Définir une fonction qui peut recevoir deux nombres entiers sous forme de chaîne de caractères et calculer leur somme, puis l'imprimer dans la console.
\par
\textbf{Indices : }Utilisez int() pour convertir une chaîne en entier.
\renewcommand{\nomfichier}{q028.py}
\begin{solution}
    \pythonfile{\chemincode \nomfichier}[][\nomfichier]
\end{solution}

%Question 29
\question
Définissez une fonction qui peut accepter deux chaînes en entrée et les concaténer, puis l'imprimer dans la console.
\par
\textbf{Indices : }Utiliser + pour concaténer les chaines
\renewcommand{\nomfichier}{q029.py}
\begin{solution}
    \pythonfile{\chemincode \nomfichier}[][\nomfichier]
\end{solution}

%Question 30
\question
Définir une fonction capable d'accepter deux chaînes de caractères en entrée et d'imprimer la chaîne de caractères de longueur maximale dans la console. Si les deux chaînes ont la même longueur, la fonction doit imprimer les deux une par ligne.
\par
\textbf{Indices : }Utilisez la fonction Len() pour obtenir la longueur d'une chaîne
\renewcommand{\nomfichier}{q030.py}
\begin{solution}
    \pythonfile{\chemincode \nomfichier}[][\nomfichier]
\end{solution}

%Question 31
\question
Définir une fonction qui accepte un nombre entier en entrée et qui imprime "C'est un nombre pair" si le nombre est pair, sinon "C'est un nombre impair".
\par
\textbf{Indices : }Utilisez un opérateur \% pour vérifier si un nombre est pair ou impair.
\renewcommand{\nomfichier}{q031.py}
\begin{solution}
    \pythonfile{\chemincode \nomfichier}[][\nomfichier]
\end{solution}

%Question 32
\question
Définir une fonction capable d'imprimer un dictionnaire dont les clés sont des nombres compris entre 1 et 3 (les deux inclus) et dont les valeurs sont des carrés des clés.
\par
\textbf{Indices : }
\begin{itemize}
	\item Utiliser le modèle dict[key]=value pour placer une entrée dans un dictionnaire.
	\item Utiliser l'opérateur ** pour obtenir la puissance d'un nombre.
\end{itemize}
\renewcommand{\nomfichier}{q032.py}
\begin{solution}
    \pythonfile{\chemincode \nomfichier}[][\nomfichier]
\end{solution}
\renewcommand{\nomfichier}{q032-01.py}
\begin{solution}
    \pythonfile{\chemincode \nomfichier}[][\nomfichier]
\end{solution}
\renewcommand{\nomfichier}{q032-02.py}
\begin{solution}
    \pythonfile{\chemincode \nomfichier}[][\nomfichier]
\end{solution}

%Question 33
\question
Définir une fonction capable d'imprimer un dictionnaire dont les clés sont des nombres compris entre 1 et 20 (les deux inclus) et dont les valeurs sont des carrés de clés.
\par
\textbf{Indices : }
\begin{itemize}
\item Utiliser le modèle dict[key]=value pour placer une entrée dans un dictionnaire.
\item Utiliser l'opérateur ** pour obtenir la puissance d'un nombre.
\item Utiliser range() pour les boucles.
\end{itemize}
\renewcommand{\nomfichier}{q033.py}
\begin{solution}
    \pythonfile{\chemincode \nomfichier}[][\nomfichier]
\end{solution}
\renewcommand{\nomfichier}{q033-01.py}
\begin{solution}
    \pythonfile{\chemincode \nomfichier}[][\nomfichier]
\end{solution}

%Question 34
\question
Définir une fonction capable de générer un dictionnaire dont les clés sont des nombres compris entre 1 et 20 (les deux inclus) et dont les valeurs sont des carrés de clés. La fonction ne doit imprimer que les valeurs.
\par
\textbf{Indices : }
\begin{itemize}
	\item Utiliser le modèle dict[key]=value pour placer une entrée dans un dictionnaire.
	\item Utiliser l'opérateur ** pour obtenir la puissance d'un nombre.
	\item Utiliser range() pour les boucles.
	\item Utiliser values() pour itérer les clés dans le dictionnaire. Nous pouvons également utiliser items() pour obtenir des paires clé/valeur.
\end{itemize}

\renewcommand{\nomfichier}{q034.py}
\begin{solution}
    \pythonfile{\chemincode \nomfichier}[][\nomfichier]
\end{solution}

%Question 35
\question
Définir une fonction capable de générer un dictionnaire dont les clés sont des nombres compris entre 1 et 20 (les deux inclus) et dont les valeurs sont des carrés de clés. La fonction ne doit imprimer que les clés.
\par
\textbf{Indices : }
\begin{itemize}
	\item Utiliser le modèle dict[key]=value pour placer une entrée dans un dictionnaire.
	\item Utiliser l'opérateur ** pour obtenir la puissance d'un nombre.
	\item Utiliser range() pour les boucles.
	\item Utiliser keys() pour itérer les clés dans le dictionnaire. Nous pouvons également utiliser items() pour obtenir des paires clé/valeur.
\end{itemize}
\renewcommand{\nomfichier}{q035.py}
\begin{solution}
    \pythonfile{\chemincode \nomfichier}[][\nomfichier]
\end{solution}

%Question 36
\question
Définir une fonction capable de générer et d'imprimer une liste dont les valeurs sont des carrés de nombres compris entre 1 et 20 (les deux inclus).
\par
\textbf{Indices : }
\begin{itemize}
	\item Utilisez ** Opérateur pour obtenir la puissance d'un nombre.
	\item Utilisez la range() pour les boucles.
	\item Utilisez list.append() pour ajouter des valeurs dans une liste.
\end{itemize}
\renewcommand{\nomfichier}{q036.py}
\begin{solution}
    \pythonfile{\chemincode \nomfichier}[][\nomfichier]
\end{solution}

%Question 37
\question
Définir une fonction capable de générer une liste dont les valeurs sont des carrés de nombres compris entre 1 et 20 (les deux inclus). La fonction doit ensuite imprimer les 5 derniers éléments de la liste.
\par
\textbf{Indices : }
\begin{itemize}
	\item Utilisez ** Opérateur pour obtenir la puissance d'un nombre.
	\item Utilisez la range() pour les boucles.
	\item Utilisez list.append() pour ajouter des valeurs dans une liste.
	\item Utilisez [N1: N2] pour slicer une liste
\end{itemize}
\renewcommand{\nomfichier}{q037.py}
\begin{solution}
    \pythonfile{\chemincode \nomfichier}[][\nomfichier]
\end{solution}

%Question 38
\question
Définir une fonction capable de générer et d'imprimer un tuple dont les valeurs sont des carrés de nombres compris entre 1 et 20 (les deux inclus).
\par
\textbf{Indices : }
\begin{itemize}
\item Utilisez ** Opérateur pour obtenir la puissance d'un nombre.
\item Utilisez la range() pour les boucles.
\item Utilisez list.append() pour ajouter des valeurs dans une liste.
\item Utilisez tuple() pour obtenir un tuple d'une liste.
\end{itemize}
\renewcommand{\nomfichier}{q038.py}
\begin{solution}
    \pythonfile{\chemincode \nomfichier}[][\nomfichier]
\end{solution}

\renewcommand{\nomfichier}{q038-01.py}
\begin{solution}
    \pythonfile{\chemincode \nomfichier}[][\nomfichier]
\end{solution}

%Question 39
\question
Ecrivez un programme pour générer et imprimer un autre tuple dont les valeurs sont des nombres pairs dans le tuple donné (1,2,3,4,5,6,7,8,9,10).
\par
\textbf{Indices : }
\begin{itemize}
\item Utilisez "for" pour itérer le tuple
\item Utilisez Tuple() pour générer un tuple à partir d'une liste.
\end{itemize}
\renewcommand{\nomfichier}{q039.py}
\begin{solution}
    \pythonfile{\chemincode \nomfichier}[][\nomfichier]
\end{solution}
\renewcommand{\nomfichier}{q039-01.py}
\begin{solution}
    \pythonfile{\chemincode \nomfichier}[][\nomfichier]
\end{solution}
\renewcommand{\nomfichier}{q039-02.py}
\begin{solution}
    \pythonfile{\chemincode \nomfichier}[][\nomfichier]
\end{solution}

%Question 40
\question
Écrire un programme qui accepte une chaîne de caractères en entrée pour imprimer "Oui" si la chaîne est "oui" ou "OUI" ou "Oui", sinon imprimer "Non".

\renewcommand{\nomfichier}{q040.py}
\begin{solution}
    \pythonfile{\chemincode \nomfichier}[][\nomfichier]
\end{solution}

%Question 41
\question
Écrivez un programme qui peut filtrer les nombres pairs dans une liste en utilisant la fonction filter. 

La liste est la suivante : [1,2,3,4,5,6,7,8,9,10].
\par
\textbf{Indices : }
\begin{itemize}
\item Utilisez filter() pour filtrer certains éléments dans une liste.
\item Utilisez lambda pour définir des fonctions anonymes.
\end{itemize}
\renewcommand{\nomfichier}{q041.py}
\begin{solution}
    \pythonfile{\chemincode \nomfichier}[][\nomfichier]
\end{solution}

%Question 42
\question
Écrivez un programme qui peut utiliser map() pour créer une liste dont les éléments sont le carré des éléments de [1,2,3,4,5,6,7,8,9,10].
\par
\textbf{Indices : }
\begin{itemize}
\item Utilisez map() pour générer une liste.
\item Utilisez lambda pour définir des fonctions anonymes.
\end{itemize}
\renewcommand{\nomfichier}{q042.py}
\begin{solution}
    \pythonfile{\chemincode \nomfichier}[][\nomfichier]
\end{solution}

%Question 43
\question
Écrivez un programme qui peut utiliser map() et filter() pour créer une liste dont les éléments sont les carrés des nombres pairs de la liste :

[1,2,3,4,5,6,7,8,9,10].
\par
\textbf{Indices : }
\begin{itemize}
\item Utilisez map() pour générer une liste.
\item Utilisez filter() pour filtrer les éléments d'une liste.
\item Utilisez lambda pour définir des fonctions anonymes.
\end{itemize}
\renewcommand{\nomfichier}{q043.py}
\begin{solution}
    \pythonfile{\chemincode \nomfichier}[][\nomfichier]
\end{solution}

%Question 44
\question
Écrivez un programme qui peut filtrer() pour faire une liste dont les éléments sont des nombres pairs entre 1 et 20 (les deux inclus).
\par
\textbf{Indices : }
Utilisez filter() pour filtrer les éléments d'une liste.
Utilisez lambda pour définir des fonctions anonymes.
\renewcommand{\nomfichier}{q044.py}
\begin{solution}
    \pythonfile{\chemincode \nomfichier}[][\nomfichier]
\end{solution}

\renewcommand{\nomfichier}{q044-01.py}
\begin{solution}
    \pythonfile{\chemincode \nomfichier}[][\nomfichier]
\end{solution}

%Question 45
\question
Écrivez un programme qui peut utiliser map() pour créer une liste dont les éléments sont des carrés de nombres compris entre 1 et 20 (les deux inclus).
\par
\textbf{Indices : }
Utilisez map() pour générer une liste.
Utilisez lambda pour définir des fonctions anonymes.
\renewcommand{\nomfichier}{q045.py}
\begin{solution}
    \pythonfile{\chemincode \nomfichier}[][\nomfichier]
\end{solution}

%Question 46
\question
Question POO
%Définissez une classe nommée American qui possède une méthode statique appelée printNationality.
%\par
%\textbf{Indices : }
%Utilisez @StaticMethod Decorator pour définir la méthode statique de classe.
%\renewcommand{\nomfichier}{q046.py}
%\begin{solution}
%    \pythonfile{\chemincode \nomfichier}[][\nomfichier]
%\end{solution}

%Question 47
\question
Question POO
%Définissez une classe nommée American et sa sous-classe Newyorker.
%
%\renewcommand{\nomfichier}{q047.py}
%\begin{solution}
%    \pythonfile{\chemincode \nomfichier}[][\nomfichier]
%\end{solution}
%
%\renewcommand{\nomfichier}{q047-01.py}
%\begin{solution}
%    \pythonfile{\chemincode \nomfichier}[][\nomfichier]
%\end{solution}

%Question 48
\question
Question POO
%Définir une classe nommée Cercle qui peut être construite par un rayon. La classe Cercle possède une méthode qui permet de calculer la surface.
%
%
%Puis définir une classe rectangle qui peut être construit par une longueur et une largeur. La classe Rectangle possède une méthode qui permet de calculer la surface.
%
%
%
%\par
%\textbf{Indices : }
%Utilisez Def nom\_de\_le\_methode(Self) pour définir une méthode.
%\renewcommand{\nomfichier}{q048.py}
%\begin{solution}
%    \pythonfile{\chemincode \nomfichier}[][\nomfichier]
%\end{solution}
%
%\renewcommand{\nomfichier}{q048-01.py}
%\begin{solution}
%    \pythonfile{\chemincode \nomfichier}[][\nomfichier]
%\end{solution}

%Question 49
\question
Question POO
%Définissez une classe nommée Shape et sa sous-classe Square. La classe Square possède une fonction init qui prend une longueur en argument. Les deux classes disposent d'une fonction area qui permet d'imprimer l'aire de la forme, l'aire de Shape étant égale à 0 par défaut.
%\par
%\textbf{Indices : }Pour remplacer une méthode dans une super-classe, nous pouvons définir une méthode portant le même nom dans la super-classe.
%\renewcommand{\nomfichier}{q048bis.py}
%\begin{solution}
%    \pythonfile{\chemincode \nomfichier}[][\nomfichier]
%\end{solution}


%Question 50
\question
En supposant que nous avons des adresses e-mail au format \textbf{username@companyname.com}, veuillez écrire un programme pour imprimer le nom d'utilisateur d'une adresse e-mail donnée.Les noms d'utilisateurs et les noms d'entreprise sont composés de lettres uniquement.\newline

Exemple:\newline
Si l'adresse e-mail suivante est donnée comme entrée au programme:\newline

John@google.com\newline

Ensuite, la sortie du programme doit être:\newline

John\newline

\par
\textbf{Indices : }aidez vous du package "re"
\renewcommand{\nomfichier}{q049.py}
\begin{solution}
    \pythonfile{\chemincode \nomfichier}[][\nomfichier]
\end{solution}
\renewcommand{\nomfichier}{q049-01.py}
\begin{solution}
    \pythonfile{\chemincode \nomfichier}[][\nomfichier]
\end{solution}

%Question 51
\question
Écrivez un programme qui accepte une séquence de mots séparés par des espaces comme entrée et qui génère une liste contenant toutes les valeurs numériques de cette entrée.\newline
Exemple:\newline
Si les mots suivants sont donnés en entrée au programme:\newline

2 chats et 3 chiens.\newline

Ensuite, la sortie du programme doit être:\newline

['2', '3']\newline

\par
\textbf{Indices : }Utilisez re.findall() pour trouver tous les sous-chaînes à l'aide de regex.
\renewcommand{\nomfichier}{q050.py}
\begin{solution}
    \pythonfile{\chemincode \nomfichier}[][\nomfichier]
\end{solution}
\renewcommand{\nomfichier}{q050-01.py}
\begin{solution}
    \pythonfile{\chemincode \nomfichier}[][\nomfichier]
\end{solution}
\renewcommand{\nomfichier}{q050-02.py}
\begin{solution}
    \pythonfile{\chemincode \nomfichier}[][\nomfichier]
\end{solution}

%%Question 51
%\question
%Écrivez un commentaire spécial pour indiquer qu'un fichier de code source Python est dans Unicode.
%\par
%\textbf{Indices : }
%\renewcommand{\nomfichier}{q051.py}
%\begin{solution}
%    \pythonfile{\chemincode \nomfichier}[][\nomfichier]
%\end{solution}

%Question 52
\question
Écrivez un programme pour calculer:

f (n) = f (n - 1) +100 quand n> 0\newline
et f (0) = 1\newline

avec une entrée n donnée par console (n> 0).

Exemple:\newline
Si le n suivant est donné en entrée au programme:\newline

5

Ensuite, la sortie du programme doit être:\newline

500

\par
\textbf{Indices : }Nous pouvons définir une fonction récursive dans Python.
\renewcommand{\nomfichier}{q052.py}
\begin{solution}
    \pythonfile{\chemincode \nomfichier}[][\nomfichier]
\end{solution}

\renewcommand{\nomfichier}{q052-01.py}
\begin{solution}
    \pythonfile{\chemincode \nomfichier}[][\nomfichier]
\end{solution}

%Question 53
\question
La séquence Fibonacci est calculée en fonction de la formule suivante:

$f(n)=0 \textrm{ si } n=0$\newline
$f(n)=1 \textrm{ si } n=1$\newline
$f(n)=f (n - 1) + f(n - 2) \textrm{ si } n> 1$\newline

Veuillez écrire un programme pour calculer la valeur de F (n) avec une entrée n donnée par console.\newline

Exemple:\newline
Si le n suivant est donné en entrée au programme:\newline

7

Ensuite, la sortie du programme doit être:\newline

13

\par
\textbf{Indices : }Nous pouvons définir une fonction récursive dans Python.
\renewcommand{\nomfichier}{q053.py}
\begin{solution}
    \pythonfile{\chemincode \nomfichier}[][\nomfichier]
\end{solution}
\renewcommand{\nomfichier}{q053-01.py}
\begin{solution}
    \pythonfile{\chemincode \nomfichier}[][\nomfichier]
\end{solution}
\renewcommand{\nomfichier}{q053-02.py}
\begin{solution}
    \pythonfile{\chemincode \nomfichier}[][\nomfichier]
\end{solution}
\renewcommand{\nomfichier}{q053-03.py}
\begin{solution}
    \pythonfile{\chemincode \nomfichier}[][\nomfichier]
\end{solution}
%Question 54
\question
La séquence Fibonacci est calculée en fonction de la formule suivante:


$f(n)=0 \textrm{ si } n=0$\newline
$f(n)=1 \textrm{ si } n=1$\newline
$f(n)=f (n - 1) + f(n - 2) \textrm{ si } n> 1$\newline

Veuillez écrire un programme en utilisant la compréhension de la liste pour imprimer la séquence Fibonacci sous forme de virgule séparée avec une entrée N donnée par console.\newline

Exemple:\newline
Si le n suivant est donné en entrée au programme:\newline

7

Ensuite, la sortie du programme doit être:\newline

0,1,1,2,3,5,8,13
\par
\textbf{Indices : }
\begin{itemize}
\item Nous pouvons définir une fonction récursive dans Python.
\item Utilisez la compréhension de la liste pour générer une liste à partir d'une liste existante.
\item Utilisez <string>.Join() pour concaténer une liste de chaînes.
\end{itemize}
\renewcommand{\nomfichier}{q054.py}
\begin{solution}
    \pythonfile{\chemincode \nomfichier}[][\nomfichier]
\end{solution}
\renewcommand{\nomfichier}{q054-01.py}
\begin{solution}
    \pythonfile{\chemincode \nomfichier}[][\nomfichier]
\end{solution}
\renewcommand{\nomfichier}{q054-02.py}
\begin{solution}
    \pythonfile{\chemincode \nomfichier}[][\nomfichier]
\end{solution}
\renewcommand{\nomfichier}{q054-03.py}
\begin{solution}
    \pythonfile{\chemincode \nomfichier}[][\nomfichier]
\end{solution}



%Question 55
\question
Écrire un programme à l'aide du générateur pour imprimer les nombres pair entre 0 et N sous forme d'une suite de valeur séparées par des virgules. La valeur N est fournie par l'utilisateur.\newline

Exemple:\newline
Si la valeur de N est :\newline

10

La sortie du programme doit être:\newline

0,2,4,6,8,10
\par
\textbf{Indices : }Utilisez yield pour produire la valeur suivante dans le générateur.

\renewcommand{\nomfichier}{q055.py}
\begin{solution}
    \pythonfile{\chemincode \nomfichier}[][\nomfichier]
\end{solution}
\renewcommand{\nomfichier}{q055-01.py}
\begin{solution}
    \pythonfile{\chemincode \nomfichier}[][\nomfichier]
\end{solution}

%Question 56
\question
Veuillez écrire un programme utilisant un générateur pour imprimer les nombres divisibles par 5 et 7 entre 0 et n sous la forme d'une liste séparée par des virgules. La valeur n est fournie par l'utilisateur.\newline

Exemple:\newline
Si le n suivant est donné en entrée au programme:\newline

100

Ensuite, la sortie du programme doit être:\newline

0,35,70
\par
\textbf{Indices : }Utilisez le yield pour produire la valeur suivante dans le générateur.
\renewcommand{\nomfichier}{q056.py}
\begin{solution}
    \pythonfile{\chemincode \nomfichier}[][\nomfichier]
\end{solution}
\renewcommand{\nomfichier}{q056-01.py}
\begin{solution}
    \pythonfile{\chemincode \nomfichier}[][\nomfichier]
\end{solution}


%Question 57
\question
Écrire un code pour vérifier que tous les nombres de la liste [2,4,6,8] sont pairs.
\par
\textbf{Indices : }Utilisez "assert expression" pour effectuer l'opération.
\renewcommand{\nomfichier}{q057.py}
\begin{solution}
    \pythonfile{\chemincode \nomfichier}[][\nomfichier]
\end{solution}

%Question 58
\question
Veuillez écrire une fonction de recherche binaire qui recherche un élément dans une liste triée. La fonction doit renvoyer l'index de l'élément à rechercher dans la liste.
\par
\textbf{Indices : }Utilisez if / elif pour gérer les conditions.
\renewcommand{\nomfichier}{q058.py}
\begin{solution}
    \pythonfile{\chemincode \nomfichier}[][\nomfichier]
\end{solution}
\renewcommand{\nomfichier}{q058-01.py}
\begin{solution}
    \pythonfile{\chemincode \nomfichier}[][\nomfichier]
\end{solution}
\renewcommand{\nomfichier}{q058-02.py}
\begin{solution}
    \pythonfile{\chemincode \nomfichier}[][\nomfichier]
\end{solution}
\renewcommand{\nomfichier}{q058-03.py}
\begin{solution}
    \pythonfile{\chemincode \nomfichier}[][\nomfichier]
\end{solution}

%Question 59
\question
Veuillez générer un flottant aléatoire où la valeur se situe entre 10 et 100 à l'aide du module math.
\par
\textbf{Indices : }Utilisez random.random () pour générer un flottant aléatoire dans [0,1].
\renewcommand{\nomfichier}{q059.py}
\begin{solution}
    \pythonfile{\chemincode \nomfichier}[][\nomfichier]
\end{solution}
\renewcommand{\nomfichier}{q059-01.py}
\begin{solution}
    \pythonfile{\chemincode \nomfichier}[][\nomfichier]
\end{solution}

%Question 60
\question
Veuillez écrire un programme pour produire un nombre pair aléatoire entre 0 et 10 inclus en utilisant le module aléatoire et la compréhension de la liste.
\par
\textbf{Indices : }Utilisez random.choice() à un élément aléatoire d'une liste.
\renewcommand{\nomfichier}{q060.py}
\begin{solution}
    \pythonfile{\chemincode \nomfichier}[][\nomfichier]
\end{solution}
\renewcommand{\nomfichier}{q060-01.py}
\begin{solution}
    \pythonfile{\chemincode \nomfichier}[][\nomfichier]
\end{solution}


%Question 61
\question
Veuillez rédiger un programme pour générer une liste avec 5 nombres aléatoires entre 100 et 200 inclusifs.
\par
\textbf{Indices : }Utilisez random.sample() pour générer une liste de valeurs aléatoires.
\renewcommand{\nomfichier}{q061.py}
\begin{solution}
    \pythonfile{\chemincode \nomfichier}[][\nomfichier]
\end{solution}

%Question 62
\question
Veuillez écrire un programme pour générer de manière aléatoire une liste avec 5 nombres, qui sont divisibles par 5 et 7, entre 1 et 1000 inclusifs.
\par
\textbf{Indices : }Utilisez random.sample() pour générer une liste de valeurs aléatoires.
\renewcommand{\nomfichier}{q062.py}
\begin{solution}
    \pythonfile{\chemincode \nomfichier}[][\nomfichier]
\end{solution}

%Question 63
\question
Veuillez écrire un programme pour imprimer au hasard un numéro entier entre 7 et 15 inclusif.
\par
\textbf{Indices : }Utilisez random.randrange()
\renewcommand{\nomfichier}{q063.py}
\begin{solution}
    \pythonfile{\chemincode \nomfichier}[][\nomfichier]
\end{solution}

%Question 64
\question
Veuillez écrire un programme pour comprimer et décompresser la chaîne "Hello World! Hello World! Hello World! Hello World!".
\par
\textbf{Indices : }Utilisez zlib.compress () et zlib.decompress () pour compresser et décompresser une chaîne.
\renewcommand{\nomfichier}{q064.py}
\begin{solution}
    \pythonfile{\chemincode \nomfichier}[][\nomfichier]
\end{solution}

%Question 65
\question
Rédiger un programme pour mélanger et imprimer la liste [3,6,7,8].
\par
\textbf{Indices : }Utilisez la fonction Shuffle() pour mélanger une liste.
\renewcommand{\nomfichier}{q065.py}
\begin{solution}
    \pythonfile{\chemincode \nomfichier}[][\nomfichier]
\end{solution}
\renewcommand{\nomfichier}{q065-01.py}
\begin{solution}
    \pythonfile{\chemincode \nomfichier}[][\nomfichier]
\end{solution}

%Question 66
\question
Écrire un programme pour générer toutes les phrases où le sujet est dans ["I", "You"] et le verbe est dans ["Play", "Love"] et l'objet est dans ["Hockey", "Football"].

\renewcommand{\nomfichier}{q066.py}
\begin{solution}
    \pythonfile{\chemincode \nomfichier}[][\nomfichier]
\end{solution}
\renewcommand{\nomfichier}{q066-01.py}
\begin{solution}
    \pythonfile{\chemincode \nomfichier}[][\nomfichier]
\end{solution}

%Question 67
\question
En utilisant la compréhension de liste, veuillez écrire un programme pour imprimer la liste après avoir supprimé les nombres divisibles par 5 et 7 dans [12,24,35,70,88,120,155].

\renewcommand{\nomfichier}{q067.py}
\begin{solution}
    \pythonfile{\chemincode \nomfichier}[][\nomfichier]
\end{solution}

%Question 68
\question
En utilisant la compréhension de liste, écrivez un programme qui génère un tableau 3D 3*5*8 dont chaque élément est 0.
.
\renewcommand{\nomfichier}{q068.py}
\begin{solution}
    \pythonfile{\chemincode \nomfichier}[][\nomfichier]
\end{solution}

%Question 69
\question
En utilisant la compréhension de liste, veuillez écrire un programme pour imprimer la liste après avoir enlevé la valeur 24 dans [12,24,35,24,88,120,155].
\par
\textbf{Indices : }Utilisez la méthode de suppression de la liste pour supprimer une valeur.
\renewcommand{\nomfichier}{q069.py}
\begin{solution}
    \pythonfile{\chemincode \nomfichier}[][\nomfichier]
\end{solution}
\renewcommand{\nomfichier}{q069-01.py}
\begin{solution}
    \pythonfile{\chemincode \nomfichier}[][\nomfichier]
\end{solution}

%Question 70
\question
Question POO
%Définissez une classe Personne et ses deux classes enfants : Homme et Femme. Toutes les classes ont une méthode "getGenre" qui peut afficher "Homme" pour la classe Homme et "Femme" pour la classe Femme.
%\par
%\textbf{Indices : }Utilisez la subclass(parentClass) pour définir une classe d'enfants.
%\renewcommand{\nomfichier}{q070.py}
%\begin{solution}
%    \pythonfile{\chemincode \nomfichier}[][\nomfichier]
%\end{solution}
%\renewcommand{\nomfichier}{q070-01.py}
%\begin{solution}
%    \pythonfile{\chemincode \nomfichier}[][\nomfichier]
%\end{solution}

%Question 71
\question
Veuillez écrire un programme qui accepte une chaîne de la console et l'imprimez dans l'ordre inverse.\newline

Exemple:\newline
Si la chaîne suivante est donnée en entrée au programme:\newline

Rise pour voter Sir\newline

Ensuite, la sortie du programme doit être:\newline

riS retov ruop esiR

\renewcommand{\nomfichier}{q071.py}
\begin{solution}
    \pythonfile{\chemincode \nomfichier}[][\nomfichier]
\end{solution}

%Question 72
\question
Veuillez écrire un programme qui accepte une chaîne de caractères de la console et qui imprime les caractères qui ont des index pairs.\newline

Exemple:\newline
Si la chaîne suivante est donnée en entrée au programme:\newline

H1E2L3L4O5W6O7R8L9D

Ensuite, la sortie du programme doit être:\newline

HELLOWORLD
\par
\textbf{Indices : }Utilisez la liste [:: 2] pour itérer une liste par étape 2.
\renewcommand{\nomfichier}{q072.py}
\begin{solution}
    \pythonfile{\chemincode \nomfichier}[][\nomfichier]
\end{solution}

\renewcommand{\nomfichier}{q072-01.py}
\begin{solution}
    \pythonfile{\chemincode \nomfichier}[][\nomfichier]
\end{solution}

\renewcommand{\nomfichier}{q072-02.py}
\begin{solution}
    \pythonfile{\chemincode \nomfichier}[][\nomfichier]
\end{solution}

%Question 73
\question
Veuillez écrire un programme qui imprime toutes les permutations de [1,2,3]
\par
\textbf{Indices : }Utilisez itertools.permutations() pour obtenir des permutations de liste.
\renewcommand{\nomfichier}{q073.py}
\begin{solution}
    \pythonfile{\chemincode \nomfichier}[][\nomfichier]
\end{solution}

\renewcommand{\nomfichier}{q073-01.py}
\begin{solution}
    \pythonfile{\chemincode \nomfichier}[][\nomfichier]
\end{solution}

%Question 73
\question
Écrire un programme pour résoudre un casse-tête classique de la Chine ancienne :
Nous comptons 35 têtes et 94 pattes parmi les poulets et les lapins d'une ferme. Combien de lapins et de poulets avons-nous ?
\par
\textbf{Indices : }Utilisez pour la boucle pour itérer toutes les solutions possibles.
\renewcommand{\nomfichier}{q074.py}
\begin{solution}
    \pythonfile{\chemincode \nomfichier}[][\nomfichier]
\end{solution}

%Question 74
\question
Écrivez une fonction pour calculer 5/0 et utilisez try/except pour attraper les exceptions.

\renewcommand{\nomfichier}{q117.py}
\begin{solution}
    \pythonfile{\chemincode \nomfichier}[][\nomfichier]
\end{solution}
%Question 75
\question
Définir une classe d'exception personnalisée qui prend un message sous forme de chaîne comme attribut.

\renewcommand{\nomfichier}{q118.py}
\begin{solution}
    \pythonfile{\chemincode \nomfichier}[][\nomfichier]
\end{solution}

%Question 76
\question
Écrire un programme pour calculer $1/2+2/3+3/4+...+n/n+1$ avec une entrée n.
Avec la valeur suivante :\newline
5\newline
La sortie sera :\newline
3.55
\renewcommand{\nomfichier}{q119.py}
\begin{solution}
    \pythonfile{\chemincode \nomfichier}[][\nomfichier]
\end{solution}

\renewcommand{\nomfichier}{q119-01.py}
\begin{solution}
    \pythonfile{\chemincode \nomfichier}[][\nomfichier]
\end{solution}


%Question 77
\question
Veuillez écrire un programme pour imprimer la liste après avoir enlevé les nombres pairs dans [5,6,77,45,22,12,24].
\renewcommand{\nomfichier}{q120.py}
\begin{solution}
    \pythonfile{\chemincode \nomfichier}[][\nomfichier]
\end{solution}

\renewcommand{\nomfichier}{q120-01.py}
\begin{solution}
    \pythonfile{\chemincode \nomfichier}[][\nomfichier]
\end{solution}

%Question 78
\question
En utilisant la compréhension de liste, veuillez écrire un programme pour imprimer la liste après avoir enlevé les 0ème, 2ème, 4ème, 6ème nombres dans [12,24,35,70,88,120,155].
\renewcommand{\nomfichier}{q121.py}
\begin{solution}
    \pythonfile{\chemincode \nomfichier}[][\nomfichier]
\end{solution}

\renewcommand{\nomfichier}{q121-01.py}
\begin{solution}
    \pythonfile{\chemincode \nomfichier}[][\nomfichier]
\end{solution}

%Question 79
\question
En utilisant la compréhension de liste, veuillez écrire un programme pour imprimer la liste après avoir enlevé les 2ème à 4ème nombres dans [12,24,35,70,88,120,155].
\renewcommand{\nomfichier}{q122.py}
\begin{solution}
    \pythonfile{\chemincode \nomfichier}[][\nomfichier]
\end{solution}

\renewcommand{\nomfichier}{q122-01.py}
\begin{solution}
    \pythonfile{\chemincode \nomfichier}[][\nomfichier]
\end{solution}

%Question 80
\question
En utilisant la compréhension de liste, veuillez écrire un programme pour imprimer la liste après avoir enlevé les 0ème, 4ème et 5ème nombres dans [12,24,35,70,88,120,155].
\renewcommand{\nomfichier}{q123.py}
\begin{solution}
    \pythonfile{\chemincode \nomfichier}[][\nomfichier]
\end{solution}

\renewcommand{\nomfichier}{q123-01.py}
\begin{solution}
    \pythonfile{\chemincode \nomfichier}[][\nomfichier]
\end{solution}

%Question 81
\question
Avec deux listes données [1,3,6,78,35,55] et [12,24,35,24,88,120,155], écrivez un programme pour créer une liste dont les éléments sont l'intersection des listes données ci-dessus.
\renewcommand{\nomfichier}{q124.py}
\begin{solution}
    \pythonfile{\chemincode \nomfichier}[][\nomfichier]
\end{solution}

\renewcommand{\nomfichier}{q124-01.py}
\begin{solution}
    \pythonfile{\chemincode \nomfichier}[][\nomfichier]
\end{solution}

%Question 82
\question
Avec une liste donnée [12,24,35,24,88,120,155,88,120,155], écrivez un programme pour imprimer cette liste après avoir supprimé toutes les valeurs en double, en conservant l'ordre original.
\renewcommand{\nomfichier}{q125.py}
\begin{solution}
    \pythonfile{\chemincode \nomfichier}[][\nomfichier]
\end{solution}

\renewcommand{\nomfichier}{q125-01.py}
\begin{solution}
    \pythonfile{\chemincode \nomfichier}[][\nomfichier]
\end{solution}

%Question 83
\question
Veuillez écrire un programme qui compte et imprime les numéros de chaque caractère dans une chaîne de caractères saisie par la console.
Par exemple, avec l'entrée suivante :\newline
abcdefgabc\newline
La sortie est :\newline
a,2\newline
b,2\newline
c,2\newline
d,1\newline
e,1\newline
f,1\newline
g,1
\renewcommand{\nomfichier}{q126.py}
\begin{solution}
    \pythonfile{\chemincode \nomfichier}[][\nomfichier]
\end{solution}

\renewcommand{\nomfichier}{q126-01.py}
\begin{solution}
    \pythonfile{\chemincode \nomfichier}[][\nomfichier]
\end{solution}

\renewcommand{\nomfichier}{q126-02.py}
\begin{solution}
    \pythonfile{\chemincode \nomfichier}[][\nomfichier]
\end{solution}

\renewcommand{\nomfichier}{q126-03.py}
\begin{solution}
    \pythonfile{\chemincode \nomfichier}[][\nomfichier]
\end{solution}

%Question 84
\question
A partir de la feuille de résultats des participants à la journée sportive de votre université, vous devez trouver le score du deuxième. On vous donne les scores. Classez-les dans une liste et trouvez le score du deuxième.

Si la chaîne suivante est donnée en entrée au programme :\newline
5\newline
2 3 6 6 5\newline
La sortie est :\newline
5
\renewcommand{\nomfichier}{q127.py}
\begin{solution}
    \pythonfile{\chemincode \nomfichier}[][\nomfichier]
\end{solution}

\renewcommand{\nomfichier}{q127-01.py}
\begin{solution}
    \pythonfile{\chemincode \nomfichier}[][\nomfichier]
\end{solution}

\renewcommand{\nomfichier}{q127-02.py}
\begin{solution}
    \pythonfile{\chemincode \nomfichier}[][\nomfichier]
\end{solution}

%Question 85
\question
On vous donne une chaîne de caractères S et une largeur W. Votre tâche consiste à envelopper la chaîne de caractères dans un paragraphe de largeur.

Si la chaîne suivante est donnée en entrée au programme :\newline
ABCDEFGHIJKLIMNOQRSTUVWXYZ\newline
4\newline
La sortie est :\newline
ABCD\newline
EFGH\newline
IJKL\newline
IMNO\newline
QRST\newline
UVWX\newline
YZ
\renewcommand{\nomfichier}{q128.py}
\begin{solution}
    \pythonfile{\chemincode \nomfichier}[][\nomfichier]
\end{solution}

\renewcommand{\nomfichier}{q128-01.py}
\begin{solution}
    \pythonfile{\chemincode \nomfichier}[][\nomfichier]
\end{solution}

\renewcommand{\nomfichier}{q128-02.py}
\begin{solution}
    \pythonfile{\chemincode \nomfichier}[][\nomfichier]
\end{solution}

\renewcommand{\nomfichier}{q128-03.py}
\begin{solution}
    \pythonfile{\chemincode \nomfichier}[][\nomfichier]
\end{solution}

\renewcommand{\nomfichier}{q128-04.py}
\begin{solution}
    \pythonfile{\chemincode \nomfichier}[][\nomfichier]
\end{solution}

%Question 86
\question
On vous donne un nombre entier, N. Votre tâche consiste à imprimer un rangoli alphabétique de taille N. (Le rangoli est une forme d'art populaire indien basé sur la création de motifs).

Différentes tailles de rangoli alphabétique sont présentées ci-dessous :\newline
size 3\newline\newline

----c----\newline
--c-b-c--\newline
c-b-a-b-c\newline
--c-b-c--\newline
----c----\newline

size 5\newline

--------e--------\newline
------e-d-e------\newline
----e-d-c-d-e----\newline
--e-d-c-b-c-d-e--\newline
e-d-c-b-a-b-c-d-e\newline
--e-d-c-b-c-d-e--\newline
----e-d-c-d-e----\newline
------e-d-e------\newline
--------e--------
\renewcommand{\nomfichier}{q129.py}
\begin{solution}
    \pythonfile{\chemincode \nomfichier}[][\nomfichier]
\end{solution}

\renewcommand{\nomfichier}{q129-01.py}
\begin{solution}
    \pythonfile{\chemincode \nomfichier}[][\nomfichier]
\end{solution}

%Question 87
\question
Etant donné 2 ensembles d'entiers, M et N, imprimez leur différence symétrique par ordre croissant. Le terme "différence symétrique" indique les valeurs qui existent dans M ou N mais qui n'existent pas dans les deux.

La première ligne d'entrée contient un entier, M. La deuxième ligne contient M entiers séparés par des espaces.La troisième ligne contient un entier, N.La quatrième ligne contient N entiers séparés par des espaces.\newline
4\newline
2 4 5 9\newline
4\newline
2 4 11 12\newline

La sortie est :\newline
5\newline
9\newline
11\newline
12
\renewcommand{\nomfichier}{q130.py}
\begin{solution}
    \pythonfile{\chemincode \nomfichier}[][\nomfichier]
\end{solution}

%Question 88
\question
On vous donne des mots. Certains mots peuvent se répéter. Pour chaque mot, indiquez le nombre d'occurrences. L'ordre de sortie doit correspondre à l'ordre d'apparition du mot en entrée. 

Voir l'exemple d'entrée/sortie pour plus de précisions.\newline

4\newline
bcdef\newline
abcdefg\newline
bcde\newline
bcdef

La sortie est :\newline
3\newline
2 1 1
\renewcommand{\nomfichier}{q131.py}
\begin{solution}
    \pythonfile{\chemincode \nomfichier}[][\nomfichier]
\end{solution}


%Question 89
\question
Votre tâche consiste à compter la fréquence des lettres de la chaîne et à imprimer les lettres par ordre décroissant de fréquence.

Si la chaîne suivante est donnée en entrée du programme :\newline

aabbbccde

La sortie est :\newline
b 3\newline
a 2\newline
c 2\newline
d 1\newline
e 1
\renewcommand{\nomfichier}{q131.py}
\begin{solution}
    \pythonfile{\chemincode \nomfichier}[][\nomfichier]
\end{solution}

\renewcommand{\nomfichier}{q131-01.py}
\begin{solution}
    \pythonfile{\chemincode \nomfichier}[][\nomfichier]
\end{solution}

\renewcommand{\nomfichier}{q131-02.py}
\begin{solution}
    \pythonfile{\chemincode \nomfichier}[][\nomfichier]
\end{solution}

%
%Question 1
\question
	Écrivez une fonction \textbf{precedent\_suivant()} qui lit un numéro entier et renvoie ses numéros précédents et suivants.
	
	Exemple d'entrée:
	
	precedent\_suivant(179)
	
	Exemple de sortie:
	
	(178, 180)
  \par
  \renewcommand{\nomfichier}{q075.py}
  \begin{solution}
      \pythonfile{\chemincode \nomfichier}[][\nomfichier]
  \end{solution}       
%Question 2
  \question
  N étudiants prennent K pommes et les distribuent entre eux uniformément.La partie restante (indivisible) reste dans le panier.Combien de pommes aura chaque étudiante et combien resteront dans le panier ?
  
  La fonction lit les nombres n et k et renvoie les deux réponses pour les questions ci-dessus.

	Exemple d'entrée:
	
	Apple\_sharing(6, 50)
	
	Exemple de sortie:
	
	(8, 2)
  \par
  \renewcommand{\nomfichier}{q076.py}
  \begin{solution}
      \pythonfile{\chemincode \nomfichier}[][\nomfichier]
  \end{solution}
    
%Question 3
  \question
  Écrivez une fonction appelée \textbf{carre()} qui calcule la valeur du carré d'un nombre.

	Exemple d'entrée:
	
	carre(6)
	
	Exemple de sortie:
	
	36
  \par
  \renewcommand{\nomfichier}{q077.py}
  \begin{solution}
      \pythonfile{\chemincode \nomfichier}[][\nomfichier]
  \end{solution}
        
%Question 4
  \question
  Écrire la fonction \textbf{heures\_minutes()} pour transformer le nombre donné en secondes en heures et minutes.
	
	Exemple 1:
	
	heures\_minutes(3900)\newline
	sortie : (1, 5)
	
	Exemple 2:
	
	heures\_minutes(60)\newline
	sortie : (0, 1)
        \par
        \renewcommand{\nomfichier}{q078.py}
        \begin{solution}
            \pythonfile{\chemincode \nomfichier}[][\nomfichier]
        \end{solution}
        
%Question 5
    \question
    Étant donné deux horodatages du même jour.
    Chaque horodatage est représenté par un nombre :
    \begin{itemize}
    \item d'heures
    \item de minutes
    \item de secondes
    \end{itemize}
    
    L'instant du premier horodatage s'est produit avant l'instant du second. Calculez le nombre de secondes qui se sont écoulées entre les deux.

		Exemple 1:
		
		two\_timestamp(1,1,1,2,2,2)\newline
		Sortie : 3661
		
		Exemple 2:
		
		two\_timestamp(1,2,30,1,3,20)\newline
		Sortie : 50
        \par
        \renewcommand{\nomfichier}{q079.py}
        \begin{solution}
            \pythonfile{\chemincode \nomfichier}[][\nomfichier]
        \end{solution}
        
%Question 6
    \question
    Créez une fonction nommée two\_digits().
    
    Étant donné un entier à deux chiffres, two\_digits() renvoie son chiffre gauche (le chiffre des dizaines) puis son chiffre droit (le chiffre des unités).

		Exemple d'entrée:
		
		two\_digits(79)
		
		Exemple de sortie:
		
		(7, 9)
    \par
    \renewcommand{\nomfichier}{q080.py}
    \begin{solution}
        \pythonfile{\chemincode \nomfichier}[][\nomfichier]
    \end{solution}
        
%Question 7
    \question
    Écrire la fonction nommée swap\_digits().
    
    Étant donné un entier à deux chiffres, swap\_digits() échange ses chiffres et imprimez le résultat.

		Exemple d'entrée:
		
		swap\_digits(79)
		
		Exemple de sortie:
		
		97
    \par
    \renewcommand{\nomfichier}{q081.py}
    \begin{solution}
        \pythonfile{\chemincode \nomfichier}[][\nomfichier]
    \end{solution}
        
%Question 8
    \question
    Écrire la fonction last\_two\_digits().Étant donné un entier supérieur à 9, last\_two\_digits() imprime ses deux derniers chiffres.

		Exemple d'entrée:
		
		last\_two\_digits(1234)
		
		Exemple de sortie:
		
		34
    \par
    \renewcommand{\nomfichier}{q082.py}
    \begin{solution}
        \pythonfile{\chemincode \nomfichier}[][\nomfichier]
    \end{solution}
        
%Question 9
    \question
    Écrire la fonction tens\_digit().
    
    Étant donné un entier, tens\_digit() renvoie son chiffre de dizaines.

		Exemple 1:
		
		tens\_digit(1234)\newline
		Sortie : 3
		
		Exemple 2:
		
		tens\_digit(179)\newline
		Sortie : 7
    \par
    \renewcommand{\nomfichier}{q083.py}
    \begin{solution}
        \pythonfile{\chemincode \nomfichier}[][\nomfichier]
    \end{solution}
        
%Question 10
    \question
    Écrire la fonction digits\_sum().
    
    Étant donné un numéro à trois chiffres, digits\_sum() trouve la somme de ses chiffres.

		Exemple d'entrée:
		
		digits\_sum(123)
		
		Exemple de sortie:
		
		6
    \par
    \renewcommand{\nomfichier}{q084.py}
    \begin{solution}
        \pythonfile{\chemincode \nomfichier}[][\nomfichier]
    \end{solution}
        
%Question 11
    \question
    Écrire la fonction first\_digit(). Étant donné un nombre réel positif, first\_digit() renvoie son premier chiffre (à droite de la virgule).

		Exemple d'entrée:
		
		first\_digit(1.79)
		
		Exemple de sortie:
		
		7
    \par
    \renewcommand{\nomfichier}{q085.py}
    \begin{solution}
        \pythonfile{\chemincode \nomfichier}[][\nomfichier]
    \end{solution}
        
%Question 12
		\question
		Une voiture peut parcourir une distance de N kilomètres par jour. Combien de jours lui faudra-t-il pour parcourir un itinéraire d'une longueur de M kilomètres ?
		Instructions :
		
    Écrire une fonction car\_route() qui prend deux arguments :
    \begin{itemize}
	    \item la distance qu'elle peut parcourir en un jour
	    \item la distance à parcourir
    \end{itemize}
    
    Cette fonction calcule le nombre de jours qu'il faudra pour parcourir cette distance.
		
		Exemple d'entrée:
		
		car\_route(20, 40)
		
		Exemple de sortie:
		
		2
    \par
    \renewcommand{\nomfichier}{q086.py}
    \begin{solution}
        \pythonfile{\chemincode \nomfichier}[][\nomfichier]
    \end{solution}
    
%Question 13
		\question
		Écrivez une fonction century().
		Cette dernière prend une année en paramètre sous la forme d'un entier et renvoi le numéro du siècle.
		
		Exemple d'entrée:
		
		century(2001)
		
		Exemple de sortie:
		
		21
		\par
		\renewcommand{\nomfichier}{q087.py}
		\begin{solution}
		    \pythonfile{\chemincode \nomfichier}[][\nomfichier]
		\end{solution}
        
%Question 14
		\question
		Un petit gâteau coûte d euros et c centimes. Écrivez une fonction qui détermine le nombre d'euros et de centimes qu'une personne devrait payer pour n petits gâteaux. La fonction reçoit trois nombres : d, c, n et doit renvoyer deux nombres : le coût total en euros et en centimes.
		
		Exemple d'entrée:
		
		total\_cost(15, 22, 4)
		
		Sortie :
		
		(60, 88)
		\par
		\renewcommand{\nomfichier}{q088.py}
		\begin{solution}
		    \pythonfile{\chemincode \nomfichier}[][\nomfichier]
		\end{solution}
        
%Question 15
		\question
		Écrire une fonction day\_of\_week(). On lui fourni un entier k compris entre 1 et 365, la fonction day\_of\_week() trouve le numéro du jour de la semaine pour le k-ième jour de l'année, à condition que le 1er janvier de cette année soit un jeudi.
		
		Les jours de la semaine sont numérotés comme:
		
		\begin{enumerate}
		\item[0] Dimanche
		\item[1] Lundi
		\item[2] Mardi ...
		
		\item[6] Samedi
		\end{enumerate}
		
		Exemple d'entrée:
		
		day\_of\_week(1)
		
		Exemple de sortie:
		
		4
		\par
		\renewcommand{\nomfichier}{q089.py}
		\begin{solution}
		    \pythonfile{\chemincode \nomfichier}[][\nomfichier]
		\end{solution}
        
%Question 16
		\question
		Soit l'entier n - le nombre de minutes qui se sont écoulées depuis minuit, combien d'heures et de minutes sont affichées sur l'horloge numérique de 24 heures ? Écrivez une fonction digital\_clock() pour le calculer. La fonction doit afficher deux nombres : le nombre d'heures (entre 0 et 23) et le nombre de minutes (entre 0 et 59).
		
		Exemple d'entrée:
		
		digital\_clock(150)
		
		Exemple de sortie:
		
		(2, 30)
		\par
		\renewcommand{\nomfichier}{q090.py}
		\begin{solution}
		    \pythonfile{\chemincode \nomfichier}[][\nomfichier]
		\end{solution}
        
%Question 17 Supprimer Q2 site 1
\question
Question supprimée, reste la question 2 du site 1
%		\question
%		Créez une fonction nommée factorial (), qui reçoit un nombre en tant que paramètre et renvoie le factoriel de cette valeur.
%		
%		Exemple d'entrée:
%		
%		factorielle(8)
%		
%		Exemple de sortie:
%		
%		40320
%		\par
%		\renewcommand{\nomfichier}{q091.py}
%		\begin{solution}
%		    \pythonfile{\chemincode \nomfichier}[][\nomfichier]
%		\end{solution}
        
%Question 18
		\question
		Créez une fonction nommée racine(), qui reçoit un nombre en tant que paramètre et renvoie la racine carrée.
		
		Si le nombre résultant a des décimales, veuillez ne garder que les 2 premiers.
		
		Exemple d'entrée:
		
		racine(50)
		
		Exemple de sortie:
		
		7.07
		\par
		\renewcommand{\nomfichier}{q092.py}
		\begin{solution}
		    \pythonfile{\chemincode \nomfichier}[][\nomfichier]
		\end{solution}
        
%Question 19
		\question
		Créez une fonction appelée squares\_dictionary ().La fonction reçoit un nombre n et devrait générer un dictionnaire qui contient des paires de la forme (n: n * n) pour chaque nombre dans la plage de 1 à n, inclus.
		
		Imprimez le dictionnaire résultant.
		
		Exemple d'entrée:
		
		squares\_dictionary(8)
		
		Exemple de sortie:
		
		\{1: 1, 2: 4, 3: 9, 4: 16, 5: 25, 6: 36, 7: 49, 8: 64\}
		\par
		\renewcommand{\nomfichier}{q093.py}
		\begin{solution}
		    \pythonfile{\chemincode \nomfichier}[][\nomfichier]
		\end{solution}
        
%Question 20
		\question
		Créez une fonction appelée list\_and\_tuple(), qui prend en entrée n nombres et renvoie une liste et un tuple de ces nombres sous forme de chaîne.
		
		Imprimez la liste et le tuple sur deux lignes.
		
		Exemple d'entrée:
		
		list\_and\_tuple(34,67,55,33,12,98)
		
		Exemple de sortie:
		
		['34', '67', '55', '33', '12', '98']
		('34', '67', '55', '33', '12', '98')
		\par
		\renewcommand{\nomfichier}{q094.py}
		\begin{solution}
		    \pythonfile{\chemincode \nomfichier}[][\nomfichier]
		\end{solution}
        
%Question 21
		\question
Question POO
%		Définissez une classe appelée InputOutString qui a au moins deux méthodes:
%		
%		get\_string pour obtenir une chaîne à partir de l'entrée de la console.
%		print\_string pour imprimer la chaîne en majuscule.
%		
%		Testez les méthodes de votre classe.
%		\par
%		\renewcommand{\nomfichier}{q095.py}
%		\begin{solution}
%		  \pythonfile{\chemincode \nomfichier}[][\nomfichier]
%		\end{solution}

%Question 22
		\question
		Écrivez une fonction print\_formula(), avec un paramètre qui calcule et imprime la valeur en fonction de la formule donnée:
		
		Q = racine carrée de (2 * c * d) / h
		
		Voici les valeurs fixes de C et H:
		
		C est de 50.\newline
		H est 30.\newline
		D serait le paramètre de la fonction.
		
		Exemple d'entrée:
		
		print\_formula(150)
		
		Sortie:
		
		22
		\par
		\renewcommand{\nomfichier}{q096.py}
		\begin{solution}
		    \pythonfile{\chemincode \nomfichier}[][\nomfichier]
		\end{solution}
        
%Question 23
		\question
		Écrivez une fonction two\_dimensional\_list(), qui prend 2 chiffres (x, y) en entrée et génère une liste à 2 dimensions.
		
		La valeur de l'élément dans la ligne i et la colonne j doit être i * j.
		
		Exemple d'entrée:
		
		two\_dimensional\_list(3,5)
		
		Exemple de sortie:
		
		[[0, 0, 0, 0, 0], [0, 1, 2, 3, 4], [0, 2, 4, 6, 8]]
		\par
		\renewcommand{\nomfichier}{q097.py}
		\begin{solution}
		    \pythonfile{\chemincode \nomfichier}[][\nomfichier]
		\end{solution}
        
%Question 24
		\question
		Écrire une fonction sequence\_of\_words, qui accepte en entrée une séquence de mots séparés par des virgules (une chaîne).
		
    Imprimer les mots dans une séquence séparée par des virgules après les avoir triés par ordre alphabétique.
	
		Exemple d'entrée:
		
		sequence\_of\_words("sans, bonjour, sac, monde")
		
		Exemple de sortie:
		
		Sac, bonjour, sans, monde
		\par
		\renewcommand{\nomfichier}{q098.py}
		\begin{solution}
		    \pythonfile{\chemincode \nomfichier}[][\nomfichier]
		\end{solution}
        
%Question 25
		\question
		Écrire une fonction appelée remove\_duplicate\_words() qui accepte en entrée une séquence de mots séparés par des espaces et qui renvoie les mots après avoir supprimé tous les mots en double et les avoir triés par ordre alphanumérique.
		
		Exemple d'entrée:
		
		remove\_duplicate\_words("Hello World and Practice rend à nouveau parfait et bonjour le monde")
		
		Exemple de sortie:
		
		Encore une fois et bonjour fait un monde de pratique parfait
		\par
		\renewcommand{\nomfichier}{q099.py}
		\begin{solution}
		    \pythonfile{\chemincode \nomfichier}[][\nomfichier]
		\end{solution}
        
%Question 26
		\question
		Écrire une fonction divisible\_binary() qui prend en entrée une séquence de nombres binaires à 4 chiffres séparés par des virgules et vérifie s'ils sont divisibles par 5. Imprimer les nombres qui sont divisibles par 5 dans une séquence séparée par des virgules.
		
		Exemple d'entrée:
		
		divisible\_binary("1000,1100,1010,1111")
		
		Exemple de sortie:
		
		1010,1111
		\par
		\renewcommand{\nomfichier}{q100.py}
		\begin{solution}
		    \pythonfile{\chemincode \nomfichier}[][\nomfichier]
		\end{solution}
        
%Question 27
		\question
		Définir une fonction nommée all\_digits\_even() pour identifier et imprimer tous les nombres entre 1000 et 3000 (inclus) où chaque chiffre du nombre est un nombre pair. Affichez les nombres résultants dans une séquence séparée par des virgules sur une seule ligne.
		\par
		\renewcommand{\nomfichier}{q101.py}
		\begin{solution}
		    \pythonfile{\chemincode \nomfichier}[][\nomfichier]
		\end{solution}
        
%Question 28
			\question
			Écrire une fonction nommée letters\_and\_digits() qui prend une phrase en entrée et calcule le nombre de lettres et de chiffres qu'elle contient.
			
			Exemple d'entrée:
			
			letters\_and\_digits("Hello World! 123")
			
			Exemple de sortie:
			
			Lettres 10
			Chiffres 3
			\par
			\renewcommand{\nomfichier}{q102.py}
			\begin{solution}
			    \pythonfile{\chemincode \nomfichier}[][\nomfichier]
			\end{solution}
        
%Question 29
		\question
		Écrivez un programme number\_of\_uppercase() qui accepte une phrase et calcule le nombre de lettres majuscules et minuscules.
		
		Exemple d'entrée:
		
		number\_of\_uppercase("Hello World!")
		
		Exemple de sortie:
		
		Majuscule 1
		Minuscule 9
		\par
		\renewcommand{\nomfichier}{q103.py}
		\begin{solution}
		  \pythonfile{\chemincode \nomfichier}[][\nomfichier]
		\end{solution}
        
%Question 30
		\question
		Écrivez un programme computed\_value() pour calculer la somme d'un + aa + aaa + aaaa, où «a» est un chiffre donné.
		
		Exemple d'entrée:
		
		computed\_value(9)
		
		Exemple de sortie:
		
		11106
		\par
		\renewcommand{\nomfichier}{q104.py}
		\begin{solution}
		    \pythonfile{\chemincode \nomfichier}[][\nomfichier]
		\end{solution}
        
%Question 31
		\question
		Écrivez une fonction nommée square\_odd\_numbers() qui accepte en entrée une chaîne de nombres séparés par des virgules, ne met au carré que les nombres impairs et renvoie les résultats sous la forme d'une liste.
		
		Exemple d'entrée:
		
		square\_odd\_numbers("1,2,3,4,5,6,7,8,9")
		
		Exemple de sortie:
		
		[1, 9, 25, 49, 81]
		\par
		\renewcommand{\nomfichier}{q105.py}
		\begin{solution}
		  \pythonfile{\chemincode \nomfichier}[][\nomfichier]
		\end{solution}
        
%Question 32
		\question
		Écrire une fonction nommée net\_amount() qui calcule le montant net d'un compte bancaire sur la base d'un journal de transactions provenant de l'entrée. Le format du journal des transactions est le suivant :
		
		D 100\newline
		W 200
		
		D signifie dépôt tandis que w signifie le retrait.\newline
		Exemple d'entrée:
		
		net\_amount("D 300 D 300 W 200 D 100")
		
		Exemple de sortie:
		
		500
		\par
		\renewcommand{\nomfichier}{q106.py}
		\begin{solution}
		    \pythonfile{\chemincode \nomfichier}[][\nomfichier]
		\end{solution}
        
%Question 33
		\question
		Un site Web oblige les utilisateurs à saisir un nom d'utilisateur et un mot de passe pour s'inscrire.Écrivez une fonction nommée valid\_password() pour vérifier la validité de l'entrée de mot de passe par les utilisateurs.Voici les critères de vérification du mot de passe:
		
	\begin{itemize}
	\item 	Au moins 1 lettre entre [A-Z].
	\item 		Au moins 1 nombre entre [0-9].
	\item 		Au moins 1 lettre entre [A-Z].
	\item 		Au moins 1 caractère de [\$ \# @].
	\item 		Longueur minimale du mot de passe: 6.
	\item 		Longueur maximale du mot de passe: 12.
	\end{itemize}
		
		Votre programme doit accepter un mot de passe et le vérifier en fonction des critères précédents.Si le mot de passe est validé avec succès, la fonction renvoie la chaîne suivante "Mot de passe valide".Sinon, il renvoie "mot de passe non valide. Veuillez réessayer".
		Exemple d'entrée:
		
		valid\_password("ABD1234 @ 1")
		
		Exemple de sortie:
		
		"Mot de passe valide"
		\par
		\renewcommand{\nomfichier}{q107.py}
		\begin{solution}
		    \pythonfile{\chemincode \nomfichier}[][\nomfichier]
		\end{solution}
        
%Question 34
		\question
		Écrivez une fonction sort\_tuples\_ascending() pour trier les tuples (nom, âge, score) par ordre croissant, où nom, âge et score sont tous des chaînes de caractères. Les critères de tri sont :
		
		\begin{itemize}
		\item Trier basé sur le nom.
		\item Puis trier en fonction de l'âge.
		\item Puis trier par score.
		\end{itemize}
		
		La priorité est le nom> Age> Score.\newline
		Exemple d'entrée:
		
		sort\_tuples\_ascending([«Tom, 19,80», «John, 20,90», «Jony, 17,91», «Jony, 17,93», «Jason, 21,85»])
		
		Exemple de sortie:
		
		[('Jason', '21', '85'), ('John', '20', '90'), ('Jony', '17', '91'), ('Jony', '17',' 93 '), (' Tom ',' 19 ',' 80 ')]]
		\par
		\renewcommand{\nomfichier}{q108.py}
		\begin{solution}
		    \pythonfile{\chemincode \nomfichier}[][\nomfichier]
		\end{solution}
        
%Question 35
		\question
		Question POO
%		Définir une classe avec une fonction génératrice qui peut itérer les nombres qui sont divisibles par 7 entre un intervalle donné 0 et n.
%		\par
%		\renewcommand{\nomfichier}{q109.py}
%		\begin{solution}
%		    \pythonfile{\chemincode \nomfichier}[][\nomfichier]
%		\end{solution}
        
%Question 36
		\question
		Un robot se déplace dans un plan à partir du point d'origine (0,0). Le robot peut se déplacer vers le HAUT, le BAS, la GAUCHE et la DROITE avec des étapes données. La trace du mouvement du robot est présentée sous la forme d'une liste comme la suivante :
		
		["UP 5", "DOWN 3", "LEFT 3", "RIGHT 2"]
		
		Les nombres qui suivent la direction sont des pas. Veuillez écrire un programme nommé compute\_robot\_distance() pour calculer la distance finale après une séquence de mouvements à partir du point d'origine. Si la distance est un flotteur, il suffit d'imprimer l'entier le plus proche.
		Exemple d'entrée :
		
		compute\_robot\_distance(["UP 5", "DOWN 3", "LEFT 3", "RIGHT 2"])
		
		Exemple de sortie:
		
		2
		\par
		\renewcommand{\nomfichier}{q110.py}
		\begin{solution}
		    \pythonfile{\chemincode \nomfichier}[][\nomfichier]
		\end{solution}
        
%Question 37
		\question
		Écrivez une fonction appelée compute\_word\_frequency() pour calculer la fréquence des mots à partir d'une chaîne de caractères.
		
		  \begin{itemize}
		  \item Placez chaque mot séparé par un espace dans un dictionnaire et comptez sa fréquence.
		  \item Classez le dictionnaire par ordre alphanumérique et imprimez dans la console chaque clé sur une nouvelle ligne.
		  \end{itemize}
		
		Exemple d'entrée:
		
		compute\_word\_frequency("New to Python or choosing between Python 2 and Python 3? Read Python 2 or Python 3.")
		
		Exemple de sortie:
		
		2: 2\newline
		3.: 1\newline
		3?: 1\newline
		New: 1\newline
		Python: 5\newline
		Read: 1\newline
		and: 1\newline
		between: 1\newline
		choosing: 1\newline
		or: 2\newline
		to: 1
		\par
		\renewcommand{\nomfichier}{q111.py}
		\begin{solution}
		    \pythonfile{\chemincode \nomfichier}[][\nomfichier]
		\end{solution}
        
%Question 38
		\question
		Question POO
%		En Python, une classe est une structure qui permet d'organiser et d'encapsuler des données et des fonctionnalités connexes. Les classes sont une caractéristique fondamentale de la programmation orientée objet (POO), un paradigme de programmation qui utilise des objets pour modéliser et organiser le code.
%		
%		En termes simples, une classe est comme un plan ou un modèle pour créer des objets. Un objet est une instance spécifique d'une classe à laquelle sont associés des attributs (données) et des méthodes (fonctions). Les attributs représentent les caractéristiques de l'objet et les méthodes représentent les actions que l'objet peut effectuer.
%		Exemple :
%		\renewcommand{\nomfichier}{q112depart.py}
%		\pythonfile{\chemincode \nomfichier}[][\nomfichier]
%		
%		Dans ce code :
%		
%    \begin{itemize}
%    \item La classe Student possède une méthode \_\_init\_\_ pour initialiser les attributs nom, âge et classe de l'étudiant.
%    \item introduce est une méthode qui imprime un message de présentation de l'étudiant.
%    \item study est une méthode qui simule l'acte d'étudier et met à jour la note de l'étudiant.
%    \end{itemize}
%		
%		Instructions :
%		
%    Pour réaliser cet exercice, copiez le code fourni dans l'exemple et collez-le dans votre fichier. Exécutez le code et testez sa fonctionnalité. Essayez de modifier différents aspects du code pour observer son comportement. Cette approche pratique vous aidera à comprendre la structure et le comportement de la classe Étudiant. Une fois que vous serez familiarisé avec le code et ses effets, n'hésitez pas à passer à l'exercice suivant.
%		\par
%		\renewcommand{\nomfichier}{q112.py}
%		\begin{solution}
%		    \pythonfile{\chemincode \nomfichier}[][\nomfichier]
%		\end{solution}
        
%Question 39
        \question
        Question POO
%		\textbf{Méthodes \_\_init\_\_ et \_\_str\_\_}
%		
%		En général, lorsque vous travaillez avec des classes, vous rencontrez des méthodes de la forme \_\_<méthode>\_\_ ; ces méthodes sont appelées "méthodes magiques". Il en existe un grand nombre, chacune ayant un objectif spécifique. Cette fois-ci, nous nous concentrerons sur l'apprentissage de deux des méthodes les plus fondamentales.
%		
%		La méthode magique \_\_init\_\_ est essentielle pour l'initialisation des objets au sein d'une classe. Elle est automatiquement exécutée lorsqu'une nouvelle instance de la classe est créée, ce qui permet d'attribuer des valeurs initiales aux attributs de l'objet.
%		
%		La méthode \_\_str\_\_ est utilisée pour fournir une représentation sous forme de chaîne de caractères lisible de l'instance, ce qui permet de personnaliser la sortie lors de l'impression de l'objet. Cette méthode est particulièrement utile pour améliorer la lisibilité du code et faciliter le débogage, car elle définit une version conviviale des informations contenues dans l'objet.
%		
%		Exemple :
%		
%		\renewcommand{\nomfichier}{q113depart.py}
%		\pythonfile{\chemincode \nomfichier}[][\nomfichier]
%
%		Instructions:
%		
%    \begin{itemize}
%    \item Créez une classe appelée Book qui possède les méthodes \_\_init\_\_ et \_\_str\_\_.
%    \item La méthode \_\_init\_\_ doit initialiser les attributs title, author et year.
%    \item La méthode \_\_str\_\_ doit renvoyer une chaîne de caractères représentant les informations d'une instance du livre suivant de cette manière :\newline
%    book1 = ("The Great Gatsby", "F. Scott Fitzgerald", 1925)\newline
%    print(book1)\newline
%    \newline
%    \# Sortie :\newline
%    \#\newline
%    \# Title : Le Grand Gatsby\newline
%    \# Author : F. Scott Fitzgerald\newline
%    \# Year: 1925\newline
%    \end{itemize}
%    \par
%    \renewcommand{\nomfichier}{q113.py}
%    \begin{solution}
%        \pythonfile{\chemincode \nomfichier}[][\nomfichier]
%    \end{solution}
    
%Question 40
    \question
    Question POO
%		\textbf{Héritage et polymorphisme}
%		
%		Maintenant que nous avons compris ce qu'est une classe et certaines de ses caractéristiques, abordons deux nouveaux concepts liés aux classes : l'héritage et le polymorphisme. Prenons l'exemple suivant :
%		
%		\renewcommand{\nomfichier}{q114depart.py}
%		\pythonfile{\chemincode \nomfichier}[][\nomfichier]
%		
%
%		En supposant que la classe Student de l'exercice précédent soit codée juste au-dessus de la classe HighSchoolStudent, pour hériter de ses méthodes et attributs, il suffit d'inclure le nom de la classe dont nous voulons hériter (la classe mère) entre les parenthèses de la classe enfant (HighSchoolStudent). Comme vous pouvez le constater, nous pouvons maintenant utiliser la méthode introduce de la classe Student sans avoir à la coder à nouveau, ce qui rend notre code plus efficace. Il en va de même pour les attributs ; nous n'avons pas besoin de les redéfinir.
%		
%		En outre, nous avons la possibilité d'ajouter de nouvelles méthodes exclusivement pour cette classe ou même de remplacer une méthode héritée si nécessaire, comme le montre la méthode study, qui est légèrement modifiée à partir de la méthode Student ; c'est ce qu'on appelle le polymorphisme.
%		
%		\textbf{Instructions :}
%
%		\begin{itemize}
%		\item Créez une classe appelée CollegeStudent qui hérite de la classe Student déjà définie.
%		\item Ajoutez un nouvel attribut appelé major pour représenter la spécialité étudiée.
%		\item Modifiez la méthode introduce héritée pour qu'elle renvoie cette chaîne de caractères :\newline
%		"Bonjour ! Je m'appelle <nom> et je suis étudiant en <major>."
%		\item Ajoutez une nouvelle méthode appelée attend\_lecture qui renvoie la chaîne suivante :\newline
%		"<nom> assiste à une conférence pour les étudiants de <major>".
%		\item Créez une instance de votre nouvelle classe et appelez chacune de ses méthodes. 
%		\item Exécutez votre code pour vous assurer qu'il fonctionne.
%		\end{itemize}
%    \par
%    \renewcommand{\nomfichier}{q114.py}
%    \begin{solution}
%        \pythonfile{\chemincode \nomfichier}[][\nomfichier]
%    \end{solution}
    
%Question 41
		\question
		Question POO
%		\textbf{Méthodes statiques}
%		
%		Une méthode statique en Python est une méthode liée à une classe plutôt qu'à une instance de la classe. Contrairement aux méthodes ordinaires, les méthodes statiques n'ont pas accès à l'instance ou à la classe elle-même.
%		
%		Les méthodes statiques sont souvent utilisées lorsqu'une méthode particulière ne dépend pas de l'état de l'instance ou de la classe. Elles ressemblent davantage à des fonctions utilitaires associées à une classe.
%		
%		\renewcommand{\nomfichier}{q115depart.py}
%		\pythonfile{\chemincode \nomfichier}[][\nomfichier]
%		
%		Dans cet exemple:
%		
%    La méthode statique is\_adult vérifie si une personne est un adulte en fonction de son âge. Elle n'a pas accès directement aux variables d'instance ou de classe.
%
%		\textbf{Instructions :}
%
%    \begin{itemize}
%    \item Créez une classe appelée MathOperations.
%    \item Créez une méthode statique nommée add\_numbers qui prend deux nombres en paramètre et renvoie leur somme.
%    \item Créez une instance de la classe MathOperations.
%    \item Utilisez la méthode statique add\_numbers pour additionner deux nombres, par exemple 10 et 15.
%    \item Imprimez le résultat.
%    \end{itemize}
%		
%		Exemple d'entrée:
%		
%		math\_operations\_instance = MathOperations()
%		sum\_of\_numbers = MathOperations.add\_numbers(10, 15)
%		
%		Sortie:
%		
%		25
%		\par
%		\renewcommand{\nomfichier}{q115.py}
%		\begin{solution}
%		    \pythonfile{\chemincode \nomfichier}[][\nomfichier]
%		\end{solution}
        
%Question 42
        \question
        Question POO
%		\textbf{Méthodes de classe}
%		
%		Une méthode de classe est une méthode liée à la classe et non à l'instance de la classe. Elle prend comme premier paramètre la classe elle-même, souvent nommée "cls". Les méthodes de classe sont définies à l'aide du décorateur @classmethod.
%		
%		La principale caractéristique d'une méthode de classe est qu'elle peut accéder et modifier les attributs au niveau de la classe, mais qu'elle ne peut pas accéder ou modifier les attributs spécifiques à l'instance puisqu'elle n'a pas accès à une instance de la classe. Les méthodes de classe sont souvent utilisées pour des tâches qui impliquent la classe elle-même plutôt que des instances individuelles.
%		
%		\renewcommand{\nomfichier}{q116depart.py}
%		\pythonfile{\chemincode \nomfichier}[][\nomfichier]
%		
%		Dans cet exemple:
%		
%    La méthode de classe get\_total\_people renvoie le nombre total de personnes créées (instances de la classe Person).
%
%		\textbf{Instructions :}
%
%		\begin{itemize}
%		\item Créez une classe appelée MathOperations.
%		\item Dans cette classe, définissez les éléments suivants :
%			\begin{itemize}
%			\item Une variable de classe nommée pi avec une valeur de 3,14159.
%			\item Une méthode de classe nommée calculate\_circle\_area qui prend un rayon comme paramètre et renvoie l'aire d'un cercle à l'aide de la formule : $aire = \pi × rayon^2$.
%			\end{itemize}
%		\item Utilisez la méthode de classe calculate\_circle\_area pour calculer l'aire d'un cercle de rayon 5.
%		\item Imprimez le résultat. (Il n'est pas nécessaire de créer une instance)
%		\end{itemize}
%		
%		Exemple d'entrée:
%		
%		circle\_area = MathOperations.calculate\_circle\_area(5)
%		
%		Sortie:
%		
%		78.53975
%		\par
%		\renewcommand{\nomfichier}{q116.py}
%		\begin{solution}
%		    \pythonfile{\chemincode \nomfichier}[][\nomfichier]
%		\end{solution}
        


\end{questions}

%*********************************************
\section{Site 2}

\href{https://github.com/4GeeksAcademy/master-python-programming-exercises/tree/master}{Lien vers le site d'origine}\par
\begin{questions}


%Question 1
\question
	Écrivez une fonction \textbf{precedent\_suivant()} qui lit un numéro entier et renvoie ses numéros précédents et suivants.
	
	Exemple d'entrée:
	
	precedent\_suivant(179)
	
	Exemple de sortie:
	
	(178, 180)
  \par
  \renewcommand{\nomfichier}{q075.py}
  \begin{solution}
      \pythonfile{\chemincode \nomfichier}[][\nomfichier]
  \end{solution}       
%Question 2
  \question
  N étudiants prennent K pommes et les distribuent entre eux uniformément.La partie restante (indivisible) reste dans le panier.Combien de pommes aura chaque étudiante et combien resteront dans le panier ?
  
  La fonction lit les nombres n et k et renvoie les deux réponses pour les questions ci-dessus.

	Exemple d'entrée:
	
	Apple\_sharing(6, 50)
	
	Exemple de sortie:
	
	(8, 2)
  \par
  \renewcommand{\nomfichier}{q076.py}
  \begin{solution}
      \pythonfile{\chemincode \nomfichier}[][\nomfichier]
  \end{solution}
    
%Question 3
  \question
  Écrivez une fonction appelée \textbf{carre()} qui calcule la valeur du carré d'un nombre.

	Exemple d'entrée:
	
	carre(6)
	
	Exemple de sortie:
	
	36
  \par
  \renewcommand{\nomfichier}{q077.py}
  \begin{solution}
      \pythonfile{\chemincode \nomfichier}[][\nomfichier]
  \end{solution}
        
%Question 4
  \question
  Écrire la fonction \textbf{heures\_minutes()} pour transformer le nombre donné en secondes en heures et minutes.
	
	Exemple 1:
	
	heures\_minutes(3900)\newline
	sortie : (1, 5)
	
	Exemple 2:
	
	heures\_minutes(60)\newline
	sortie : (0, 1)
        \par
        \renewcommand{\nomfichier}{q078.py}
        \begin{solution}
            \pythonfile{\chemincode \nomfichier}[][\nomfichier]
        \end{solution}
        
%Question 5
    \question
    Étant donné deux horodatages du même jour.
    Chaque horodatage est représenté par un nombre :
    \begin{itemize}
    \item d'heures
    \item de minutes
    \item de secondes
    \end{itemize}
    
    L'instant du premier horodatage s'est produit avant l'instant du second. Calculez le nombre de secondes qui se sont écoulées entre les deux.

		Exemple 1:
		
		two\_timestamp(1,1,1,2,2,2)\newline
		Sortie : 3661
		
		Exemple 2:
		
		two\_timestamp(1,2,30,1,3,20)\newline
		Sortie : 50
        \par
        \renewcommand{\nomfichier}{q079.py}
        \begin{solution}
            \pythonfile{\chemincode \nomfichier}[][\nomfichier]
        \end{solution}
        
%Question 6
    \question
    Créez une fonction nommée two\_digits().
    
    Étant donné un entier à deux chiffres, two\_digits() renvoie son chiffre gauche (le chiffre des dizaines) puis son chiffre droit (le chiffre des unités).

		Exemple d'entrée:
		
		two\_digits(79)
		
		Exemple de sortie:
		
		(7, 9)
    \par
    \renewcommand{\nomfichier}{q080.py}
    \begin{solution}
        \pythonfile{\chemincode \nomfichier}[][\nomfichier]
    \end{solution}
        
%Question 7
    \question
    Écrire la fonction nommée swap\_digits().
    
    Étant donné un entier à deux chiffres, swap\_digits() échange ses chiffres et imprimez le résultat.

		Exemple d'entrée:
		
		swap\_digits(79)
		
		Exemple de sortie:
		
		97
    \par
    \renewcommand{\nomfichier}{q081.py}
    \begin{solution}
        \pythonfile{\chemincode \nomfichier}[][\nomfichier]
    \end{solution}
        
%Question 8
    \question
    Écrire la fonction last\_two\_digits().Étant donné un entier supérieur à 9, last\_two\_digits() imprime ses deux derniers chiffres.

		Exemple d'entrée:
		
		last\_two\_digits(1234)
		
		Exemple de sortie:
		
		34
    \par
    \renewcommand{\nomfichier}{q082.py}
    \begin{solution}
        \pythonfile{\chemincode \nomfichier}[][\nomfichier]
    \end{solution}
        
%Question 9
    \question
    Écrire la fonction tens\_digit().
    
    Étant donné un entier, tens\_digit() renvoie son chiffre de dizaines.

		Exemple 1:
		
		tens\_digit(1234)\newline
		Sortie : 3
		
		Exemple 2:
		
		tens\_digit(179)\newline
		Sortie : 7
    \par
    \renewcommand{\nomfichier}{q083.py}
    \begin{solution}
        \pythonfile{\chemincode \nomfichier}[][\nomfichier]
    \end{solution}
        
%Question 10
    \question
    Écrire la fonction digits\_sum().
    
    Étant donné un numéro à trois chiffres, digits\_sum() trouve la somme de ses chiffres.

		Exemple d'entrée:
		
		digits\_sum(123)
		
		Exemple de sortie:
		
		6
    \par
    \renewcommand{\nomfichier}{q084.py}
    \begin{solution}
        \pythonfile{\chemincode \nomfichier}[][\nomfichier]
    \end{solution}
        
%Question 11
    \question
    Écrire la fonction first\_digit(). Étant donné un nombre réel positif, first\_digit() renvoie son premier chiffre (à droite de la virgule).

		Exemple d'entrée:
		
		first\_digit(1.79)
		
		Exemple de sortie:
		
		7
    \par
    \renewcommand{\nomfichier}{q085.py}
    \begin{solution}
        \pythonfile{\chemincode \nomfichier}[][\nomfichier]
    \end{solution}
        
%Question 12
		\question
		Une voiture peut parcourir une distance de N kilomètres par jour. Combien de jours lui faudra-t-il pour parcourir un itinéraire d'une longueur de M kilomètres ?
		Instructions :
		
    Écrire une fonction car\_route() qui prend deux arguments :
    \begin{itemize}
	    \item la distance qu'elle peut parcourir en un jour
	    \item la distance à parcourir
    \end{itemize}
    
    Cette fonction calcule le nombre de jours qu'il faudra pour parcourir cette distance.
		
		Exemple d'entrée:
		
		car\_route(20, 40)
		
		Exemple de sortie:
		
		2
    \par
    \renewcommand{\nomfichier}{q086.py}
    \begin{solution}
        \pythonfile{\chemincode \nomfichier}[][\nomfichier]
    \end{solution}
    
%Question 13
		\question
		Écrivez une fonction century().
		Cette dernière prend une année en paramètre sous la forme d'un entier et renvoi le numéro du siècle.
		
		Exemple d'entrée:
		
		century(2001)
		
		Exemple de sortie:
		
		21
		\par
		\renewcommand{\nomfichier}{q087.py}
		\begin{solution}
		    \pythonfile{\chemincode \nomfichier}[][\nomfichier]
		\end{solution}
        
%Question 14
		\question
		Un petit gâteau coûte d euros et c centimes. Écrivez une fonction qui détermine le nombre d'euros et de centimes qu'une personne devrait payer pour n petits gâteaux. La fonction reçoit trois nombres : d, c, n et doit renvoyer deux nombres : le coût total en euros et en centimes.
		
		Exemple d'entrée:
		
		total\_cost(15, 22, 4)
		
		Sortie :
		
		(60, 88)
		\par
		\renewcommand{\nomfichier}{q088.py}
		\begin{solution}
		    \pythonfile{\chemincode \nomfichier}[][\nomfichier]
		\end{solution}
        
%Question 15
		\question
		Écrire une fonction day\_of\_week(). On lui fourni un entier k compris entre 1 et 365, la fonction day\_of\_week() trouve le numéro du jour de la semaine pour le k-ième jour de l'année, à condition que le 1er janvier de cette année soit un jeudi.
		
		Les jours de la semaine sont numérotés comme:
		
		\begin{enumerate}
		\item[0] Dimanche
		\item[1] Lundi
		\item[2] Mardi ...
		
		\item[6] Samedi
		\end{enumerate}
		
		Exemple d'entrée:
		
		day\_of\_week(1)
		
		Exemple de sortie:
		
		4
		\par
		\renewcommand{\nomfichier}{q089.py}
		\begin{solution}
		    \pythonfile{\chemincode \nomfichier}[][\nomfichier]
		\end{solution}
        
%Question 16
		\question
		Soit l'entier n - le nombre de minutes qui se sont écoulées depuis minuit, combien d'heures et de minutes sont affichées sur l'horloge numérique de 24 heures ? Écrivez une fonction digital\_clock() pour le calculer. La fonction doit afficher deux nombres : le nombre d'heures (entre 0 et 23) et le nombre de minutes (entre 0 et 59).
		
		Exemple d'entrée:
		
		digital\_clock(150)
		
		Exemple de sortie:
		
		(2, 30)
		\par
		\renewcommand{\nomfichier}{q090.py}
		\begin{solution}
		    \pythonfile{\chemincode \nomfichier}[][\nomfichier]
		\end{solution}
        
%Question 17 Supprimer Q2 site 1
\question
Question supprimée, reste la question 2 du site 1
%		\question
%		Créez une fonction nommée factorial (), qui reçoit un nombre en tant que paramètre et renvoie le factoriel de cette valeur.
%		
%		Exemple d'entrée:
%		
%		factorielle(8)
%		
%		Exemple de sortie:
%		
%		40320
%		\par
%		\renewcommand{\nomfichier}{q091.py}
%		\begin{solution}
%		    \pythonfile{\chemincode \nomfichier}[][\nomfichier]
%		\end{solution}
        
%Question 18
		\question
		Créez une fonction nommée racine(), qui reçoit un nombre en tant que paramètre et renvoie la racine carrée.
		
		Si le nombre résultant a des décimales, veuillez ne garder que les 2 premiers.
		
		Exemple d'entrée:
		
		racine(50)
		
		Exemple de sortie:
		
		7.07
		\par
		\renewcommand{\nomfichier}{q092.py}
		\begin{solution}
		    \pythonfile{\chemincode \nomfichier}[][\nomfichier]
		\end{solution}
        
%Question 19
		\question
		Créez une fonction appelée squares\_dictionary ().La fonction reçoit un nombre n et devrait générer un dictionnaire qui contient des paires de la forme (n: n * n) pour chaque nombre dans la plage de 1 à n, inclus.
		
		Imprimez le dictionnaire résultant.
		
		Exemple d'entrée:
		
		squares\_dictionary(8)
		
		Exemple de sortie:
		
		\{1: 1, 2: 4, 3: 9, 4: 16, 5: 25, 6: 36, 7: 49, 8: 64\}
		\par
		\renewcommand{\nomfichier}{q093.py}
		\begin{solution}
		    \pythonfile{\chemincode \nomfichier}[][\nomfichier]
		\end{solution}
        
%Question 20
		\question
		Créez une fonction appelée list\_and\_tuple(), qui prend en entrée n nombres et renvoie une liste et un tuple de ces nombres sous forme de chaîne.
		
		Imprimez la liste et le tuple sur deux lignes.
		
		Exemple d'entrée:
		
		list\_and\_tuple(34,67,55,33,12,98)
		
		Exemple de sortie:
		
		['34', '67', '55', '33', '12', '98']
		('34', '67', '55', '33', '12', '98')
		\par
		\renewcommand{\nomfichier}{q094.py}
		\begin{solution}
		    \pythonfile{\chemincode \nomfichier}[][\nomfichier]
		\end{solution}
        
%Question 21
		\question
Question POO
%		Définissez une classe appelée InputOutString qui a au moins deux méthodes:
%		
%		get\_string pour obtenir une chaîne à partir de l'entrée de la console.
%		print\_string pour imprimer la chaîne en majuscule.
%		
%		Testez les méthodes de votre classe.
%		\par
%		\renewcommand{\nomfichier}{q095.py}
%		\begin{solution}
%		  \pythonfile{\chemincode \nomfichier}[][\nomfichier]
%		\end{solution}

%Question 22
		\question
		Écrivez une fonction print\_formula(), avec un paramètre qui calcule et imprime la valeur en fonction de la formule donnée:
		
		Q = racine carrée de (2 * c * d) / h
		
		Voici les valeurs fixes de C et H:
		
		C est de 50.\newline
		H est 30.\newline
		D serait le paramètre de la fonction.
		
		Exemple d'entrée:
		
		print\_formula(150)
		
		Sortie:
		
		22
		\par
		\renewcommand{\nomfichier}{q096.py}
		\begin{solution}
		    \pythonfile{\chemincode \nomfichier}[][\nomfichier]
		\end{solution}
        
%Question 23
		\question
		Écrivez une fonction two\_dimensional\_list(), qui prend 2 chiffres (x, y) en entrée et génère une liste à 2 dimensions.
		
		La valeur de l'élément dans la ligne i et la colonne j doit être i * j.
		
		Exemple d'entrée:
		
		two\_dimensional\_list(3,5)
		
		Exemple de sortie:
		
		[[0, 0, 0, 0, 0], [0, 1, 2, 3, 4], [0, 2, 4, 6, 8]]
		\par
		\renewcommand{\nomfichier}{q097.py}
		\begin{solution}
		    \pythonfile{\chemincode \nomfichier}[][\nomfichier]
		\end{solution}
        
%Question 24
		\question
		Écrire une fonction sequence\_of\_words, qui accepte en entrée une séquence de mots séparés par des virgules (une chaîne).
		
    Imprimer les mots dans une séquence séparée par des virgules après les avoir triés par ordre alphabétique.
	
		Exemple d'entrée:
		
		sequence\_of\_words("sans, bonjour, sac, monde")
		
		Exemple de sortie:
		
		Sac, bonjour, sans, monde
		\par
		\renewcommand{\nomfichier}{q098.py}
		\begin{solution}
		    \pythonfile{\chemincode \nomfichier}[][\nomfichier]
		\end{solution}
        
%Question 25
		\question
		Écrire une fonction appelée remove\_duplicate\_words() qui accepte en entrée une séquence de mots séparés par des espaces et qui renvoie les mots après avoir supprimé tous les mots en double et les avoir triés par ordre alphanumérique.
		
		Exemple d'entrée:
		
		remove\_duplicate\_words("Hello World and Practice rend à nouveau parfait et bonjour le monde")
		
		Exemple de sortie:
		
		Encore une fois et bonjour fait un monde de pratique parfait
		\par
		\renewcommand{\nomfichier}{q099.py}
		\begin{solution}
		    \pythonfile{\chemincode \nomfichier}[][\nomfichier]
		\end{solution}
        
%Question 26
		\question
		Écrire une fonction divisible\_binary() qui prend en entrée une séquence de nombres binaires à 4 chiffres séparés par des virgules et vérifie s'ils sont divisibles par 5. Imprimer les nombres qui sont divisibles par 5 dans une séquence séparée par des virgules.
		
		Exemple d'entrée:
		
		divisible\_binary("1000,1100,1010,1111")
		
		Exemple de sortie:
		
		1010,1111
		\par
		\renewcommand{\nomfichier}{q100.py}
		\begin{solution}
		    \pythonfile{\chemincode \nomfichier}[][\nomfichier]
		\end{solution}
        
%Question 27
		\question
		Définir une fonction nommée all\_digits\_even() pour identifier et imprimer tous les nombres entre 1000 et 3000 (inclus) où chaque chiffre du nombre est un nombre pair. Affichez les nombres résultants dans une séquence séparée par des virgules sur une seule ligne.
		\par
		\renewcommand{\nomfichier}{q101.py}
		\begin{solution}
		    \pythonfile{\chemincode \nomfichier}[][\nomfichier]
		\end{solution}
        
%Question 28
			\question
			Écrire une fonction nommée letters\_and\_digits() qui prend une phrase en entrée et calcule le nombre de lettres et de chiffres qu'elle contient.
			
			Exemple d'entrée:
			
			letters\_and\_digits("Hello World! 123")
			
			Exemple de sortie:
			
			Lettres 10
			Chiffres 3
			\par
			\renewcommand{\nomfichier}{q102.py}
			\begin{solution}
			    \pythonfile{\chemincode \nomfichier}[][\nomfichier]
			\end{solution}
        
%Question 29
		\question
		Écrivez un programme number\_of\_uppercase() qui accepte une phrase et calcule le nombre de lettres majuscules et minuscules.
		
		Exemple d'entrée:
		
		number\_of\_uppercase("Hello World!")
		
		Exemple de sortie:
		
		Majuscule 1
		Minuscule 9
		\par
		\renewcommand{\nomfichier}{q103.py}
		\begin{solution}
		  \pythonfile{\chemincode \nomfichier}[][\nomfichier]
		\end{solution}
        
%Question 30
		\question
		Écrivez un programme computed\_value() pour calculer la somme d'un + aa + aaa + aaaa, où «a» est un chiffre donné.
		
		Exemple d'entrée:
		
		computed\_value(9)
		
		Exemple de sortie:
		
		11106
		\par
		\renewcommand{\nomfichier}{q104.py}
		\begin{solution}
		    \pythonfile{\chemincode \nomfichier}[][\nomfichier]
		\end{solution}
        
%Question 31
		\question
		Écrivez une fonction nommée square\_odd\_numbers() qui accepte en entrée une chaîne de nombres séparés par des virgules, ne met au carré que les nombres impairs et renvoie les résultats sous la forme d'une liste.
		
		Exemple d'entrée:
		
		square\_odd\_numbers("1,2,3,4,5,6,7,8,9")
		
		Exemple de sortie:
		
		[1, 9, 25, 49, 81]
		\par
		\renewcommand{\nomfichier}{q105.py}
		\begin{solution}
		  \pythonfile{\chemincode \nomfichier}[][\nomfichier]
		\end{solution}
        
%Question 32
		\question
		Écrire une fonction nommée net\_amount() qui calcule le montant net d'un compte bancaire sur la base d'un journal de transactions provenant de l'entrée. Le format du journal des transactions est le suivant :
		
		D 100\newline
		W 200
		
		D signifie dépôt tandis que w signifie le retrait.\newline
		Exemple d'entrée:
		
		net\_amount("D 300 D 300 W 200 D 100")
		
		Exemple de sortie:
		
		500
		\par
		\renewcommand{\nomfichier}{q106.py}
		\begin{solution}
		    \pythonfile{\chemincode \nomfichier}[][\nomfichier]
		\end{solution}
        
%Question 33
		\question
		Un site Web oblige les utilisateurs à saisir un nom d'utilisateur et un mot de passe pour s'inscrire.Écrivez une fonction nommée valid\_password() pour vérifier la validité de l'entrée de mot de passe par les utilisateurs.Voici les critères de vérification du mot de passe:
		
	\begin{itemize}
	\item 	Au moins 1 lettre entre [A-Z].
	\item 		Au moins 1 nombre entre [0-9].
	\item 		Au moins 1 lettre entre [A-Z].
	\item 		Au moins 1 caractère de [\$ \# @].
	\item 		Longueur minimale du mot de passe: 6.
	\item 		Longueur maximale du mot de passe: 12.
	\end{itemize}
		
		Votre programme doit accepter un mot de passe et le vérifier en fonction des critères précédents.Si le mot de passe est validé avec succès, la fonction renvoie la chaîne suivante "Mot de passe valide".Sinon, il renvoie "mot de passe non valide. Veuillez réessayer".
		Exemple d'entrée:
		
		valid\_password("ABD1234 @ 1")
		
		Exemple de sortie:
		
		"Mot de passe valide"
		\par
		\renewcommand{\nomfichier}{q107.py}
		\begin{solution}
		    \pythonfile{\chemincode \nomfichier}[][\nomfichier]
		\end{solution}
        
%Question 34
		\question
		Écrivez une fonction sort\_tuples\_ascending() pour trier les tuples (nom, âge, score) par ordre croissant, où nom, âge et score sont tous des chaînes de caractères. Les critères de tri sont :
		
		\begin{itemize}
		\item Trier basé sur le nom.
		\item Puis trier en fonction de l'âge.
		\item Puis trier par score.
		\end{itemize}
		
		La priorité est le nom> Age> Score.\newline
		Exemple d'entrée:
		
		sort\_tuples\_ascending([«Tom, 19,80», «John, 20,90», «Jony, 17,91», «Jony, 17,93», «Jason, 21,85»])
		
		Exemple de sortie:
		
		[('Jason', '21', '85'), ('John', '20', '90'), ('Jony', '17', '91'), ('Jony', '17',' 93 '), (' Tom ',' 19 ',' 80 ')]]
		\par
		\renewcommand{\nomfichier}{q108.py}
		\begin{solution}
		    \pythonfile{\chemincode \nomfichier}[][\nomfichier]
		\end{solution}
        
%Question 35
		\question
		Question POO
%		Définir une classe avec une fonction génératrice qui peut itérer les nombres qui sont divisibles par 7 entre un intervalle donné 0 et n.
%		\par
%		\renewcommand{\nomfichier}{q109.py}
%		\begin{solution}
%		    \pythonfile{\chemincode \nomfichier}[][\nomfichier]
%		\end{solution}
        
%Question 36
		\question
		Un robot se déplace dans un plan à partir du point d'origine (0,0). Le robot peut se déplacer vers le HAUT, le BAS, la GAUCHE et la DROITE avec des étapes données. La trace du mouvement du robot est présentée sous la forme d'une liste comme la suivante :
		
		["UP 5", "DOWN 3", "LEFT 3", "RIGHT 2"]
		
		Les nombres qui suivent la direction sont des pas. Veuillez écrire un programme nommé compute\_robot\_distance() pour calculer la distance finale après une séquence de mouvements à partir du point d'origine. Si la distance est un flotteur, il suffit d'imprimer l'entier le plus proche.
		Exemple d'entrée :
		
		compute\_robot\_distance(["UP 5", "DOWN 3", "LEFT 3", "RIGHT 2"])
		
		Exemple de sortie:
		
		2
		\par
		\renewcommand{\nomfichier}{q110.py}
		\begin{solution}
		    \pythonfile{\chemincode \nomfichier}[][\nomfichier]
		\end{solution}
        
%Question 37
		\question
		Écrivez une fonction appelée compute\_word\_frequency() pour calculer la fréquence des mots à partir d'une chaîne de caractères.
		
		  \begin{itemize}
		  \item Placez chaque mot séparé par un espace dans un dictionnaire et comptez sa fréquence.
		  \item Classez le dictionnaire par ordre alphanumérique et imprimez dans la console chaque clé sur une nouvelle ligne.
		  \end{itemize}
		
		Exemple d'entrée:
		
		compute\_word\_frequency("New to Python or choosing between Python 2 and Python 3? Read Python 2 or Python 3.")
		
		Exemple de sortie:
		
		2: 2\newline
		3.: 1\newline
		3?: 1\newline
		New: 1\newline
		Python: 5\newline
		Read: 1\newline
		and: 1\newline
		between: 1\newline
		choosing: 1\newline
		or: 2\newline
		to: 1
		\par
		\renewcommand{\nomfichier}{q111.py}
		\begin{solution}
		    \pythonfile{\chemincode \nomfichier}[][\nomfichier]
		\end{solution}
        
%Question 38
		\question
		Question POO
%		En Python, une classe est une structure qui permet d'organiser et d'encapsuler des données et des fonctionnalités connexes. Les classes sont une caractéristique fondamentale de la programmation orientée objet (POO), un paradigme de programmation qui utilise des objets pour modéliser et organiser le code.
%		
%		En termes simples, une classe est comme un plan ou un modèle pour créer des objets. Un objet est une instance spécifique d'une classe à laquelle sont associés des attributs (données) et des méthodes (fonctions). Les attributs représentent les caractéristiques de l'objet et les méthodes représentent les actions que l'objet peut effectuer.
%		Exemple :
%		\renewcommand{\nomfichier}{q112depart.py}
%		\pythonfile{\chemincode \nomfichier}[][\nomfichier]
%		
%		Dans ce code :
%		
%    \begin{itemize}
%    \item La classe Student possède une méthode \_\_init\_\_ pour initialiser les attributs nom, âge et classe de l'étudiant.
%    \item introduce est une méthode qui imprime un message de présentation de l'étudiant.
%    \item study est une méthode qui simule l'acte d'étudier et met à jour la note de l'étudiant.
%    \end{itemize}
%		
%		Instructions :
%		
%    Pour réaliser cet exercice, copiez le code fourni dans l'exemple et collez-le dans votre fichier. Exécutez le code et testez sa fonctionnalité. Essayez de modifier différents aspects du code pour observer son comportement. Cette approche pratique vous aidera à comprendre la structure et le comportement de la classe Étudiant. Une fois que vous serez familiarisé avec le code et ses effets, n'hésitez pas à passer à l'exercice suivant.
%		\par
%		\renewcommand{\nomfichier}{q112.py}
%		\begin{solution}
%		    \pythonfile{\chemincode \nomfichier}[][\nomfichier]
%		\end{solution}
        
%Question 39
        \question
        Question POO
%		\textbf{Méthodes \_\_init\_\_ et \_\_str\_\_}
%		
%		En général, lorsque vous travaillez avec des classes, vous rencontrez des méthodes de la forme \_\_<méthode>\_\_ ; ces méthodes sont appelées "méthodes magiques". Il en existe un grand nombre, chacune ayant un objectif spécifique. Cette fois-ci, nous nous concentrerons sur l'apprentissage de deux des méthodes les plus fondamentales.
%		
%		La méthode magique \_\_init\_\_ est essentielle pour l'initialisation des objets au sein d'une classe. Elle est automatiquement exécutée lorsqu'une nouvelle instance de la classe est créée, ce qui permet d'attribuer des valeurs initiales aux attributs de l'objet.
%		
%		La méthode \_\_str\_\_ est utilisée pour fournir une représentation sous forme de chaîne de caractères lisible de l'instance, ce qui permet de personnaliser la sortie lors de l'impression de l'objet. Cette méthode est particulièrement utile pour améliorer la lisibilité du code et faciliter le débogage, car elle définit une version conviviale des informations contenues dans l'objet.
%		
%		Exemple :
%		
%		\renewcommand{\nomfichier}{q113depart.py}
%		\pythonfile{\chemincode \nomfichier}[][\nomfichier]
%
%		Instructions:
%		
%    \begin{itemize}
%    \item Créez une classe appelée Book qui possède les méthodes \_\_init\_\_ et \_\_str\_\_.
%    \item La méthode \_\_init\_\_ doit initialiser les attributs title, author et year.
%    \item La méthode \_\_str\_\_ doit renvoyer une chaîne de caractères représentant les informations d'une instance du livre suivant de cette manière :\newline
%    book1 = ("The Great Gatsby", "F. Scott Fitzgerald", 1925)\newline
%    print(book1)\newline
%    \newline
%    \# Sortie :\newline
%    \#\newline
%    \# Title : Le Grand Gatsby\newline
%    \# Author : F. Scott Fitzgerald\newline
%    \# Year: 1925\newline
%    \end{itemize}
%    \par
%    \renewcommand{\nomfichier}{q113.py}
%    \begin{solution}
%        \pythonfile{\chemincode \nomfichier}[][\nomfichier]
%    \end{solution}
    
%Question 40
    \question
    Question POO
%		\textbf{Héritage et polymorphisme}
%		
%		Maintenant que nous avons compris ce qu'est une classe et certaines de ses caractéristiques, abordons deux nouveaux concepts liés aux classes : l'héritage et le polymorphisme. Prenons l'exemple suivant :
%		
%		\renewcommand{\nomfichier}{q114depart.py}
%		\pythonfile{\chemincode \nomfichier}[][\nomfichier]
%		
%
%		En supposant que la classe Student de l'exercice précédent soit codée juste au-dessus de la classe HighSchoolStudent, pour hériter de ses méthodes et attributs, il suffit d'inclure le nom de la classe dont nous voulons hériter (la classe mère) entre les parenthèses de la classe enfant (HighSchoolStudent). Comme vous pouvez le constater, nous pouvons maintenant utiliser la méthode introduce de la classe Student sans avoir à la coder à nouveau, ce qui rend notre code plus efficace. Il en va de même pour les attributs ; nous n'avons pas besoin de les redéfinir.
%		
%		En outre, nous avons la possibilité d'ajouter de nouvelles méthodes exclusivement pour cette classe ou même de remplacer une méthode héritée si nécessaire, comme le montre la méthode study, qui est légèrement modifiée à partir de la méthode Student ; c'est ce qu'on appelle le polymorphisme.
%		
%		\textbf{Instructions :}
%
%		\begin{itemize}
%		\item Créez une classe appelée CollegeStudent qui hérite de la classe Student déjà définie.
%		\item Ajoutez un nouvel attribut appelé major pour représenter la spécialité étudiée.
%		\item Modifiez la méthode introduce héritée pour qu'elle renvoie cette chaîne de caractères :\newline
%		"Bonjour ! Je m'appelle <nom> et je suis étudiant en <major>."
%		\item Ajoutez une nouvelle méthode appelée attend\_lecture qui renvoie la chaîne suivante :\newline
%		"<nom> assiste à une conférence pour les étudiants de <major>".
%		\item Créez une instance de votre nouvelle classe et appelez chacune de ses méthodes. 
%		\item Exécutez votre code pour vous assurer qu'il fonctionne.
%		\end{itemize}
%    \par
%    \renewcommand{\nomfichier}{q114.py}
%    \begin{solution}
%        \pythonfile{\chemincode \nomfichier}[][\nomfichier]
%    \end{solution}
    
%Question 41
		\question
		Question POO
%		\textbf{Méthodes statiques}
%		
%		Une méthode statique en Python est une méthode liée à une classe plutôt qu'à une instance de la classe. Contrairement aux méthodes ordinaires, les méthodes statiques n'ont pas accès à l'instance ou à la classe elle-même.
%		
%		Les méthodes statiques sont souvent utilisées lorsqu'une méthode particulière ne dépend pas de l'état de l'instance ou de la classe. Elles ressemblent davantage à des fonctions utilitaires associées à une classe.
%		
%		\renewcommand{\nomfichier}{q115depart.py}
%		\pythonfile{\chemincode \nomfichier}[][\nomfichier]
%		
%		Dans cet exemple:
%		
%    La méthode statique is\_adult vérifie si une personne est un adulte en fonction de son âge. Elle n'a pas accès directement aux variables d'instance ou de classe.
%
%		\textbf{Instructions :}
%
%    \begin{itemize}
%    \item Créez une classe appelée MathOperations.
%    \item Créez une méthode statique nommée add\_numbers qui prend deux nombres en paramètre et renvoie leur somme.
%    \item Créez une instance de la classe MathOperations.
%    \item Utilisez la méthode statique add\_numbers pour additionner deux nombres, par exemple 10 et 15.
%    \item Imprimez le résultat.
%    \end{itemize}
%		
%		Exemple d'entrée:
%		
%		math\_operations\_instance = MathOperations()
%		sum\_of\_numbers = MathOperations.add\_numbers(10, 15)
%		
%		Sortie:
%		
%		25
%		\par
%		\renewcommand{\nomfichier}{q115.py}
%		\begin{solution}
%		    \pythonfile{\chemincode \nomfichier}[][\nomfichier]
%		\end{solution}
        
%Question 42
        \question
        Question POO
%		\textbf{Méthodes de classe}
%		
%		Une méthode de classe est une méthode liée à la classe et non à l'instance de la classe. Elle prend comme premier paramètre la classe elle-même, souvent nommée "cls". Les méthodes de classe sont définies à l'aide du décorateur @classmethod.
%		
%		La principale caractéristique d'une méthode de classe est qu'elle peut accéder et modifier les attributs au niveau de la classe, mais qu'elle ne peut pas accéder ou modifier les attributs spécifiques à l'instance puisqu'elle n'a pas accès à une instance de la classe. Les méthodes de classe sont souvent utilisées pour des tâches qui impliquent la classe elle-même plutôt que des instances individuelles.
%		
%		\renewcommand{\nomfichier}{q116depart.py}
%		\pythonfile{\chemincode \nomfichier}[][\nomfichier]
%		
%		Dans cet exemple:
%		
%    La méthode de classe get\_total\_people renvoie le nombre total de personnes créées (instances de la classe Person).
%
%		\textbf{Instructions :}
%
%		\begin{itemize}
%		\item Créez une classe appelée MathOperations.
%		\item Dans cette classe, définissez les éléments suivants :
%			\begin{itemize}
%			\item Une variable de classe nommée pi avec une valeur de 3,14159.
%			\item Une méthode de classe nommée calculate\_circle\_area qui prend un rayon comme paramètre et renvoie l'aire d'un cercle à l'aide de la formule : $aire = \pi × rayon^2$.
%			\end{itemize}
%		\item Utilisez la méthode de classe calculate\_circle\_area pour calculer l'aire d'un cercle de rayon 5.
%		\item Imprimez le résultat. (Il n'est pas nécessaire de créer une instance)
%		\end{itemize}
%		
%		Exemple d'entrée:
%		
%		circle\_area = MathOperations.calculate\_circle\_area(5)
%		
%		Sortie:
%		
%		78.53975
%		\par
%		\renewcommand{\nomfichier}{q116.py}
%		\begin{solution}
%		    \pythonfile{\chemincode \nomfichier}[][\nomfichier]
%		\end{solution}
        


\end{questions}
%*********************************************
\section{Site 3}
\href{https://github.com/dbojado/python-exercises}{Lien vers le site d'origine}\par
\begin{questions}

\renewcommand{\chemincode}{../../code/}
%Question 1
\question
Définir une fonction nommée is\_two. Elle doit accepter une entrée et retourner True si l'entrée passée est soit le nombre, soit la chaîne 2, False sinon.

\renewcommand{\nomfichier}{q200.py}
\begin{solution}
    \pythonfile{\chemincode \nomfichier}[][\nomfichier]
\end{solution}

%Question 2
\question
Définir une fonction nommée is\_vowel. Elle doit renvoyer True si la chaîne passée est une voyelle, False sinon.

\renewcommand{\nomfichier}{q201.py}
\begin{solution}
    \pythonfile{\chemincode \nomfichier}[][\nomfichier]
\end{solution}

%Question 3
\question
Définir une fonction nommée is\_consonant. Elle doit retourner True si la chaîne passée est une consonne, False sinon. Utilisez votre fonction is\_vowel

\renewcommand{\nomfichier}{q202.py}
\begin{solution}
    \pythonfile{\chemincode \nomfichier}[][\nomfichier]
\end{solution}

%Question 4
\question
Définir une fonction qui accepte une chaîne de caractères qui est un mot. La fonction doit mettre en majuscule la première lettre du mot si celui-ci commence par une consonne.


\renewcommand{\nomfichier}{q203.py}
\begin{solution}
    \pythonfile{\chemincode \nomfichier}[][\nomfichier]
\end{solution}


%Question 5
\question
Définissez une fonction nommée calculate\_tip. Elle doit accepter un pourcentage de pourboire (un nombre entre 0 et 1) et le total de l'addition, et renvoyer le montant du pourboire.


\renewcommand{\nomfichier}{q204.py}
\begin{solution}
    \pythonfile{\chemincode \nomfichier}[][\nomfichier]
\end{solution}

%Question 6
\question
Définissez une fonction nommée apply\_discount. Elle doit accepter un prix d'origine et un pourcentage de remise, et renvoyer le prix après la remise.


\renewcommand{\nomfichier}{q205.py}
\begin{solution}
    \pythonfile{\chemincode \nomfichier}[][\nomfichier]
\end{solution}

%Question 7
\question
Définissez une fonction nommée handle\_commas. Elle doit accepter en entrée une chaîne de caractères qui est un nombre contenant des virgules, et retourner un nombre en sortie.

\renewcommand{\nomfichier}{q206.py}
\begin{solution}
    \pythonfile{\chemincode \nomfichier}[][\nomfichier]
\end{solution}

%Question 8
\question
Définissez une fonction nommée get\_letter\_grade. Elle doit accepter un nombre et retourner la lettre associée à ce nombre (A-F).

\renewcommand{\nomfichier}{q207.py}
\begin{solution}
    \pythonfile{\chemincode \nomfichier}[][\nomfichier]
\end{solution}

%Question 9
\question
Définissez une fonction nommée remove\_vowels qui accepte une chaîne et renvoie une chaîne dont toutes les voyelles ont été supprimées.

\renewcommand{\nomfichier}{q208.py}
\begin{solution}
    \pythonfile{\chemincode \nomfichier}[][\nomfichier]
\end{solution}

%Question 10
\question
Définissez une fonction nommée normalize\_name. Elle doit accepter une chaîne et retourner un identifiant python valide, c'est-à-dire :\newline
\begin{itemize}
	\item tout ce qui n'est pas un identifiant python valide doit être supprimé
	\item les espaces blancs de début et de fin doivent être supprimés
	\item tout doit être en minuscules
	\item les espaces doivent être remplacés par des traits de soulignement
\end{itemize}

par exemple :\newline
\begin{itemize}
	\item Nom deviendra nom
	\item Prénom deviendra prénom
	\item Completed deviendra completed
\end{itemize}

\renewcommand{\nomfichier}{q209.py}
\begin{solution}
    \pythonfile{\chemincode \nomfichier}[][\nomfichier]
\end{solution}

%Question 11
\question
Écrivez une fonction nommée cumulative\_sum qui accepte une liste de nombres et renvoie une liste qui est la somme cumulative des nombres de la liste.
\begin{itemize}
\item cumulative\_sum([1, 1, 1]) renvoie [1, 2, 3]
\item cumulative\_sum([1, 2, 3, 4]) renvoie [1, 3, 6, 10]
\end{itemize}

\renewcommand{\nomfichier}{q210.py}
\begin{solution}
    \pythonfile{\chemincode \nomfichier}[][\nomfichier]
\end{solution}

%Question 12
\question
Soit les deux listes suivantes :\newline
fruits = ['mango', 'kiwi', 'strawberry', 'guava', 'pineapple', 'mandarin orange']\newline
numbers = [2, 3, 4, 5, 6, 7, 8, 9, 10, 11, 13, 17, 19, 23, 256, -8, -4, -2, 5, -9]\newline
Réécrire l'exemple de code ci-dessus en utilisant la syntaxe de compréhension de liste. Créez une variable nommée uppercased\_fruits pour contenir la sortie de la compréhension de liste. La sortie devrait être ['MANGO', 'KIWI', etc...].
\renewcommand{\nomfichier}{q211.py}
\begin{solution}
    \pythonfile{\chemincode \nomfichier}[][\nomfichier]
\end{solution}

%Question 13
\question
Créer une variable nommée capitalized\_fruits et utiliser la syntaxe de compréhension de liste pour produire des résultats comme ['Mango', 'Kiwi', 'Strawberry', etc...].

\renewcommand{\nomfichier}{q212.py}
\begin{solution}
    \pythonfile{\chemincode \nomfichier}[][\nomfichier]
\end{solution}

%Question 14
\question
Utilisez une compréhension de liste pour créer une variable nommée fruits\_avec\_plus\_de\_deux\_voyelles.

Astuce : Vous aurez besoin d'un moyen de vérifier si quelque chose est une voyelle.


\renewcommand{\nomfichier}{q213.py}
\begin{solution}
    \pythonfile{\chemincode \nomfichier}[][\nomfichier]
\end{solution}

%Question 15
\question
Créer une variable nommée fruits\_avec\_seulement\_deux\_voyelles.

Le résultat devrait être ['mangue', 'kiwi', 'fraise'].

\renewcommand{\nomfichier}{q214.py}
\begin{solution}
    \pythonfile{\chemincode \nomfichier}[][\nomfichier]
\end{solution}

%Question 16
\question
Faire une liste qui contient chaque fruit avec plus de 5 caractères


\renewcommand{\nomfichier}{q215.py}
\begin{solution}
    \pythonfile{\chemincode \nomfichier}[][\nomfichier]
\end{solution}

%Question 17
\question
Faire une liste qui contient chaque fruit avec exactement 5 caractères

\renewcommand{\nomfichier}{q216.py}
\begin{solution}
    \pythonfile{\chemincode \nomfichier}[][\nomfichier]
\end{solution}

%Question 18
\question
Faire une liste qui contient des fruits qui ont moins de 5 caractères

\renewcommand{\nomfichier}{q217.py}
\begin{solution}
    \pythonfile{\chemincode \nomfichier}[][\nomfichier]
\end{solution}

%Question 19
\question
Faites une liste contenant le nombre de caractères de chaque fruit. Les résultats seraient 5, 4, 10, etc...

\renewcommand{\nomfichier}{q218.py}
\begin{solution}
    \pythonfile{\chemincode \nomfichier}[][\nomfichier]
\end{solution}

%Question 20
\question
Créez une variable nommée fruits\_avec\_lettre\_a qui contient une liste des seuls fruits contenant la lettre "a"

\renewcommand{\nomfichier}{q219.py}
\begin{solution}
    \pythonfile{\chemincode \nomfichier}[][\nomfichier]
\end{solution}

%Question 21
\question
Créer une variable nommée even\_numbers qui ne contiendra que les nombres pairs

\renewcommand{\nomfichier}{q220.py}
\begin{solution}
    \pythonfile{\chemincode \nomfichier}[][\nomfichier]
\end{solution}

%Question 22
\question
Créer une variable nommée nombres\_impairs qui ne contient que les nombres impairs


\renewcommand{\nomfichier}{q221.py}
\begin{solution}
    \pythonfile{\chemincode \nomfichier}[][\nomfichier]
\end{solution}

%Question 23
\question
Créer une variable nommée nombres\_positifs qui ne contient que les nombres positifs


\renewcommand{\nomfichier}{q222.py}
\begin{solution}
    \pythonfile{\chemincode \nomfichier}[][\nomfichier]
\end{solution}

%Question 24
\question
Créer une variable nommée nombres\_négatifs qui ne contient que les nombres négatifs

\renewcommand{\nomfichier}{q223.py}
\begin{solution}
    \pythonfile{\chemincode \nomfichier}[][\nomfichier]
\end{solution}

%Question 25
\question
Utiliser une compréhension de liste avec un conditionnel afin de produire une liste de nombres avec 2 chiffres ou plus

\renewcommand{\nomfichier}{q224.py}
\begin{solution}
    \pythonfile{\chemincode \nomfichier}[][\nomfichier]
\end{solution}

%Question 26
\question
Créez une variable nommée numbers\_squared qui contient la liste des nombres avec chaque élément au carré. La sortie est [4, 9, 16, etc...]

\renewcommand{\nomfichier}{q225.py}
\begin{solution}
    \pythonfile{\chemincode \nomfichier}[][\nomfichier]
\end{solution}

%Question 27
\question
Créez une variable nommée nombres\_impairs\_négatifs qui ne contient que les nombres qui sont à la fois impairs et négatifs.

\renewcommand{\nomfichier}{q226.py}
\begin{solution}
    \pythonfile{\chemincode \nomfichier}[][\nomfichier]
\end{solution}

%Question 28
\question
Créez une variable nommée nombres\_plus\_5. Dans cette variable, renvoyez une liste contenant chaque nombre plus cinq.

\renewcommand{\nomfichier}{q227.py}
\begin{solution}
    \pythonfile{\chemincode \nomfichier}[][\nomfichier]
\end{solution}

%Question 29
\question
Créez une variable nommée "primes" qui est une liste contenant les nombres premiers de la liste des nombres. *Astuce : vous pouvez créer ou trouver une fonction d'aide qui détermine si un nombre donné est premier ou non.

\renewcommand{\nomfichier}{q228.py}
\begin{solution}
    \pythonfile{\chemincode \nomfichier}[][\nomfichier]
\end{solution}


\end{questions}
%*********************************************
\section{Site 4}
\href{https://rtavenar.github.io/exos_python/gen/D_1_%20Objet%20_%20creation%20sans%20parametre.html}{Lien vers le site d'origine}\par
Transfert vers cours POO
%\begin{questions}
%	
%	\question
%	%
        \question
        Écrire le code permettant de créer un objet de classe Cercle nommé “monCercle”.
        \par
        \renewcommand{\nomfichier}{q230.py}
        \begin{solution}
            \pythonfile{\chemincode \nomfichier}[][\nomfichier]
        \end{solution}
        

        \question
        Écrire le code permettant de créer un objet de classe Cercle nommé “monCercle”, avec un rayon de 20, une position en X du centre de 5 et une position en Y du centre de 10.
        \par
        \renewcommand{\nomfichier}{q231.py}
        \begin{solution}
            \pythonfile{\chemincode \nomfichier}[][\nomfichier]
        \end{solution}
        

        \question
        Écrire le code permettant de créer un objet de classe Cercle nommé “monCercle”, avec une position en X du centre de 5 et une position en Y du centre de -5. Vous devez utiliser uniquement ces deux valeurs pour la construction de l’objet et ne rien renseigner pour le rayon afin de laisser la valeur par défaut.
        \par
        \renewcommand{\nomfichier}{q232.py}
        \begin{solution}
            \pythonfile{\chemincode \nomfichier}[][\nomfichier]
        \end{solution}
        

        \question
        Écrire le code permettant de créer un objet de classe Cercle nommé “monCercle”, avec un rayon de 10 et les valeurs par défaut pour les positions en X et Y du centre. Créer une variable “perimetre” contenant le résultat du calcul du périmètre de l’objet “monCercle”. Pour ce calcul vous devez uniquement utiliser la constante pi importée au début du programme et l’objet monCercle créé.
        \par
        \renewcommand{\nomfichier}{q233.py}
        \begin{solution}
            \pythonfile{\chemincode \nomfichier}[][\nomfichier]
        \end{solution}
        

        \question
        Écrire le code permettant de créer un objet de classe Cercle nommé “monCercle”, avec un rayon de 20 et les valeurs par défaut pour les positions en X et Y du centre. Créer une variable “surface” contenant le résultat du calcul de la surface de l’objet “monCercle”. Pour obtenir ce résultat vous devez utiliser uniquement l’objet monCercle créé sans écrire vous-même le calcul de la surface.
        \par
        \renewcommand{\nomfichier}{q234.py}
        \begin{solution}
            \pythonfile{\chemincode \nomfichier}[][\nomfichier]
        \end{solution}
        

        \question
        Écrire le code permettant de créer un objet de classe Cercle nommé “monCercle” avec une position du centre en X de 5, une position du centre en Y de 10 et la valeur par défaut pour le rayon.

Écrire ensuite le code permettant de déplacer le centre de l’objet “monCercle” de 5 en X et de -8 en Y. Pour faire se déplacement vous devez uniquement utiliser une méthode de l’objet “monCercle”, avec des paramètres 5 et -8.
        \par
        \renewcommand{\nomfichier}{q235.py}
        \begin{solution}
            \pythonfile{\chemincode \nomfichier}[][\nomfichier]
        \end{solution}
        

        \question
        Écrire le code permettant de définir une classe Personne et son initialisateur. Les objets de classe Personne possèderont un attribut “nom” dont la valeur sera passée en paramètre à la construction de l’objet.
        \par
        \renewcommand{\nomfichier}{q236.py}
        \begin{solution}
            \pythonfile{\chemincode \nomfichier}[][\nomfichier]
        \end{solution}
        

        \question
        Écrire le code permettant de définir une classe Personne et son initialisateur. Les objets de classe Personne possèderont des attributs “nom”, “age”, “poids” et “taille” dont les valeurs seront passées en paramètres à la construction de l’objet (dans cet ordre). Le poids sera donné en kilogrammes et la taille en mètres.

Ajouter à la classe Personne une méthode d’instance nommée “imc” permettant de calculer et de retourner l’IMC (Indice de Masse Corporelle) de la personne. L’IMC d’une personne est égal à son poids (en kilogrammes) divisé par le carré de sa taille (en mètres).
        \par
        \renewcommand{\nomfichier}{q237.py}
        \begin{solution}
            \pythonfile{\chemincode \nomfichier}[][\nomfichier]
        \end{solution}
        

%	
%\end{questions}
%*********************************************
\section{Site 5}
\href{https://pynative.com/python-object-oriented-programming-oop-exercise/#h-oop-exercise-1-create-a-class-with-instance-attributes}{Lien vers le site d'origine}\par
Transfert vers cours POO
%\begin{questions}
%	
%	\question
%	%
        \question
        Créer une classe Véhicule avec les attributs d'instance vitesse\_max et kilométrage.
        \par
        \renewcommand{\nomfichier}{q238.py}
        \begin{solution}
            \pythonfile{\chemincode \nomfichier}[][\nomfichier]
        \end{solution}
        

        \question
        Créer une classe Véhicule sans variables ni méthodes
        \par
        \renewcommand{\nomfichier}{q239.py}
        \begin{solution}
            \pythonfile{\chemincode \nomfichier}[][\nomfichier]
        \end{solution}
        

        \question
        Créer une classe enfant Bus qui héritera de toutes les variables et méthodes de la classe Véhicule.\newline
        Vous partirez du code suivant :
        \renewcommand{\nomfichier}{q240depart.py}
        \pythonfile{\chemincode \nomfichier}[][\nomfichier]
				Créer un objet Bus qui héritera de toutes les variables et méthodes de la classe parente "Véhicule" et l'afficher.\newline
				Sortie attendue :\newline
				Nom du véhicule : School Volvo Vitesse : 180 Kilométrage : 12
        \par
        \renewcommand{\nomfichier}{q240.py}
        \begin{solution}
            \pythonfile{\chemincode \nomfichier}[][\nomfichier]
        \end{solution}
        

        \question
        Héritage des classes

				Créez une classe Bus qui hérite de la classe Véhicule. Donnez à l'argument capacité de \textbf{Bus.seating\_capacity()} une valeur par défaut de 50.\newline
				
				Utilisez le code suivant pour votre classe mère Vehicle.
        \renewcommand{\nomfichier}{q241depart.py}
        \pythonfile{\chemincode \nomfichier}[][\nomfichier]				
				Sortie attendue :\newline
				La capacité d'accueil d'un bus est de 50 passagers.
        \par
        \textbf{Indices : }Tout d'abord, utilisez la surcharge de méthode.\newline
   Ensuite, utilisez l'argument de méthode par défaut dans la définition de la méthode seating\_capacity() d'une classe de bus.
        \renewcommand{\nomfichier}{q241.py}
        \begin{solution}
            \pythonfile{\chemincode \nomfichier}[][\nomfichier]
        \end{solution}
        

        \question
        Définir une propriété qui doit avoir la même valeur pour chaque instance de classe (objet)\newline

Définir un attribut de classe "color" dont la valeur par défaut est white.\newline

Utilisez le code suivant pour cet exercice.\newline
        \renewcommand{\nomfichier}{q242depart.py}
        \pythonfile{\chemincode \nomfichier}[][\nomfichier]		

Résultat attendu :\newline

Couleur : Blanc, Nom du véhicule : School Volvo, Vitesse : 180, Kilométrage : 12\newline
Couleur : Blanc, Nom du véhicule : Audi Q5, Vitesse : 240, Kilométrage : 18


        \par
        \textbf{Indices : }Définir une couleur comme variable de classe dans une classe de véhicule
        \renewcommand{\nomfichier}{q242.py}
        \begin{solution}
            \pythonfile{\chemincode \nomfichier}[][\nomfichier]
            Les variables créées dans .\_\_init\_\_() sont appelées variables d'instance. La valeur d'une variable d'instance est spécifique à une instance particulière de la classe. Par exemple, dans la solution, tous les objets Véhicule ont un nom et une vitesse maximale, mais les valeurs des variables nom et vitesse maximale varient en fonction de l'instance de Véhicule.
            
            En revanche, la variable de classe est partagée par toutes les instances de la classe. Vous pouvez définir un attribut de classe en attribuant une valeur à un nom de variable en dehors de .\_\_init\_\_().
        \end{solution}
        

        \question
        Héritage des classes

Créez une classe enfant Bus qui hérite de la classe Véhicule. Le tarif par défaut de tout véhicule est égal au \textbf{nombre de places * 100}. Si le véhicule est une instance de bus, nous devons ajouter 10 \% au tarif total à titre de frais de maintenance. Ainsi, le tarif total pour l'instance de bus deviendra le \textbf{montant final = tarif total + 10 \% du tarif total}.

Remarque : le nombre de places assises dans le bus est de 50, le montant final du tarif devrait donc être de 5500. Vous devez surcharger la méthode fare() de la classe Vehicle dans la classe Bus.

Utilisez le code suivant pour votre classe de véhicule parent. Nous devons accéder à la classe mère à partir d'une méthode d'une classe enfant.

        \renewcommand{\nomfichier}{q243depart.py}
        \pythonfile{\chemincode \nomfichier}[][\nomfichier]	

Résultat attendu :\newline

Le prix total du billet d'autobus est de 5500.0
        \par
        \renewcommand{\nomfichier}{q243.py}
        \begin{solution}
            \pythonfile{\chemincode \nomfichier}[][\nomfichier]
        \end{solution}
        

        \question
        Vérifier le type d'un objet

Écrire un programme permettant de déterminer à quelle classe appartient un objet Bus donné.
Vous partirez du code suivant :
        \renewcommand{\nomfichier}{q244depart.py}
        \pythonfile{\chemincode \nomfichier}[][\nomfichier]	
        \par
        \textbf{Indices : }Utilisez la fonction intégrée type() de Python.
        \renewcommand{\nomfichier}{q244.py}
        \begin{solution}
            \pythonfile{\chemincode \nomfichier}[][\nomfichier]
        \end{solution}
        

        \question
        Déterminer si School\_bus est également une instance de la classe Vehicle
        Vous partirez du code suivant :
        \renewcommand{\nomfichier}{q245depart.py}
        \pythonfile{\chemincode \nomfichier}[][\nomfichier]	        
        \par
        \textbf{Indices : }Utiliser la fonction isinstance()
        \renewcommand{\nomfichier}{q245.py}
        \begin{solution}
            \pythonfile{\chemincode \nomfichier}[][\nomfichier]
        \end{solution}
        

        

%	
%\end{questions}
%*********************************************
\section{Site 6}
\href{https://github.com/karan/Projects-Solutions}{Lien vers le site d'origine}\par
\begin{questions}
	
	\renewcommand{\chemincode}{../../code/}
        \question
Projet d'inventaire de produits - Créer une application qui gère un inventaire de produits. Créez une classe de produits avec un prix, un identifiant et une quantité disponible. Créez ensuite une classe d'inventaire qui garde la trace des différents produits et peut résumer la valeur de l'inventaire.
        \par
        
        \begin{solution}
        voir les fichiers q300 et q300-01
%        		\renewcommand{\nomfichier}{q300.py}
%            \pythonfile{\chemincode \nomfichier}[][\nomfichier]
%            \renewcommand{\nomfichier}{q300-01.py}
%            \pythonfile{\chemincode \nomfichier}[][\nomfichier]
            
        \end{solution}

\question
Système de réservation de billets d'avion ou de chambres d'hôtel - Créer un système de réservation de billets d'avion ou de chambres d'hôtel. Il applique différents tarifs pour des sections particulières de l'avion ou de l'hôtel. Par exemple, la première classe coûtera plus cher que la première classe. Les chambres d'hôtel ont des suites penthouse qui coûtent plus cher. Gardez une trace de la disponibilité des chambres et de leur programmation.

\question
Gestionnaire de compte bancaire - Créez une classe appelée Compte qui sera une classe abstraite pour trois autres classes appelées CompteChèque, CompteÉpargne et CompteAffaires. Gérez les crédits et les débits de ces comptes à l'aide d'un programme de type distributeur automatique de billets.

       \begin{solution}
       Voir le fichier q301
%        		\renewcommand{\nomfichier}{q301.py}
%            \pythonfile{\chemincode \nomfichier}[][\nomfichier]
        \end{solution}
        
\question
Classes de surface et de périmètre des formes - Créez une classe abstraite appelée Forme et héritez-en d'autres formes comme le diamant, le rectangle, le cercle, le triangle, etc. Ensuite, chaque classe doit surcharger les fonctionnalités de surface et de périmètre pour gérer chaque type de forme.

       \begin{solution}
       		Voir le fichier q302
%        		\renewcommand{\nomfichier}{q302.py}
%            \pythonfile{\chemincode \nomfichier}[][\nomfichier]
        \end{solution}
       
	
\end{questions}
%*********************************************
\section{Divers}
%\href{https://github.com/karan/Projects-Solutions}{Lien vers le site d'origine}\par
\begin{questions}
	
	
%--------------------%
        \question
Nous avons mis en place un système d'amende pour les chasseurs de notre commune.
Chaque chasseur se voit pénaliser d'un certain nombre de point par faute.
Le barème des pénalités est le suivant :
\begin{itemize}
\item s'il tue une poule : 1 point.
\item s'il tue un chien : 3 points.
\item s'il tue une vache : 5 points.
\item s'il tue un ami : 10 points.
\end{itemize}
Un point à une valeur de 2€.

Écrire une fonction amende qui reçoit le nombre de victimes du chasseur et qui renvoie la somme due.

Utilisez cette fonction dans un programme principal qui demande le nombre de victimes et qui affiche la somme que le chasseur doit débourser.

        \begin{solution}
                \renewcommand{\nomfichier}{q133.py}
                \pythonfile{\chemincode \nomfichier}[][\nomfichier]
        \end{solution}
%--------------------%
\question
Soit des comptes bancaires d'individus définis par la liste :\par
\renewcommand{\nomfichier}{q134-comptes.py}
\pythonfile{\chemincode \nomfichier}[][\nomfichier]
On considère que les individus qui portent le même 'nom' sont de la même famille.
En cas d'absence de revenu attribué à un individu, nous considérerons que son épargne est nulle (cas de 'Bernard Gueux').

Écrire une fonction qui retourne le nom de la famille la plus pauvre et de la plus riche avec le montant de leur épargne respective.
Ici, ('Gueux', 1253) et ('Durois', 310000).

\begin{solution}
        \renewcommand{\nomfichier}{q134.py}
        \pythonfile{\chemincode \nomfichier}[][\nomfichier][breakable]
\end{solution}
%%--------------------%
%\question
%
%
%\begin{solution}
%        \renewcommand{\nomfichier}{pascorrige.py}
%        \pythonfile{\chemincode \nomfichier}[][\nomfichier]
%\end{solution}
%%--------------------%
%\question
%
%
%\begin{solution}
%\renewcommand{\nomfichier}{pascorrige.py}
%\pythonfile{\chemincode \nomfichier}[][\nomfichier]
%\end{solution}
%%--------------------%
%\question
%
%
%\begin{solution}
%        \renewcommand{\nomfichier}{pascorrige.py}
%        \pythonfile{\chemincode \nomfichier}[][\nomfichier]
%\end{solution}
%%--------------------%
%\question
%
%
%\begin{solution}
%\renewcommand{\nomfichier}{pascorrige.py}
%\pythonfile{\chemincode \nomfichier}[][\nomfichier]
%\end{solution}
%%--------------------%
%\question
%
%
%\begin{solution}
%        \renewcommand{\nomfichier}{pascorrige.py}
%        \pythonfile{\chemincode \nomfichier}[][\nomfichier]
%\end{solution}
%%--------------------%
%\question
%
%
%\begin{solution}
%\renewcommand{\nomfichier}{pascorrige.py}
%\pythonfile{\chemincode \nomfichier}[][\nomfichier]
%\end{solution}
%%--------------------%
%\question
%
%
%\begin{solution}
%        \renewcommand{\nomfichier}{pascorrige.py}
%        \pythonfile{\chemincode \nomfichier}[][\nomfichier]
%\end{solution}
%%--------------------%
%\question
%
%
%\begin{solution}
%\renewcommand{\nomfichier}{pascorrige.py}
%\pythonfile{\chemincode \nomfichier}[][\nomfichier]
%\end{solution}
%%--------------------%
%\question
%
%
%\begin{solution}
%        \renewcommand{\nomfichier}{pascorrige.py}
%        \pythonfile{\chemincode \nomfichier}[][\nomfichier]
%\end{solution}
%%--------------------%
%\question
%
%
%\begin{solution}
%\renewcommand{\nomfichier}{pascorrige.py}
%\pythonfile{\chemincode \nomfichier}[][\nomfichier]
%\end{solution}
%%--------------------%
%\question
%
%
%\begin{solution}
%        \renewcommand{\nomfichier}{pascorrige.py}
%        \pythonfile{\chemincode \nomfichier}[][\nomfichier]
%\end{solution}
%%--------------------%
%\question
%
%
%\begin{solution}
%\renewcommand{\nomfichier}{pascorrige.py}
%\pythonfile{\chemincode \nomfichier}[][\nomfichier]
%\end{solution}
%%--------------------%
%\question
%
%
%\begin{solution}
%        \renewcommand{\nomfichier}{pascorrige.py}
%        \pythonfile{\chemincode \nomfichier}[][\nomfichier]
%\end{solution}
%%--------------------%
%\question
%
%
%\begin{solution}
%\renewcommand{\nomfichier}{pascorrige.py}
%\pythonfile{\chemincode \nomfichier}[][\nomfichier]
%\end{solution}
%%--------------------%
%\question
%
%
%\begin{solution}
%        \renewcommand{\nomfichier}{pascorrige.py}
%        \pythonfile{\chemincode \nomfichier}[][\nomfichier]
%\end{solution}
%%--------------------%
%\question
%
%
%\begin{solution}
%\renewcommand{\nomfichier}{pascorrige.py}
%\pythonfile{\chemincode \nomfichier}[][\nomfichier]
%\end{solution}
%%--------------------%
%\question
%
%
%\begin{solution}
%        \renewcommand{\nomfichier}{pascorrige.py}
%        \pythonfile{\chemincode \nomfichier}[][\nomfichier]
%\end{solution}
%%--------------------%
%\question
%
%
%\begin{solution}
%\renewcommand{\nomfichier}{pascorrige.py}
%\pythonfile{\chemincode \nomfichier}[][\nomfichier]
%\end{solution}
%%--------------------%
%\question
%
%
%\begin{solution}
%        \renewcommand{\nomfichier}{pascorrige.py}
%        \pythonfile{\chemincode \nomfichier}[][\nomfichier]
%\end{solution}
%%--------------------%
%\question
%
%
%\begin{solution}
%\renewcommand{\nomfichier}{pascorrige.py}
%\pythonfile{\chemincode \nomfichier}[][\nomfichier]
%\end{solution}
	
\end{questions}
%*********************************************
\section{Site 7}
\href{https://bbookman.github.io/Python-list-comprehension1/}{Lien vers le site d'origine}\par
\begin{questions}
	
	
%--------------------%
        \question
        Trouvez tous les nombres compris entre 1 et 1000 qui sont divisibles par 7.

        \begin{solution}
                \renewcommand{\nomfichier}{q135.py}
                \pythonfile{\chemincode \nomfichier}[][\nomfichier]
        \end{solution}
%--------------------%
\question
Trouvez tous les nombres de 1-1000 qui contiennent un 3.

\begin{solution}
        \renewcommand{\nomfichier}{q136.py}
        \pythonfile{\chemincode \nomfichier}[][\nomfichier]
\end{solution}
%--------------------%
\question
Compter le nombre d'espaces dans une chaine.

\begin{solution}
        \renewcommand{\nomfichier}{q137.py}
        \pythonfile{\chemincode \nomfichier}[][\nomfichier]
\end{solution}
%--------------------%
\question
Créer une liste de toutes les consonnes de la chaîne "Les Yaks jaunes aiment crier et bailler et hier ils ont jodlé en mangeant des ignames yuky".

\begin{solution}
\renewcommand{\nomfichier}{q138.py}
\pythonfile{\chemincode \nomfichier}[][\nomfichier][breakable]
\end{solution}
%--------------------%
\question
Obtenir l'indice et la valeur sous forme de tuple pour les éléments de la liste ["hi", 4, 8.99, 'apple', ('t,b','n')].  Le résultat ressemblerait à [(index, valeur), (index, valeur)].

\begin{solution}
        \renewcommand{\nomfichier}{q139.py}
        \pythonfile{\chemincode \nomfichier}[][\nomfichier]
\end{solution}
%--------------------%
\question
Trouver les nombres communs à deux listes (sans utiliser de tuple ou d'ensemble) list\_a = [1, 2, 3, 4], list\_b = [2, 3, 4, 5]

\begin{solution}
\renewcommand{\nomfichier}{q140.py}
\pythonfile{\chemincode \nomfichier}[][\nomfichier]
\end{solution}
%--------------------%
\question
Dans une phrase comme "En 1984, il y a eu 13 cas de manifestations ayant rassemblé plus de 1 000 personnes", il ne faut retenir que les chiffres.  Le résultat est une liste de nombres comme [3,4,5].

\begin{solution}
        \renewcommand{\nomfichier}{q141.py}
        \pythonfile{\chemincode \nomfichier}[][\nomfichier]
\end{solution}
%--------------------%
\question
Étant donné numbers = range(20), produisez une liste contenant le mot "even" si un des nombres est pair, et le mot "odd" si le nombre est impair.  Le résultat ressemblerait à ['odd', 'odd', 'even'].

\begin{solution}
\renewcommand{\nomfichier}{q142.py}
\pythonfile{\chemincode \nomfichier}[][\nomfichier]
\end{solution}
%--------------------%
\question
Produisez une liste de tuples composée uniquement des nombres correspondants dans ces listes list\_a = [1, 2, 3, 4, 5, 6, 7, 8, 9], list\_b = [2, 7, 1, 12].  Le résultat ressemblerait à (4,4), (12,12)

\begin{solution}
        \renewcommand{\nomfichier}{q143.py}
        \pythonfile{\chemincode \nomfichier}[][\nomfichier]
\end{solution}
%--------------------%
\question
Trouver tous les mots d'une chaîne de moins de 4 lettres

\begin{solution}
\renewcommand{\nomfichier}{q144.py}
\pythonfile{\chemincode \nomfichier}[][\nomfichier]
\end{solution}
%--------------------%
\question
Utilisez la compréhension d'une liste imbriquée pour trouver tous les nombres de 1 à 100 qui sont divisibles par n'importe quel chiffre à part 1 (2-9).

\begin{solution}
        \renewcommand{\nomfichier}{q145.py}
        \pythonfile{\chemincode \nomfichier}[][\nomfichier]
\end{solution}

	
\end{questions}
%*********************************************
\section{Site 8}
\href{https://www.tutorjoes.in/python_programming_tutorial/list_comprehensions_exercises_in_python}{List de comprehension}\par
\href{https://www.tutorjoes.in/python_programming_tutorial/tuple_comprehensions_exercises_in_python}{Tuple de comprehension}\par
\href{https://www.tutorjoes.in/python_programming_tutorial/set_comprehensions_exercises_in_python}{Set de comprehension}\par
\href{https://www.tutorjoes.in/python_programming_tutorial/dictionary_comprehensions_exercises_in_python}{Dictionnary de comprehension}\par

\begin{questions}
	
	\renewcommand{\chemincode}{../../code/}
        \question
        Créer une liste de carrés de nombres de 1 à 10

Exemple de sortie

[1, 4, 9, 16, 25, 36, 49, 64, 81, 100]
        \par
        \begin{solution}
            \renewcommand{\nomfichier}{q500.py}
            \pythonfile{\chemincode \nomfichier}[][\nomfichier]
            Le code que vous avez fourni est écrit en Python et utilise une compréhension de liste pour créer une liste appelée squares qui contient les carrés des nombres de 1 à 10. Voici une explication pas à pas du code :\par

\begin{itemize}
\item     squares = [x**2 for x in range(1, 11)] : Cette ligne de code initialise une variable nommée carrés et lui affecte le résultat d'une compréhension de liste.

\begin{itemize}
\item         for x in range(1, 11) : Cette partie met en place une boucle qui parcourt les nombres de 1 à 10 (inclus). La fonction range(1, 11) génère une séquence de nombres commençant par 1 et se terminant par 10.
\item         x**2 : pour chaque valeur de x dans la plage, cette expression calcule le carré de x.
\item{}    [x**2 for x in range(1, 11)]  : Il s'agit de la compréhension de la liste elle-même. Elle parcourt les nombres de l'intervalle spécifié (1 à 10) et, pour chaque nombre, calcule son carré. Les carrés obtenus sont rassemblés dans une nouvelle liste.
\end{itemize}
    \item print(squares) : Cette ligne de code imprime simplement la liste des carrés sur la console.
\end{itemize}
        \end{solution}
        

        \question
        Créer une liste de nombres pairs de 1 à 20

Exemple de résultat

[2, 4, 6, 8, 10, 12, 14, 16, 18, 20]
        \par
        \begin{solution}
            \renewcommand{\nomfichier}{q501.py}
            \pythonfile{\chemincode \nomfichier}[][\nomfichier]
            Ce code Python crée une liste appelée evens à l'aide d'une compréhension de liste, qui contient les nombres pairs de 1 à 20. Voici un aperçu du fonctionnement du code :\par

\begin{itemize}
\item     evens = [x for x in range(1, 21) if x \% 2 == 0] : Cette ligne de code initialise une variable nommée evens et lui affecte le résultat d'une compréhension de liste.

 \begin{itemize}
 \item        for x in range(1, 21) : Cette partie met en place une boucle qui parcourt les nombres de 1 à 20 (inclus). La fonction range(1, 21) génère une séquence de nombres commençant par 1 et se terminant par 20.
 \item         if x \% 2 == 0 : il s'agit d'une condition qui filtre les nombres. Elle vérifie si la valeur actuelle de x est paire. L'opérateur \% calcule le reste lorsque x est divisé par 2. Si le reste est égal à 0, cela signifie que x est pair.
 \item{}         [x for x in range(1, 21) if x \% 2 == 0] : Il s'agit de la compréhension de la liste elle-même. Elle parcourt les nombres de l'intervalle spécifié (1 à 20) et, pour chaque nombre, vérifie s'il est pair. Si le nombre est pair, il est inclus dans la nouvelle liste.
 \end{itemize}
   \item print(evens) : Cette ligne de code imprime la liste evens sur la console.
\end{itemize}
        \end{solution}
        

        \question
        Générer une liste de caractères à partir d'une chaîne de caractères

Exemple de sortie

['H', 'e', 'l', 'l', 'o', 'w', 'o', 'r', 'l', 'd']
        \par
        \begin{solution}
            \renewcommand{\nomfichier}{q502.py}
            \pythonfile{\chemincode \nomfichier}[][\nomfichier]
            Ce code Python crée une liste appelée chars en utilisant une compréhension de liste pour extraire les caractères alphabétiques de la chaîne donnée "Hello, world !". Voici comment fonctionne ce code :\par

\begin{itemize}
\item     string = "Hello, world !": Cette ligne initialise une variable nommée string et lui affecte la valeur "Hello, world !", qui est une chaîne contenant des lettres, des espaces et de la ponctuation.
\item     chars = [char for char in string if char.isalpha()] : Cette ligne de code initialise une variable nommée chars et lui affecte le résultat d'une compréhension de liste.

\begin{itemize}
\item         for char in string : Cette partie met en place une boucle qui parcourt chaque caractère (char) de la chaîne.
\item         if char.isalpha() : Il s'agit d'une condition qui vérifie si le caractère courant char est alphabétique. La méthode .isalpha() est une méthode de chaîne qui renvoie True si le caractère est une lettre de l'alphabet et False s'il ne l'est pas.
\item{}         [char for char in string if char.isalpha()] : Il s'agit de la compréhension de la liste elle-même. Elle parcourt chaque caractère de la chaîne et, pour chaque caractère alphabétique, l'inclut dans la nouvelle liste.
\end{itemize}
  \item  print(chars) : Cette ligne de code imprime la liste des caractères sur la console.
  \end{itemize}
        \end{solution}
        

        \question
        
        Créer une liste de longueurs de mots dans une phrase

Exemple de sortie

Voici un exemple de phrase.

[4, 2, 1, 6, 9]
        \par
        \begin{solution}
            \renewcommand{\nomfichier}{q503.py}
            \pythonfile{\chemincode \nomfichier}[][\nomfichier]
            Ce code Python analyse une phrase et crée une liste appelée word\_lengths en utilisant une compréhension de liste pour stocker les longueurs de chaque mot dans la phrase. Voici comment fonctionne le code :\par

\begin{itemize}
\item     sentence = "Ceci est un exemple de phrase" : Cette ligne initialise une variable nommée sentence et lui attribue la valeur "This is a sample sentence".
\item     word\_lengths = [len(word) for word in sentence.split()] : Cette ligne de code initialise une variable nommée word\_lengths et lui affecte le résultat d'une compréhension de liste.

\begin{itemize}
\item         sentence.split() : Cette partie du code divise la phrase en une liste de mots. Par défaut, la phrase est divisée sur les espaces blancs, ce qui permet de séparer les mots.
\item         for word in sentence.split() : Cette partie met en place une boucle qui parcourt chaque mot de la liste de mots.
\item         len(word) : Pour chaque mot de la liste, cette expression calcule la longueur du mot à l'aide de la fonction len().
\item{}         [len(word) for word in sentence.split()] : Il s'agit de la compréhension de la liste elle-même. Elle parcourt la liste de mots et, pour chaque mot, calcule sa longueur et l'inclut dans la nouvelle liste.
\end{itemize}
\item    print(sentence) : Cette ligne de code imprime la phrase originale sur la console.
\item    print(word\_lengths) : Cette ligne de code imprime la liste des longueurs de mots sur la console.
 \end{itemize}
        \end{solution}
        

        \question
        
        Générer une liste de tuples contenant un nombre et son carré

Exemple de sortie

[(1, 1), (2, 4), (3, 9), (4, 16), (5, 25)]
        \par
        \begin{solution}
            \renewcommand{\nomfichier}{q504.py}
            \pythonfile{\chemincode \nomfichier}[][\nomfichier]
            Ce code Python crée une liste appelée num\_squares en utilisant une compréhension de liste pour générer des paires de nombres et leurs carrés pour des valeurs de x allant de 1 à 5. Voici comment fonctionne le code :\par

\begin{itemize}
\item     num\_squares = [(x, x**2) for x in range(1, 6)] : Cette ligne de code initialise une variable nommée num\_squares et lui affecte le résultat d'une compréhension de liste.

\begin{itemize}
\item         for x in range(1, 6) : Cette partie met en place une boucle qui parcourt les valeurs de x de 1 à 5 (inclus). La fonction range(1, 6) génère une séquence de nombres commençant par 1 et se terminant par 5.
\item         (x, x**2) : Pour chaque valeur de x dans l'intervalle, cette expression crée un tuple contenant deux éléments : la valeur originale x et son carré x**2.
\item{}         [(x, x**2) for x in range(1, 6)] : Il s'agit de la compréhension de la liste elle-même. Elle parcourt les valeurs de x dans l'intervalle spécifié (1 à 5) et, pour chaque valeur, génère un tuple contenant x et x**2. Ces n-uplets sont rassemblés dans une nouvelle liste.
\end{itemize}
\item    print(num\_squares) : Cette ligne de code affiche la liste num\_squares sur la console.
\end{itemize}
        \end{solution}
        

        \question
        Créer une liste de lettres minuscules

Exemple de résultat

['a', 'b', 'c', 'd', 'e', 'f', 'g', 'h', 'i', 'j', 'k', 'l', 'm', 'n', 'o', 'p', 'q', 'r', 's', 't', 'u', 'v', 'w', 'x', 'y', 'z']
        \par
        \begin{solution}
            \renewcommand{\nomfichier}{q505.py}
            \pythonfile{\chemincode \nomfichier}[][\nomfichier]
            Ce code Python crée une liste appelée lowercase\_letters en utilisant une compréhension de liste pour générer des lettres minuscules de l'alphabet anglais. Voici comment fonctionne ce code :\par

 \begin{itemize}
 \item    lowercase\_letters = [chr(x) for x in range(ord('a'), ord('z')+1)] : Cette ligne de code initialise une variable nommée lowercase\_letters et lui affecte le résultat d'une compréhension de liste.

 \begin{itemize}
 \item        range(ord('a'), ord('z')+1) : Cette partie du code génère une plage de valeurs entières correspondant aux valeurs Unicode des lettres minuscules de l'alphabet anglais. ord('a') renvoie la valeur Unicode de la lettre 'a', et ord('z') renvoie la valeur Unicode de la lettre 'z'. L'ajout de 1 permet de s'assurer que la lettre 'z' est incluse dans la plage.
 \item         chr(x) : Pour chaque entier x de la plage, cette expression le convertit en caractère à l'aide de la fonction chr(). chr(x) renvoie le caractère correspondant à la valeur Unicode de x.
 \item{}         [chr(x) for x in range(ord('a'), ord('z')+1)] : Il s'agit de la compréhension de la liste elle-même. Elle parcourt la gamme des valeurs Unicode pour les lettres minuscules et convertit chaque valeur en son caractère correspondant. Ces caractères (lettres minuscules) sont rassemblés dans une nouvelle liste.
 \end{itemize}
 \item   print(lowercase\_letters) : Cette ligne de code imprime la liste des lettres minuscules sur la console. 
 \end{itemize}
        \end{solution}
        

        \question
        Générer une liste de lettres majuscules

Exemple de sortie

['A', 'B', 'C', 'D', 'E', 'F', 'G', 'H', 'I', 'J', 'K', 'L', 'M', 'N', 'O', 'P', 'Q', 'R', 'S', 'T', 'U', 'V', 'W', 'X', 'Y', 'Z']
        \par
        \begin{solution}
            \renewcommand{\nomfichier}{q506.py}
            \pythonfile{\chemincode \nomfichier}[][\nomfichier]
            Ce code Python crée une liste appelée uppercase\_letters en utilisant une compréhension de liste pour générer des lettres majuscules de l'alphabet anglais. Voici comment fonctionne ce code :\par

\begin{itemize}
\item     uppercase\_letters = [chr(x) for x in range(ord('A'), ord('Z')+1)] : Cette ligne de code initialise une variable nommée uppercase\_letters et lui affecte le résultat d'une compréhension de liste.

 \begin{itemize}
 \item        range(ord('A'), ord('Z')+1) : Cette partie du code génère une plage de valeurs entières correspondant aux valeurs Unicode des lettres majuscules de l'alphabet anglais. ord('A') renvoie la valeur Unicode de la lettre 'A', et ord('Z') renvoie la valeur Unicode de la lettre 'Z'. En ajoutant 1, on s'assure que la lettre 'Z' est incluse dans la plage.
 \item         chr(x) : Pour chaque entier x de la plage, cette expression le convertit en caractère à l'aide de la fonction chr(). chr(x) renvoie le caractère correspondant à la valeur Unicode de x.
 \item{}         [chr(x) for x in range(ord('A'), ord('Z')+1)] : Il s'agit de la compréhension de la liste elle-même. Elle parcourt la gamme des valeurs Unicode pour les lettres majuscules et convertit chacune d'elle en son caractère correspondant. Ces caractères (lettres majuscules) sont rassemblés dans une nouvelle liste.
 \end{itemize}
\item    print(uppercase\_letters) : Cette ligne de code imprime la liste des lettres majuscules sur la console.
 \end{itemize}
        \end{solution}
        

        \question
        Créer une liste de nombres pairs au carré et de nombres impairs au cube de 1 à 10

Exemple de résultat

[1, 4, 27, 16, 125, 36, 343, 64, 729, 100]
        \par
        \begin{solution}
            \renewcommand{\nomfichier}{q507.py}
            \pythonfile{\chemincode \nomfichier}[][\nomfichier]
            Ce code Python crée une liste appelée result en utilisant une compréhension de liste pour calculer le carré ou le cube des nombres de 1 à 10 selon qu'ils sont pairs ou impairs. Voici comment fonctionne ce code :\par

\begin{itemize}
\item     result = [x**2 if x \% 2 == 0 else x**3 for x in range(1, 11)] : Cette ligne de code initialise une variable nommée result et lui affecte le résultat d'une compréhension de liste.

\begin{itemize}
\item         for x in range(1, 11) : Cette partie met en place une boucle qui parcourt les nombres de 1 à 10 (inclus). La fonction range(1, 11) génère une séquence de nombres commençant par 1 et se terminant par 10.
\item         x**2 if x \% 2 == 0 else x**3 : Pour chaque valeur de x dans la plage, cette expression calcule le carré (x**2) ou le cube (x**3) du nombre selon que x est pair (x \% 2 == 0) ou non. Si x est pair, elle calcule le carré ; sinon, elle calcule le cube.
\item{}         [x**2 if x \% 2 == 0 else x**3 for x in range(1, 11)] : Il s'agit de la compréhension de la liste elle-même. Elle parcourt les nombres de l'intervalle spécifié (1 à 10) et, pour chaque nombre, calcule son carré ou son cube selon qu'il est pair ou impair. Les résultats sont rassemblés dans une nouvelle liste.
\end{itemize}
\item    print(result) : Cette ligne de code imprime la liste des résultats sur la console.
\end{itemize}
        \end{solution}
        

        \question
        Générer une liste de multiples communs de 3 et 5 jusqu'à 100

Exemple de résultat

[15, 30, 45, 60, 75, 90]
        \par
        \begin{solution}
            \renewcommand{\nomfichier}{q508.py}
            \pythonfile{\chemincode \nomfichier}[][\nomfichier]
            Ce code Python crée une liste appelée common\_multiples en utilisant une compréhension de liste pour trouver et stocker les nombres de 1 à 100 qui sont des multiples de 3 et de 5. Voici comment fonctionne le code :

\begin{itemize}
	\item     common\_multiples = [x for x in range(1, 101) if x \% 3 == 0 and x \% 5 == 0] : Cette ligne de code initialise une variable nommée common\_multiples et lui affecte le résultat d'une compréhension de liste.

       \begin{itemize}
       	\item  for x in range(1, 101) : Cette partie met en place une boucle qui parcourt les nombres de 1 à 100 (inclus). La fonction range(1, 101) génère une séquence de nombres commençant par 1 et se terminant par 100.
       	\item         if x \% 3 == 0 and x \% 5 == 0 : cette condition vérifie si la valeur actuelle de x est un multiple de 3 et de 5. L'opérateur \% calcule le reste lorsque x est divisé par 3 et 5. Si le reste est égal à 0 pour les deux divisions, cela signifie que x est un multiple de 3 et de 5.
       	\item{}         [x for x in range(1, 101) if x \% 3 == 0 and x \% 5 == 0] : Il s'agit de la compréhension de la liste elle-même. Elle parcourt les nombres de l'intervalle spécifié (1 à 100) et, pour chaque nombre qui est un multiple de 3 et de 5, l'inclut dans la nouvelle liste.
       \end{itemize}
    \item print(common\_multiples) : Cette ligne de code imprime la liste common\_multiples sur la console.
   \end{itemize}
        \end{solution}
        

        \question
        Créer une liste de chaînes inversées à partir d'une autre liste

Exemple de résultat

['pomme', 'banane', 'cerise']

['elppa', 'ananab', 'yrrehc']
        \par
        \begin{solution}
            \renewcommand{\nomfichier}{q509.py}
            \pythonfile{\chemincode \nomfichier}[][\nomfichier]
            Ce code Python prend une liste de mots, inverse chaque mot de la liste et stocke les mots inversés dans une nouvelle liste appelée mots\_inversés. Voici comment fonctionne ce code :
\begin{itemize}
	\item     words = ["apple", "banana", "cherry"] : Cette ligne initialise une variable nommée words et lui affecte une liste contenant trois mots : "apple", "banana" et "cherry".
	\item     reversed\_words = [mot[::-1] pour word dans words] : Cette ligne de code initialise une variable nommée reversed\_words et lui affecte le résultat de la compréhension d'une liste.

      \begin{itemize}
      	\item   for word in words : Cette partie met en place une boucle qui parcourt chaque mot de la liste des mots.
      	\item         word[::-1] : Pour chaque mot de la liste, cette expression utilise un slice ([::-1]) pour inverser les caractères du mot. La notation de tranche [::-1] inverse l'ordre des caractères dans une chaîne.
      	\item{}         [word[::-1] for word in words] : Il s'agit de la compréhension de  liste elle-même. Elle parcourt les mots de la liste words et, pour chaque mot, l'inverse et inclut le mot inversé dans la nouvelle liste.
      \end{itemize}
    \item print(words) : Cette ligne de code imprime la liste de mots originale sur la console.
    \item print(reversed\_words) : Cette ligne de code affiche la liste des mots inversés sur la console.
    \end{itemize}
        \end{solution}
        

        \question
        Générer une liste de nombres premiers de 1 à 50

Exemple de sortie

[2, 3, 5, 7, 11, 13, 17, 19, 23, 29, 31, 37, 41, 43, 47]
        \par
        \begin{solution}
            \renewcommand{\nomfichier}{q510.py}
            \pythonfile{\chemincode \nomfichier}[][\nomfichier]
            Le code Python fourni définit une fonction is\_prime(n) qui vérifie si un entier donné n est un nombre premier ou non. Elle crée ensuite une liste appelée prime\_numbers en utilisant une compréhension de liste pour trouver et stocker les nombres premiers entre 1 et 50. Voici comment fonctionne le code :
\begin{itemize}
    \item def is\_prime(n) : Cette ligne définit une fonction nommée is\_prime qui prend un entier n comme argument.
\begin{itemize}
	\item         if n <= 1 : Cette ligne vérifie si n est inférieur ou égal à 1. Si n est inférieur ou égal à 1, la fonction renvoie False, car 1 et tout nombre négatif ne sont pas premiers.
	\item         for i in range(2, int(n**0.5) + 1): : Cette ligne met en place une boucle qui itère de 2 à la racine carrée de n (incluse) en utilisant la fonction range(). La vérification jusqu'à la racine carrée est une optimisation visant à réduire le nombre de divisions nécessaires pour déterminer la primalité.
	\item         if n \% i == 0: : À l'intérieur de la boucle, cette ligne vérifie si n est divisible par la valeur actuelle de i. Si c'est le cas, cela signifie que n n'est pas premier, et la fonction renvoie donc False.
	\item         Si aucune des conditions ci-dessus n'est remplie, cela signifie que n n'est divisible par aucun nombre de la plage, et la fonction renvoie donc True, indiquant que n est un nombre premier.
\end{itemize}
    \item prime\_numbers = [x for x in range(1, 51) if is\_prime(x)] : Cette ligne de code initialise une variable nommée prime\_numbers et lui affecte le résultat d'une compréhension de liste.
        \begin{itemize}
        	\item for x in range(1, 51) : Cette partie met en place une boucle qui parcourt les nombres de 1 à 50 (inclus). La fonction range(1, 51) génère une séquence de nombres commençant par 1 et se terminant par 50.
        	\item         if is\_prime(x) : Cette condition vérifie si la valeur actuelle de x est un nombre premier en appelant la fonction is\_prime(). Si c'est le cas, elle l'inclut dans la liste des nombres premiers.
        \end{itemize}
    \item print(prime\_numbers) : Cette ligne de code imprime la liste des nombres premiers sur la console.
   \end{itemize}
        \end{solution}
        

        \question
        Créer une liste de carrés de nombres pairs et de cubes de nombres impairs de -5 à 5

Exemple de résultat

[-125, 16, -27, 4, -1, 0, 1, 4, 27, 16, 125]
        \par
        \begin{solution}
            \renewcommand{\nomfichier}{q511.py}
            \pythonfile{\chemincode \nomfichier}[][\nomfichier]
            Ce code Python crée une liste appelée result en utilisant une compréhension de liste pour calculer le carré des nombres pairs et le cube des nombres impairs dans l'intervalle de -5 à 5. Voici comment le code fonctionne :

  \begin{itemize}
  	\item   result = [x**2 if x \% 2 == 0 else x**3 for x in range(-5, 6)] : Cette ligne de code initialise une variable nommée result et lui affecte le résultat d'une compréhension de liste.
  
     \begin{itemize}
     	\item    for x in range(-5, 6) : Cette partie met en place une boucle qui parcourt les nombres de -5 à 5 (inclus). La fonction range(-5, 6) génère une séquence de nombres commençant par -5 et se terminant par 5.
     	\item         x**2 if x \% 2 == 0 else x**3 : Pour chaque valeur de x dans l'intervalle, cette expression calcule le carré (x**2) ou le cube (x**3) du nombre selon que x est pair (x \% 2 == 0) ou impair.
     	\item{}         [x**2 if x \% 2 == 0 else x**3 for x in range(-5, 6)] : Il s'agit de la compréhension de la liste elle-même. Elle parcourt les nombres de l'intervalle spécifié (-5 à 5) et, pour chaque nombre, calcule son carré ou son cube selon qu'il est pair ou impair. Les résultats sont rassemblés dans une nouvelle liste.
     \end{itemize}
    \item print(result) : Cette ligne de code imprime la liste des résultats sur la console.\end{itemize}
        \end{solution}
        

        \question
        Générer une liste de chaînes de caractères avec leurs longueurs à partir d'une autre liste

Exemple de sortie

['pomme', 'banane', 'cerise']

[('pomme', 5), ('banane', 6), ('cerise', 6)]
        \par
        \begin{solution}
            \renewcommand{\nomfichier}{q512.py}
            \pythonfile{\chemincode \nomfichier}[][\nomfichier]
            Ce code Python crée une liste appelée word\_lengths en utilisant une compréhension de liste pour associer chaque mot de la liste des mots à sa longueur correspondante (nombre de caractères). Voici comment fonctionne le code :

 \begin{itemize}
 	\item    words = ["apple", "banana", "cherry"] : Cette ligne initialise une variable nommée words et lui affecte une liste contenant trois mots : "apple", "banana" et "cherry".
 	\item     word\_lengths = [(word, len(word)) for word in words] : Cette ligne de code initialise une variable nommée word\_lengths et lui affecte le résultat de la compréhension d'une liste.
 
    \begin{itemize}
    	\item     for word in words : Cette partie met en place une boucle qui parcourt chaque mot de la liste des mots.
    	\item         (word, len(word)) : Pour chaque mot de la liste, cette expression crée un tuple contenant deux éléments : le mot original et la longueur du mot len(word).
    	\item{}         [(word, len(word)) for word in words] : Il s'agit de la compréhension de la liste elle-même. Elle parcourt les mots de la liste words et, pour chaque mot, l'associe à sa longueur et inclut cette paire (tuple) dans la nouvelle liste.
    \end{itemize}
    \item print(words) : Cette ligne de code imprime la liste de mots originale sur la console.
    \item print(word\_lengths) : Cette ligne de code affiche la liste des word\_lengths sur la console.
    \end{itemize}
        \end{solution}
        

        \question
        Créer une liste de premiers caractères à partir d'une liste de mots

Exemple de sortie

['apple', 'banana', 'cherry']

['a', 'b', 'c']
        \par
        \begin{solution}
            \renewcommand{\nomfichier}{q513.py}
            \pythonfile{\chemincode \nomfichier}[][\nomfichier]
            Ce code Python crée une liste appelée first\_chars en utilisant une compréhension de liste pour extraire le premier caractère de chaque mot de la liste words. Voici comment fonctionne ce code :

   \begin{itemize}
   	\item  words = ["apple", "banana", "cherry"] : Cette ligne initialise une variable nommée words et lui assigne une liste contenant trois mots : "apple", "banana" et "cherry".
   	\item     first\_chars = [word[0] for word in words] : Cette ligne de code initialise une variable nommée first\_chars et lui affecte le résultat de la compréhension d'une liste.
   
       \begin{itemize}
       	\item  for word in words : Cette partie met en place une boucle qui parcourt chaque mot de la liste des mots.
       	\item         word[0] : Pour chaque mot de la liste, cette expression récupère le premier caractère du mot en utilisant l'indexation [0]. Cette indexation permet d'extraire le caractère situé à la position 0 de la chaîne, qui est le premier caractère.
       	\item{}         [word[0] for word in words] : Il s'agit de la compréhension de la liste elle-même. Elle parcourt les mots de la liste des mots et, pour chaque mot, extrait son premier caractère et l'inclut dans la nouvelle liste.
       \end{itemize}
    \item print(words) : Cette ligne de code imprime la liste de mots originale sur la console.
    \item print(first\_chars) : Cette ligne de code imprime la liste first\_chars sur la console.
    \end{itemize}
        \end{solution}
        

        \question
        Générer une liste de nombres avec leurs carrés si le nombre est pair

Exemple de sortie

[1, 2, 3, 4, 5, 6, 7, 8, 9, 10]

[4, 16, 36, 64, 100]
        \par
        \begin{solution}
            \renewcommand{\nomfichier}{q514.py}
            \pythonfile{\chemincode \nomfichier}[][\nomfichier]
            Ce code Python crée une liste appelée squared\_evens en utilisant une compréhension de liste pour calculer le carré des nombres pairs à partir de la liste des nombres. Voici comment fonctionne le code :

    nombres = [1, 2, 3, 4, 5, 6, 7, 8, 9, 10] : Cette ligne initialise une variable nommée numbers et lui affecte une liste contenant des nombres de 1 à 10.
    squared\_evens = [x**2 for x in numbers if x \% 2 == 0] : Cette ligne de code initialise une variable nommée squared\_evens et lui affecte le résultat d'une compréhension de liste.
        for x in numbers : Cette partie met en place une boucle qui parcourt chaque nombre de la liste des nombres.
        if x \% 2 == 0 : cette condition vérifie si le nombre x actuel est pair. S'il est pair (c'est-à-dire que son reste lorsqu'il est divisé par 2 est égal à 0), on passe à la partie suivante.
        x**2 : Pour chaque nombre pair, cette expression calcule son carré (x**2).
        [x**2 for x in numbers if x \% 2 == 0] : Il s'agit de la compréhension de la liste elle-même. Elle parcourt les nombres de la liste des nombres et, pour chaque nombre pair, calcule son carré et l'inclut dans la nouvelle liste.
    print(nombres) : Cette ligne de code imprime la liste originale des nombres sur la console.
    print(nombres\_pairs\_carrés) : Cette ligne de code affiche sur la console la liste squared\_evens.
        \end{solution}
        

        \question
        Créer une liste de mots en majuscules à partir d'une phrase

Exemple de sortie

Voici un exemple de phrase.

['CECI', 'EST', 'UN', 'ÉCHANTILLON', 'PHRASE'].
        \par
        \begin{solution}
            \renewcommand{\nomfichier}{q515.py}
            \pythonfile{\chemincode \nomfichier}[][\nomfichier]
            Ce code Python prend une phrase, la divise en mots et convertit chaque mot en majuscules à l'aide d'une compréhension de liste. Voici comment fonctionne ce code :

    sentence = "Ceci est un exemple de phrase" : Cette ligne initialise une variable nommée sentence et lui attribue la valeur "Ceci est un exemple de phrase".
    uppercase\_words = [word.upper() for word in sentence.split()] : Cette ligne de code initialise une variable nommée uppercase\_words et lui affecte le résultat d'une compréhension de liste.
        sentence.split() : Cette partie du code divise la phrase en une liste de mots. Par défaut, la phrase est divisée sur les espaces blancs, ce qui permet de séparer les mots.
        for word in sentence.split() : Cette partie met en place une boucle qui parcourt chaque mot de la liste de mots.
        word.upper() : Pour chaque mot de la liste, cette expression le convertit en majuscules à l'aide de la méthode .upper(). Cette méthode convertit tous les caractères de la chaîne en majuscules.
        [word.upper() for word in sentence.split()] : Il s'agit de la compréhension de la liste elle-même. Elle parcourt la liste des mots, convertit chaque mot en majuscule et inclut le mot en majuscule dans la nouvelle liste.
    print(phrase) : Cette ligne de code imprime la phrase originale sur la console.
    print(mots\_en\_majuscules) : Cette ligne de code affiche la liste des mots en majuscules sur la console.
        \end{solution}
        

        \question
        Générer une liste de chaînes de caractères dont les voyelles ont été supprimées

Exemple de sortie

['pomme', 'banane', 'cerise']

['ppl', 'bnn', 'chrry']
        \par
        \begin{solution}
            \renewcommand{\nomfichier}{q516.py}
            \pythonfile{\chemincode \nomfichier}[][\nomfichier]
            Ce code Python prend une liste de chaînes de caractères, supprime les voyelles de chaque chaîne et stocke les chaînes modifiées dans une nouvelle liste appelée no\_vowels. Voici comment fonctionne ce code :

    strings = ["apple", "banana", "cherry"] : Cette ligne initialise une variable nommée strings et lui assigne une liste contenant trois mots : "pomme", "banane" et "cerise".
    no\_vowels = [''.join([char for char in word if char.lower() not in 'aeiou']) for word in strings] : Cette ligne de code initialise une variable nommée no\_vowels et lui affecte le résultat d'une compréhension de liste.
        for word in strings : Cette partie met en place une boucle qui parcourt chaque mot de la liste des chaînes de caractères.
        for char in word if char.lower() not in 'aeiou' : Cette boucle interne parcourt chaque caractère (char) du mot actuel, mais uniquement si la version minuscule de char ne se trouve pas dans la chaîne "aeiou" (c'est-à-dire qu'elle filtre les voyelles). La méthode .lower() est utilisée pour traiter les voyelles majuscules et minuscules.
        ''.join(...) : Cette partie réunit les caractères filtrés pour former un mot modifié dont les voyelles ont été supprimées. ''.join(...) est utilisé pour concaténer les caractères sans espace ni séparateur.
        [''.join([char for char in word if char.lower() not in 'aeiou']) for word in strings] : Il s'agit de la compréhension de la liste elle-même. Elle parcourt les mots de la liste strings, supprime les voyelles de chaque mot et inclut les mots modifiés dans la nouvelle liste.
    print(strings) : Cette ligne de code imprime la liste originale des chaînes de caractères sur la console.
    print(no\_vowels) : Cette ligne de code affiche la liste no\_vowels sur la console.
        \end{solution}
        

        \question
        Créer une liste de nombres divisibles par 3 et 5 de 1 à 100

Exemple de résultat

[15, 30, 45, 60, 75, 90]
        \par
        \begin{solution}
            \renewcommand{\nomfichier}{q517.py}
            \pythonfile{\chemincode \nomfichier}[][\nomfichier]
            Ce code Python crée une liste appelée divisible\_par\_3\_et\_5 en utilisant une compréhension de liste pour trouver et stocker les nombres entre 1 et 100 qui sont divisibles à la fois par 3 et par 5. Voici comment fonctionne le code :

    divisible\_par\_3\_et\_5 = [x for x in range(1, 101) if x \% 3 == 0 and x \% 5 == 0] : Cette ligne de code initialise une variable nommée divisible\_par\_3\_et\_5 et lui affecte le résultat d'une compréhension de liste.
        for x in range(1, 101) : Cette partie met en place une boucle qui parcourt les nombres de 1 à 100 (inclus). La fonction range(1, 101) génère une séquence de nombres commençant par 1 et se terminant par 100.
        if x \% 3 == 0 and x \% 5 == 0 : Cette condition vérifie si la valeur actuelle de x est divisible à la fois par 3 et par 5. L'opérateur \% calcule le reste lorsque x est divisé par 3 et 5. Si le reste est égal à 0 pour les deux divisions, cela signifie que x est divisible à la fois par 3 et par 5.
        [x for x in range(1, 101) if x \% 3 == 0 and x \% 5 == 0] : Il s'agit de la compréhension de la liste elle-même. Elle parcourt les nombres de l'intervalle spécifié (1 à 100) et, pour chaque nombre qui est divisible par 3 et 5, l'inclut dans la nouvelle liste.
    print(divisible\_par\_3\_et\_5) : Cette ligne de code imprime la liste divisible\_par\_3\_et\_5 sur la console.
        \end{solution}
        

        \question
        Générer une liste de nombres dont les signes sont inversés

Exemple de sortie

[-2, 3, -5, 7, -11]

[2, -3, 5, -7, 11]
        \par
        \begin{solution}
            \renewcommand{\nomfichier}{q518.py}
            \pythonfile{\chemincode \nomfichier}[][\nomfichier]
            Ce code Python prend une liste de nombres, annule chaque nombre (change son signe en son opposé) et stocke les nombres annulés dans une nouvelle liste appelée signes\_opposés. Voici comment fonctionne ce code :

    nombres = [-2, 3, -5, 7, -11] : Cette ligne initialise une variable nommée numbers et lui affecte une liste contenant cinq nombres, dont certains sont négatifs.
    Signes\_opposés = [-x for x in numbers] : Cette ligne de code initialise une variable nommée opposite\_signs et lui affecte le résultat de la compréhension d'une liste.
        for x in numbers : Cette partie met en place une boucle qui parcourt chaque nombre x de la liste des nombres.
        -x : Pour chaque nombre de la liste, cette expression l'annule en plaçant un signe moins devant lui. Le signe de chaque nombre est donc inversé.
        [-x for x in numbers] : Il s'agit de la compréhension de la liste elle-même. Elle parcourt les nombres de la liste des nombres et, pour chaque nombre, l'annule et inclut le nombre annulé dans la nouvelle liste.
    print(nombres) : Cette ligne de code imprime la liste originale des nombres sur la console.
    print(signes\_opposés) : Cette ligne de code imprime la liste des signes\_opposés sur la console.
        \end{solution}
        

        \question
        Créer une liste de mots avec leur longueur à partir d'une phrase

Exemple de sortie

Voici un exemple de phrase.

[('This', 4), ('is', 2), ('a', 1), ('sample', 6), ('sentence.', 9)]
        \par
        \begin{solution}
            \renewcommand{\nomfichier}{q519.py}
            \pythonfile{\chemincode \nomfichier}[][\nomfichier]
            Ce code Python prend une phrase, la divise en mots et associe chaque mot à sa longueur correspondante (nombre de caractères). Il stocke ensuite ces paires dans une nouvelle liste appelée word\_lengths. Voici comment fonctionne le code :

    sentence = "Ceci est un exemple de phrase" : Cette ligne initialise une variable nommée sentence et lui attribue la valeur "Ceci est un exemple de phrase."
    word\_lengths = [(word, len(word)) for word in sentence.split()] : Cette ligne de code initialise une variable nommée word\_lengths et lui affecte le résultat d'une compréhension de liste.
        sentence.split() : Cette partie du code divise la phrase en une liste de mots. Par défaut, la phrase est divisée sur les espaces blancs, ce qui permet de séparer les mots.
        for word in sentence.split() : Cette partie met en place une boucle qui parcourt chaque mot de la liste de mots.
        (mot, len(mot)) : Pour chaque mot de la liste, cette expression crée un tuple contenant deux éléments : le mot original et la longueur du mot len(word) .
        [(word, len(word)) for word in sentence.split()] : Il s'agit de la compréhension de la liste elle-même. Elle parcourt les mots de la liste de phrases, associe chaque mot à sa longueur et inclut ces paires (tuples) dans la nouvelle liste.
    print(phrase) : Cette ligne de code imprime la phrase originale sur la console.
    print(longueur\_des\_mots) : Cette ligne de code imprime la liste des longueurs de mots sur la console.
        \end{solution}
        

        \question
        Générer une liste de nombres positifs à partir d'une autre liste

Exemple de sortie

[1, -2, 3, -4, 5, -6]

[1, 3, 5]
        \par
        \begin{solution}
            \renewcommand{\nomfichier}{q520.py}
            \pythonfile{\chemincode \nomfichier}[][\nomfichier]
            Ce code Python prend une liste de nombres, filtre les nombres positifs et les stocke dans une nouvelle liste appelée nombres\_positifs. Voici comment fonctionne ce code :

    nombres = [1, -2, 3, -4, 5, -6] : Cette ligne initialise une variable nommée nombres et lui affecte une liste contenant six nombres, dont certains sont négatifs.
    nombres\_positifs = [x for x in numbers if x > 0] : Cette ligne de code initialise une variable nommée nombres\_positifs et lui affecte le résultat de la compréhension d'une liste.
        for x in numbers : Cette partie met en place une boucle qui parcourt chaque nombre x de la liste des nombres.
        if x > 0 : cette condition vérifie si le nombre actuel x est supérieur à 0. Si x est positif, on passe à la partie suivante.
        [x for x in numbers if x > 0] : Il s'agit de la compréhension de la liste elle-même. Elle parcourt les nombres de la liste des nombres et, pour chaque nombre positif, l'inclut dans la nouvelle liste.
    print(nombres) : Cette ligne de code imprime la liste originale des nombres sur la console.
    print(nombres\_positifs) : Cette ligne de code imprime la liste des nombres positifs sur la console.
        \end{solution}
        

        \question
        Générer une liste de nombres qui sont des carrés parfaits de 1 à 100

Exemple de sortie

[1, 4, 9, 16, 25, 36, 49, 64, 81, 100]
        \par
        \begin{solution}
            \renewcommand{\nomfichier}{q521.py}
            \pythonfile{\chemincode \nomfichier}[][\nomfichier]
            Ce code Python crée une liste appelée perfect\_squares en utilisant une compréhension de liste pour trouver et stocker des nombres carrés parfaits entre 1 et 100. Voici comment fonctionne le code :

    perfect\_squares = [x for x in range(1, 101) if int(x**0.5)**2 == x] : Cette ligne de code initialise une variable nommée perfect\_squares et lui affecte le résultat d'une compréhension de liste.
        for x in range(1, 101) : Cette partie met en place une boucle qui parcourt les nombres de 1 à 100 (inclus). La fonction range(1, 101) génère une séquence de nombres commençant par 1 et se terminant par 100.
        if int(x**0.5)**2 == x : Il s'agit d'une condition qui vérifie si la valeur actuelle de x est un carré parfait. Pour déterminer si x est un carré parfait, il calcule la racine carrée de x en utilisant x**0,5, l'arrondit à l'entier le plus proche en utilisant int(), élève le résultat au carré et le compare au x original.
        [x for x in range(1, 101) if int(x**0.5)**2 == x] : Il s'agit de la compréhension de la liste elle-même. Elle parcourt les nombres de l'intervalle spécifié (1 à 100) et, pour chaque nombre qui est un carré parfait, l'inclut dans la nouvelle liste.
    print(carrés\_parfaits) : Cette ligne de code imprime la liste des carrés parfaits sur la console.
        \end{solution}
        

        \question
        Créer une liste de nombres avec leurs valeurs absolues

Exemple de sortie

[-2, 3, -5, 7, -11]

[2, 3, 5, 7, 11]
        \par
        \begin{solution}
            \renewcommand{\nomfichier}{q522.py}
            \pythonfile{\chemincode \nomfichier}[][\nomfichier]
            Ce code Python prend une liste de nombres, calcule leurs valeurs absolues et les stocke dans une nouvelle liste appelée absolute\_values. Voici comment fonctionne ce code :

    nombres = [-2, 3, -5, 7, -11] : Cette ligne initialise une variable nommée numbers et lui affecte une liste contenant cinq nombres, dont certains sont négatifs.
    valeurs\_absolues = [abs(x) for x in numbers] : Cette ligne de code initialise une variable nommée valeurs\_absolues et lui affecte le résultat de la compréhension d'une liste.
        for x in numbers : Cette partie met en place une boucle qui parcourt chaque nombre x de la liste des nombres.
        abs(x) : Pour chaque nombre de la liste, cette expression calcule sa valeur absolue à l'aide de la fonction abs(). La fonction abs() renvoie la magnitude (valeur positive) d'un nombre, en supprimant le signe négatif si le nombre est négatif.
        [abs(x) for x in numbers] : Il s'agit de la compréhension de la liste elle-même. Elle parcourt les nombres de la liste des nombres et, pour chaque nombre, calcule sa valeur absolue et l'inclut dans la nouvelle liste.
    print(nombres) : Cette ligne de code imprime la liste originale des nombres sur la console.
    print(valeurs\_absolues) : Cette ligne de code imprime la liste des valeurs absolues sur la console.
        \end{solution}
        

        \question
        Générer une liste de lettres majuscules en utilisant les valeurs ASCII

Exemple de sortie

['A', 'B', 'C', 'D', 'E', 'F', 'G', 'H', 'I', 'J', 'K', 'L', 'M', 'N', 'O', 'P', 'Q', 'R', 'S', 'T', 'U', 'V', 'W', 'X', 'Y', 'Z']
        \par
        \begin{solution}
            \renewcommand{\nomfichier}{q523.py}
            \pythonfile{\chemincode \nomfichier}[][\nomfichier]
            Ce code Python génère une liste appelée uppercase\_letters en utilisant une compréhension de liste pour créer des lettres majuscules de l'alphabet anglais. Pour ce faire, il utilise la fonction chr() pour convertir les valeurs ASCII en caractères. Voici comment fonctionne le code :

    uppercase\_letters = [chr(code) for code in range(65, 91)] : Cette ligne initialise une variable nommée uppercase\_letters et lui affecte le résultat d'une compréhension de liste.
        for code in range(65, 91) : Cette partie met en place une boucle qui parcourt les valeurs ASCII comprises entre 65 et 90 (inclus). Dans le tableau ASCII, ces valeurs correspondent aux lettres majuscules "A" à "Z".
        chr(code) : Pour chaque valeur ASCII comprise dans la plage spécifiée, cette expression utilise la fonction chr() pour la convertir en caractère correspondant. chr() prend une valeur ASCII en entrée et renvoie le caractère associé à cette valeur.
        [chr(code) for code in range(65, 91)] : Il s'agit de la compréhension de la liste elle-même. Elle parcourt les valeurs ASCII des lettres majuscules et inclut les caractères correspondants dans la nouvelle liste.
    print(lettres\_majuscules) : Cette ligne de code imprime la liste des lettres majuscules sur la console.
        \end{solution}
        

        \question
        Créer une liste de mots dont la longueur est supérieure à 3 à partir d'une phrase

Exemple de sortie

Voici un exemple de phrase.

['Ceci', 'échantillon', 'phrase.']
        \par
        \begin{solution}
            \renewcommand{\nomfichier}{q524.py}
            \pythonfile{\chemincode \nomfichier}[][\nomfichier]
            Ce code Python prend une phrase, la divise en mots et crée une nouvelle liste appelée mots\_longs contenant uniquement des mots de plus de 3 caractères. Voici comment fonctionne le code :

    sentence = "Ceci est un exemple de phrase" : Cette ligne initialise une variable nommée sentence et lui attribue la valeur "Ceci est un exemple de phrase".
    mots\_longs = [word for word in sentence.split() if len(word) > 3] : Cette ligne de code initialise une variable nommée mots\_longs et lui affecte le résultat d'une compréhension de liste.
        sentence.split() : Cette partie du code divise la phrase en une liste de mots. Par défaut, la phrase est divisée sur les espaces blancs, ce qui permet de séparer les mots.
        for word in sentence.split() : Cette partie met en place une boucle qui parcourt chaque mot de la liste de mots.
        if len(word) > 3 : il s'agit d'une condition qui vérifie si la longueur (nombre de caractères) du mot actuel est supérieure à 3.
        [word for word in sentence.split() if len(word) > 3] : Il s'agit de la compréhension de la liste elle-même. Elle parcourt les mots de la liste de phrases, n'inclut que les mots dont la longueur est supérieure à 3 et les inclut dans la nouvelle liste.
    print(phrase) : Cette ligne de code imprime la phrase originale sur la console.
    print(mots\_longs) : Cette ligne de code imprime la liste des mots longs sur la console.
        \end{solution}
        

        \question
        Générer une liste de carrés de nombres pairs de 1 à 20

Exemple de sortie

[4, 16, 36, 64, 100, 144, 196, 256, 324, 400]
        \par
        \begin{solution}
            \renewcommand{\nomfichier}{q525.py}
            \pythonfile{\chemincode \nomfichier}[][\nomfichier]
            Ce code Python crée une liste appelée even\_squares en utilisant une compréhension de liste pour calculer le carré des nombres pairs de 2 à 20. Voici comment fonctionne le code :

    even\_squares = [x**2 for x in range(2, 21, 2)] : Cette ligne de code initialise une variable nommée even\_squares et lui affecte le résultat d'une compréhension de liste.
        for x in range(2, 21, 2) : Cette partie met en place une boucle qui parcourt les nombres pairs de 2 à 20 (inclus). La fonction range(2, 21, 2) génère une séquence de nombres pairs commençant à 2 et se terminant à 20, avec un pas de 2.
        x**2 : Pour chaque nombre pair, cette expression calcule son carré (x**2).
        [x**2 for x in range(2, 21, 2)] : Il s'agit de la compréhension de la liste elle-même. Elle parcourt les nombres pairs dans l'intervalle spécifié et, pour chaque nombre pair, calcule son carré et l'inclut dans la nouvelle liste.
    print(even\_squares) : Cette ligne de code imprime la liste even\_squares sur la console.
        \end{solution}
        

        \question
        Créer une liste de caractères et leurs valeurs ASCII

Exemple de sortie

Bonjour à tous !

[('H', 72), ('e', 101), ('l', 108), ('l', 108), ('o', 111), (',', 44), (' ', 32), ('w', 119), ('o', 111), ('r', 114), ('l', 108), ('d', 100), ('!', 33)]
        \par
        \begin{solution}
            \renewcommand{\nomfichier}{q526.py}
            \pythonfile{\chemincode \nomfichier}[][\nomfichier]
            Ce code Python prend une chaîne de caractères, parcourt ses caractères et associe chaque caractère à sa valeur ASCII à l'aide d'une liste de compréhension. Voici comment fonctionne ce code :

    string = "Hello, world !": Cette ligne initialise une variable nommée string et lui affecte la valeur "Hello, world !".
    char\_ascii = [(char, ord(char)) for char in string] : Cette ligne de code initialise une variable nommée char\_ascii et lui affecte le résultat d'une compréhension de liste.
        for char in string : Cette partie met en place une boucle qui parcourt chaque caractère char de la chaîne.
        (char, ord(char)) : Pour chaque caractère de la chaîne, cette expression crée un tuple contenant deux éléments : le caractère original char et sa valeur ASCII obtenue à l'aide de la fonction ord(). La fonction ord() prend un caractère en entrée et renvoie la valeur ASCII correspondante.
        [(char, ord(char)) for char in string] : Il s'agit de la compréhension de la liste elle-même. Elle parcourt les caractères de la chaîne, associe chaque caractère à sa valeur ASCII et inclut ces paires (tuples) dans la nouvelle liste.
    print(string) : Cette ligne de code imprime la chaîne originale sur la console.
    print(char\_ascii) : cette ligne de code imprime la liste char\_ascii sur la console.
        \end{solution}
        

        \question
        Générer une liste de tuples contenant deux nombres dont la somme est paire

Exemple de sortie

[(1, 1), (1, 3), (1, 5), (1, 7), (1, 9), (2, 2), (2, 4), (2, 6), (2, 8), (2, 10), (3, 1), (3, 3), (3, 5), (3, 7), (3, 9), (4, 2), (4, 4), (4, 6), (4, 8), (4, 10), (5, 1), (5, 3), (5, 5), (5, 7), (5, 9), (6, 2), (6, 4), (6, 6), (6, 8), (6, 10), (7, 1), (7, 3), (7, 5), (7, 7), (7, 9), (8, 2), (8, 4), (8, 6), (8, 8), (8, 10), (9, 1), (9, 3), (9, 5), (9, 7), (9, 9), (10, 2), (10, 4), (10, 6), (10, 8), (10, 10)]
        \par
        \begin{solution}
            \renewcommand{\nomfichier}{q527.py}
            \pythonfile{\chemincode \nomfichier}[][\nomfichier]
            Ce code Python crée une liste appelée even\_sum\_tuples en utilisant une compréhension de liste imbriquée pour générer des tuples de paires de nombres entre 1 et 10 dont la somme est paire. Voici comment fonctionne le code :

    even\_sum\_tuples = [(x, y) for x in range(1, 11) for y in range(1, 11) if (x + y) \% 2 == 0] : Cette ligne de code initialise une variable appelée even\_sum\_tuples et lui affecte le résultat d'une compréhension de liste.
        for x in range(1, 11) : Cette partie du code met en place la boucle extérieure, qui parcourt les nombres de 1 à 10 (inclus). Elle génère des valeurs pour x.
        for y in range(1, 11) : Cette partie met en place la boucle interne, qui parcourt également les nombres de 1 à 10 (inclus). Elle génère des valeurs pour y.
        if (x + y) \% 2 == 0 : il s'agit d'une condition qui vérifie si la somme de x et de y est paire. Si la somme est paire (c'est-à-dire que le reste de la somme divisée par 2 est égal à 0), on passe à la partie suivante.
        [(x, y) for x in range(1, 11) for y in range(1, 11) if (x + y) \% 2 == 0] : Il s'agit de la compréhension de la liste elle-même. Elle parcourt toutes les paires possibles de nombres (x, y) de 1 à 10 et inclut dans la nouvelle liste les paires pour lesquelles la somme de x et de y est paire.
    print(even\_sum\_tuples) : Cette ligne de code affiche la liste even\_sum\_tuples sur la console.
        \end{solution}
        

        \question
        Générer une liste de paires de nombres où la somme de chaque paire est première.

Exemple de résultat

[(1, 1), (1, 2), (1, 4), (1, 6), (1, 10), (2, 1), (2, 3), (2, 5), (2, 9), (3, 2), (3, 4), (3, 8), (3, 10), (4, 1), (4, 3), (4, 7), (4, 9), (5, 2), (5, 6), (5, 8), (6, 1), (6, 5), (6, 7), (7, 4), (7, 6), (7, 10), (8, 3), (8, 5), (8, 9), (9, 2), (9, 4), (9, 8), (9, 10), (10, 1), (10, 3), (10, 7), (10, 9)]
        \par
        \begin{solution}
            \renewcommand{\nomfichier}{q528.py}
            \pythonfile{\chemincode \nomfichier}[][\nomfichier]
            Ce code Python définit une fonction is\_prime(n) pour vérifier si un nombre donné n est premier ou non. Il crée ensuite une liste appelée prime\_sum\_pairs à l'aide d'une compréhension de liste imbriquée pour générer des paires de nombres entre 1 et 10 dont la somme est un nombre premier. Voici comment fonctionne le code :

    def is\_prime(n) : Cette ligne définit une fonction nommée is\_prime qui prend un entier n en entrée et renvoie True si n est premier et False sinon. La fonction vérifie d'abord si n est inférieur ou égal à 1 et renvoie False dans ce cas. Elle parcourt ensuite les nombres compris entre 2 et la racine carrée de n et vérifie si n est divisible par l'un de ces nombres. S'il trouve un diviseur, il renvoie False. Si aucun diviseur n'est trouvé, il renvoie True, indiquant que n est premier.
    prime\_sum\_pairs = [(x, y) for x in range(1, 11) for y in range(1, 11) if is\_prime(x + y)] : Cette ligne de code initialise une variable nommée prime\_sum\_pairs et lui affecte le résultat de la compréhension d'une liste imbriquée.
        for x in range(1, 11) : Cette partie du code met en place la boucle extérieure, qui parcourt les nombres de 1 à 10 (inclus). Elle génère des valeurs pour x.
        for y in range(1, 11) : Cette partie met en place la boucle interne, qui parcourt également les nombres de 1 à 10 (inclus). Elle génère des valeurs pour y.
        if is\_prime(x + y) : Il s'agit d'une condition qui vérifie si la somme de x et de y est première en appelant la fonction is\_prime avec x + y comme argument.
        [(x, y) for x in range(1, 11) for y in range(1, 11) if is\_prime(x + y)] : Il s'agit de la compréhension de la liste imbriquée elle-même. Elle parcourt toutes les paires possibles de nombres (x, y) de 1 à 10 et inclut dans la nouvelle liste les paires pour lesquelles la somme de x et de y est première.
    print(prime\_sum\_pairs) : Cette ligne de code imprime la liste des paires prime\_sum\_pairs sur la console.
        \end{solution}
        

        \question
        Créer une liste de chaînes de caractères dont les premières lettres sont en majuscules

Exemple de sortie

['apple', 'banana', 'cherry']

['Pomme', 'Banane', 'Cerise']
        \par
        \begin{solution}
            \renewcommand{\nomfichier}{q529.py}
            \pythonfile{\chemincode \nomfichier}[][\nomfichier]
            Ce code Python prend une liste de chaînes de caractères, met en majuscule la première lettre de chaque mot de chaque chaîne et stocke les mots en majuscules dans une nouvelle liste appelée mots\_capitalisés. Voici comment fonctionne ce code :

    strings = ["apple", "banana", "cherry"] : Cette ligne initialise une variable nommée strings et lui affecte une liste contenant trois chaînes.
    mots\_capitalisés = [word.capitalize() for word in strings] : Cette ligne de code initialise une variable nommée mots\_capitalisés et lui affecte le résultat de la compréhension d'une liste.
        for word in strings : Cette partie met en place une boucle qui parcourt chaque mot de la liste strings.
        word.capitalize() : Pour chaque chaîne de la liste, cette expression utilise la méthode capitalize() pour mettre en majuscule la première lettre de la chaîne. La méthode capitalize() met le premier caractère de la chaîne en majuscule et tous les autres caractères de la chaîne en minuscule.
        [word.capitalize() for word in strings] : Il s'agit de la compréhension de la liste elle-même. Elle parcourt les chaînes de la liste strings, met en majuscules la première lettre de chaque mot de chaque chaîne et inclut les mots en majuscules dans la nouvelle liste.
    print(strings) : Cette ligne de code imprime la liste originale des chaînes sur la console.
    print(mots\_capitalisés) : Cette ligne de code affiche la liste des mots capitalisés sur la console.
        \end{solution}
        

        \question
        Générer une liste de tuples contenant des nombres et leurs carrés

Exemple de sortie

[(1, 1), (2, 4), (3, 9), (4, 16), (5, 25), (6, 36), (7, 49), (8, 64), (9, 81), (10, 100)]
        \par
        \begin{solution}
            \renewcommand{\nomfichier}{q530.py}
            \pythonfile{\chemincode \nomfichier}[][\nomfichier]
            Ce code Python crée une liste appelée num\_squares en utilisant une compréhension de liste pour générer des paires de nombres et leurs carrés. Voici comment fonctionne le code :

    num\_squares = [(x, x**2) for x in range(1, 11)] : Cette ligne de code initialise une variable nommée num\_squares et lui affecte le résultat d'une compréhension de liste.
        for x in range(1, 11) : Cette partie met en place une boucle qui parcourt les nombres de 1 à 10 (inclus). Elle génère des valeurs pour x.
        (x, x**2) : Pour chaque valeur de x, cette expression crée un tuple contenant deux éléments : la valeur originale x et son carré, calculé comme x**2.
        [(x, x**2) for x in range(1, 11)] : Il s'agit de la compréhension de la liste elle-même. Elle parcourt les nombres de l'intervalle spécifié (1 à 10) et associe chaque nombre à son carré, en incluant ces paires (tuples) dans la nouvelle liste.
    print(num\_squares) : Cette ligne de code imprime la liste num\_squares sur la console.
        \end{solution}
        

        \question
        Créer une liste de nombres où chaque nombre est doublé

Exemple de résultat

[1, 2, 3, 4, 5]

[2, 4, 6, 8, 10]
        \par
        \begin{solution}
            \renewcommand{\nomfichier}{q531.py}
            \pythonfile{\chemincode \nomfichier}[][\nomfichier]
            Ce code Python prend une liste de nombres, multiplie chaque nombre par 2 et stocke les nombres doublés dans une nouvelle liste appelée nombres\_doublés. Voici comment fonctionne ce code :

    nombres = [1, 2, 3, 4, 5] : Cette ligne initialise une variable nommée nombres et lui affecte une liste contenant cinq nombres.
    nombres\_doublés = [x * 2 pour x dans nombres] : Cette ligne de code initialise une variable nommée nombres\_doublés et lui affecte le résultat d'une compréhension de liste.
        for x in numbers : Cette partie met en place une boucle qui parcourt chaque nombre x de la liste des nombres.
        x * 2 : pour chaque nombre de la liste, cette expression calcule son double en multipliant x par 2.
        [x * 2 for x in numbers] : Il s'agit de la compréhension de la liste elle-même. Elle parcourt les nombres de la liste des nombres, double chaque nombre et inclut les nombres doublés dans la nouvelle liste.
    print(nombres) : Cette ligne de code imprime la liste originale des nombres sur la console.
    print(nombres\_doublés) : Cette ligne de code affiche la liste des nombres doublés sur la console.
        \end{solution}
        

        \question
        Créer une liste de caractères non alphanumériques à partir d'une chaîne de caractères

Exemple de sortie

Bonjour à tous !

[',', ' ', '!']
        \par
        \begin{solution}
            \renewcommand{\nomfichier}{q532.py}
            \pythonfile{\chemincode \nomfichier}[][\nomfichier]
            Ce code Python prend une chaîne, parcourt ses caractères et crée une nouvelle liste appelée non\_alphanumeric contenant les caractères qui ne sont pas alphanumériques (ni lettres ni chiffres). Voici comment fonctionne ce code :

    string = "Hello, world !": Cette ligne initialise une variable nommée string et lui affecte la valeur "Hello, world !".
    non\_alphanumeric = [char for char in string if not char.isalnum()] : Cette ligne de code initialise une variable nommée non\_alphanumérique et lui affecte le résultat d'une compréhension de liste.
        for char in string : Cette partie met en place une boucle qui parcourt chaque caractère char de la chaîne.
        if not char.isalnum() : Il s'agit d'une condition qui vérifie si le caractère courant char n'est pas alphanumérique. La méthode char.isalnum() renvoie True si char est un caractère alphanumérique (une lettre ou un chiffre) et False dans le cas contraire. Le mot-clé not annule cette condition et sélectionne donc les caractères qui ne sont pas alphanumériques.
        [char for char in string if not char.isalnum()] : Il s'agit de la compréhension de la liste elle-même. Elle parcourt les caractères de la chaîne, n'inclut que ceux qui ne sont pas alphanumériques et les inclut dans la nouvelle liste.
    print(string) : Cette ligne de code imprime la chaîne originale sur la console.
    print(non\_alphanumeric) : Cette ligne de code affiche la liste des caractères non alphanumériques sur la console.
        \end{solution}
        

        \question
        Générer une liste de nombres qui sont des puissances de 2 de 1 à 10

Exemple de sortie

[2, 4, 8, 16, 32, 64, 128, 256, 512, 1024]
        \par
        \begin{solution}
            \renewcommand{\nomfichier}{q533.py}
            \pythonfile{\chemincode \nomfichier}[][\nomfichier]
            Ce code Python génère une liste appelée powers\_of\_2 en utilisant une compréhension de liste pour calculer les puissances de 2 de $2^1$ à $2^10$. Voici comment fonctionne le code :

    puissances\_de\_2 = [2**x for x in range(1, 11)] : Cette ligne de code initialise une variable nommée puissances\_de\_2 et lui affecte le résultat d'une compréhension de liste.
        for x in range(1, 11) : Cette partie met en place une boucle qui parcourt les nombres de 1 à 10 (inclus). Elle génère des valeurs pour x.
        2**x : Pour chaque valeur de x, cette expression calcule 2 élevé à la puissance de x, ce qui est équivalent à $2^x$.
        [2**x for x in range(1, 11)] : Il s'agit de la compréhension de la liste elle-même. Elle parcourt les valeurs de x dans l'intervalle spécifié (1 à 10) et calcule $2^x$ pour chaque valeur, en incluant les résultats dans la nouvelle liste.
    print(puissances\_de\_2) : Cette ligne de code imprime la liste des puissances\_de\_2 sur la console.
        \end{solution}
        

        \question
        Créer une liste de chaînes dont les caractères sont en majuscules

Exemple de sortie

['apple', 'banana', 'cherry']

['POMME', 'BANANE', 'CERISE']
        \par
        \begin{solution}
            \renewcommand{\nomfichier}{q534.py}
            \pythonfile{\chemincode \nomfichier}[][\nomfichier]
            Ce code Python prend une liste de chaînes de caractères et crée une nouvelle liste appelée uppercase\_strings à l'aide d'une compréhension de liste. La nouvelle liste contient les mêmes mots, mais chaque mot est converti en majuscules à l'aide de la méthode upper(). Voici comment fonctionne le code :

    strings = ["apple", "banana", "cherry"] : Cette ligne initialise une variable nommée strings et lui affecte une liste contenant trois strings.
    uppercase\_strings = [word.upper() for word in strings] : Cette ligne de code initialise une variable nommée uppercase\_strings et lui affecte le résultat de la compréhension d'une liste.
        for word in strings : Cette partie met en place une boucle qui parcourt chaque mot de la liste des chaînes.
        word.upper() : Pour chaque chaîne de la liste, cette expression utilise la méthode upper() pour convertir toute la chaîne en majuscules. La méthode upper() convertit tous les caractères minuscules de la chaîne en leurs équivalents majuscules.
        [word.upper() for word in strings] : Il s'agit de la compréhension de la liste elle-même. Elle parcourt les chaînes de la liste strings, convertit chaque chaîne en majuscules et inclut les chaînes en majuscules dans la nouvelle liste.
    print(strings) : Cette ligne de code imprime la liste originale des chaînes de caractères sur la console.
    print(chaînes\_en\_majuscules) : Cette ligne de code affiche la liste des chaînes en majuscules sur la console.
        \end{solution}
        

        \question
        Générer une liste de tuples contenant des nombres pairs et impairs de 1 à 10

Exemple de sortie

[(1, 2), (3, 4), (5, 6), (7, 8), (9, 10)]
        \par
        \begin{solution}
            \renewcommand{\nomfichier}{q535.py}
            \pythonfile{\chemincode \nomfichier}[][\nomfichier]
            Ce code Python crée une liste appelée even\_odd\_pairs en utilisant une compréhension de liste pour générer des paires de nombres consécutifs dont l'un est pair et l'autre impair. Voici comment fonctionne ce code :

    even\_odd\_pairs = [(x, x + 1) for x in range(1, 11, 2)] : Cette ligne de code initialise une variable nommée even\_odd\_pairs et lui affecte le résultat d'une compréhension de liste.
        for x in range(1, 11, 2) : Cette partie met en place une boucle qui parcourt les nombres impairs de 1 à 10 (inclus) avec un pas de 2. Elle génère des valeurs pour x.
        (x, x + 1) : Pour chaque nombre impair x, cette expression crée un tuple contenant deux éléments : le nombre impair original x et le nombre consécutif suivant x + 1.
        [(x, x + 1) for x in range(1, 11, 2)] : Il s'agit de la compréhension de la liste elle-même. Elle parcourt les nombres impairs dans l'intervalle spécifié (1 à 10) et associe chaque nombre impair à son nombre pair consécutif, en incluant ces paires (tuples) dans la nouvelle liste.
    print(even\_odd\_pairs) : Cette ligne de code affiche la liste even\_odd\_pairs sur la console.
        \end{solution}
        

        \question
        Créer une liste de mots avec leur longueur à partir d'une autre liste

Exemple de résultat

['apple', 'banana', 'cherry']

[5, 6, 6]
        \par
        \begin{solution}
            \renewcommand{\nomfichier}{q536.py}
            \pythonfile{\chemincode \nomfichier}[][\nomfichier]
            Ce code Python prend une liste de mots et crée une nouvelle liste appelée word\_lengths à l'aide d'une compréhension de liste. La nouvelle liste contient les longueurs des mots de la liste originale. Voici comment fonctionne le code :

    mots = ["pomme", "banane", "cerise"] : Cette ligne initialise une variable nommée words et lui affecte une liste contenant trois mots.
    word\_lengths = [len(word) for word in words] : Cette ligne de code initialise une variable nommée word\_lengths et lui affecte le résultat de la compréhension d'une liste.
        for word in words : Cette partie met en place une boucle qui parcourt chaque mot de la liste des mots.
        len(word) : Pour chaque mot de la liste, cette expression calcule la longueur du mot à l'aide de la fonction len(). La fonction len() renvoie le nombre de caractères (lettres) d'une chaîne.
        [len(word) for word in words] : Il s'agit de la compréhension de la liste elle-même. Elle parcourt les mots de la liste, calcule la longueur de chaque mot et inclut ces longueurs dans la nouvelle liste.
    print(words) : Cette ligne de code imprime la liste de mots originale sur la console.
    print(word\_lengths) : Cette ligne de code affiche la liste word\_lengths sur la console.
        \end{solution}
        

        \question
        Générer une liste de tuples contenant des nombres et leurs signes

Exemple de sortie

[-2, 3, -5, 7, -11]

[(-2, "négatif"), (3, "positif"), (-5, "négatif"), (7, "positif"), (-11, "négatif")]
        \par
        \begin{solution}
            \renewcommand{\nomfichier}{q537.py}
            \pythonfile{\chemincode \nomfichier}[][\nomfichier]
            Ce code Python prend une liste de nombres et crée une nouvelle liste appelée num\_signs à l'aide d'une compréhension de liste. La nouvelle liste contient des paires de nombres et leur signe associé ("positif" ou "négatif") selon que le nombre est supérieur ou non à zéro. Voici comment fonctionne le code :

    nombres = [-2, 3, -5, 7, -11] : Cette ligne initialise une variable nommée numbers et lui affecte une liste contenant cinq nombres, dont certains sont négatifs.
    num\_signs = [(x, 'positif') if x > 0 else (x, 'négatif') for x in numbers] : Cette ligne de code initialise une variable nommée num\_signs et lui affecte le résultat de la compréhension d'une liste.
        for x in numbers : Cette partie met en place une boucle qui parcourt chaque nombre x dans la liste des nombres.
        (x, "positif") si x > 0 sinon (x, "négatif") : Pour chaque nombre de la liste, cette expression vérifie si x est supérieur à 0. Si x est supérieur à 0, elle crée un tuple contenant le nombre x et la chaîne "positive". Si x n'est pas supérieur à 0 (c'est-à-dire qu'il est nul ou négatif), elle crée un tuple contenant le nombre x et la chaîne "negative".
        [(x, 'positif') if x > 0 else (x, 'négatif') for x in numbers] : Il s'agit de la compréhension de la liste elle-même. Elle parcourt les nombres de la liste des nombres, détermine le signe de chaque nombre et associe chaque nombre au signe qui lui est associé, en incluant ces paires (tuples) dans la nouvelle liste.
    print(nombres) : Cette ligne de code imprime la liste originale des nombres sur la console.
    print(num\_signs) : Cette ligne de code affiche la liste num\_signs sur la console.
        \end{solution}
        

        \question
        Créer une liste de chaînes de caractères dont les voyelles sont remplacées par des astérisques.

Exemple de résultat

['apple', 'banana', 'cherry']

['*ppl*', 'b*n*n*', 'ch*rry']
        \par
        \begin{solution}
            \renewcommand{\nomfichier}{q538.py}
            \pythonfile{\chemincode \nomfichier}[][\nomfichier]
            
        \end{solution}
        

        \question
        Générer une liste de chaînes de caractères dont les premières lettres ont été supprimées

Exemple de sortie

['apple', 'banana', 'cherry']

['pple', 'anana', 'herry']
        \par
        \begin{solution}
            \renewcommand{\nomfichier}{q539.py}
            \pythonfile{\chemincode \nomfichier}[][\nomfichier]
            Ce code Python prend une liste de chaînes de caractères et crée une nouvelle liste appelée without\_first\_letters à l'aide d'une compréhension de liste. La nouvelle liste contient les mêmes mots que la liste originale, mais sans les premières lettres. Voici comment fonctionne le code :

    strings = ["apple", "banana", "cherry"] : Cette ligne initialise une variable nommée strings et lui affecte une liste contenant trois mots.
    without\_first\_letters = [word[1 :] for word in strings] : Cette ligne de code initialise une variable nommée without\_first\_letters et lui affecte le résultat d'une compréhension de liste.
        for word in strings : Cette partie met en place une boucle qui parcourt chaque mot de la liste des chaînes de caractères.
        word[1 :]: Pour chaque mot de la liste, cette expression découpe le mot à partir du deuxième caractère (index 1) et inclut tous les caractères après le premier.
        [word[1 :] for word in strings] : Il s'agit de la compréhension de la liste elle-même. Elle parcourt les mots de la liste strings, supprime la première lettre de chaque mot et inclut les mots modifiés dans la nouvelle liste.
    print(strings) : Cette ligne de code imprime la liste originale des chaînes de caractères sur la console.
    print(sans\_premières\_lettres) : Cette ligne de code affiche la liste sans\_premières\_lettres sur la console.
        \end{solution}
        

        \question
        Créer une liste de nombres avec leurs valeurs réciproques

Exemple de résultat

[2, 3, 4, 5, 6]

[0.5, 0.3333333333333333, 0.25, 0.2, 0.16666666666666666]
        \par
        \begin{solution}
            \renewcommand{\nomfichier}{q540.py}
            \pythonfile{\chemincode \nomfichier}[][\nomfichier]
            Ce code Python prend une liste de nombres et crée une nouvelle liste appelée reciprocal\_values à l'aide d'une compréhension de liste. La nouvelle liste contient les valeurs réciproques (inverses) des nombres de la liste originale. Voici comment fonctionne le code :

    nombres = [2, 3, 4, 5, 6] : Cette ligne initialise une variable nommée numbers et lui affecte une liste contenant cinq nombres.
    valeurs\_réciproques = [1/x pour x dans nombres] : Cette ligne de code initialise une variable nommée valeurs\_réciproques et lui affecte le résultat d'une compréhension de liste.
        for x in numbers : Cette partie met en place une boucle qui parcourt chaque nombre x dans la liste des nombres.
        1/x : Pour chaque nombre de la liste, cette expression calcule la valeur réciproque (inverse) en divisant 1 par x.
        [1/x for x in numbers] : Il s'agit de la compréhension de la liste elle-même. Elle parcourt les nombres de la liste, calcule la valeur réciproque de chaque nombre et inclut ces valeurs réciproques dans la nouvelle liste.
    print(nombres) : Cette ligne de code imprime la liste originale des nombres sur la console.
    print(valeurs\_réciproques) : Cette ligne de code imprime la liste des valeurs réciproques sur la console.
        \end{solution}
        

        \question
        Générer une liste de tuples contenant des nombres et leurs carrés si le nombre est premier.

Exemple de sortie

[(2, 4), (3, 9), (5, 25), (7, 49)]
        \par
        \begin{solution}
            \renewcommand{\nomfichier}{q541.py}
            \pythonfile{\chemincode \nomfichier}[][\nomfichier]
            Ce code Python définit une fonction is\_prime(n) qui vérifie si un nombre donné n est premier ou non. Il utilise ensuite une compréhension de liste pour générer des paires de nombres premiers et leurs carrés dans un intervalle spécifié. Voici comment fonctionne le code :

    def is\_prime(n) : Cette ligne définit une fonction nommée is\_prime qui prend un seul argument n et renvoie True si n est premier et False sinon.
        if n <= 1: : Cette ligne vérifie si n est inférieur ou égal à 1. Si c'est le cas, la fonction renvoie immédiatement False car 1 et tous les nombres négatifs ne sont pas premiers par définition.
        for i in range(2, int(n**0.5) + 1) : Cette ligne met en place une boucle qui parcourt les nombres de 2 jusqu'à la racine carrée de n (incluse).
        if n \% i == 0: : À l'intérieur de la boucle, la fonction vérifie si n est divisible par i (c'est-à-dire si n modulo i est égal à 0). Si c'est le cas, la fonction renvoie immédiatement False car n n'est pas premier s'il a un diviseur autre que 1 et lui-même.
        Si la boucle se termine sans trouver d'autres diviseurs que 1 et n, la fonction renvoie True, indiquant que n est premier.
    prime\_num\_squares = [(x, x**2) for x in range(1, 11) if is\_prime(x)] : Cette ligne de code initialise une variable nommée prime\_num\_squares et lui affecte le résultat d'une compréhension de liste.
        for x in range(1, 11) : Cette partie met en place une boucle qui parcourt les nombres de 1 à 10 (inclus). Elle génère des valeurs pour x.
        (x, x**2) : Pour chaque valeur de x, cette expression crée un tuple contenant deux éléments : la valeur originale x et son carré, calculé comme x**2.
        if is\_prime(x) : Cette condition vérifie si la valeur actuelle de x est première en appelant la fonction is\_prime. Si x est premier, la paire $(x, x^2)$ est incluse dans la nouvelle liste.
    print(prime\_num\_squares) : Cette ligne de code affiche la liste prime\_num\_squares sur la console.
        \end{solution}
        

        \question
        Créer une liste de mots avec leurs caractères triés.

Exemple de sortie

['apple', 'banana', 'cherry']

['aelpp', 'aaabnn', 'cehrry']
        \par
        \begin{solution}
            \renewcommand{\nomfichier}{q542.py}
            \pythonfile{\chemincode \nomfichier}[][\nomfichier]
            Ce code Python prend une liste de mots et crée une nouvelle liste appelée sorted\_chars à l'aide d'une compréhension de liste. La nouvelle liste contient les mêmes mots que la liste originale, mais avec leurs caractères triés par ordre alphabétique. Voici comment fonctionne le code :

    mots = ["pomme", "banane", "cerise"] : Cette ligne initialise une variable nommée words et lui affecte une liste contenant trois mots.
    sorted\_chars = [''.join(sorted(word)) for word in words] : Cette ligne de code initialise une variable nommée sorted\_chars et lui affecte le résultat d'une compréhension de liste.
        for word in words : Cette partie met en place une boucle qui parcourt chaque mot de la liste des mots.
        sorted(word) : Pour chaque mot de la liste, cette expression trie ses caractères par ordre alphabétique à l'aide de la fonction sorted(). La fonction sorted() renvoie une liste de caractères triés.
        ''.join(sorted(word)) : Cette partie réunit les caractères triés en une seule chaîne de caractères à l'aide de la méthode join(). Le résultat est un mot dont les caractères sont triés par ordre alphabétique.
        [''.join(sorted(word)) for word in words] : Il s'agit de la compréhension de la liste elle-même. Elle parcourt les mots de la liste words, trie les caractères de chaque mot et inclut les mots triés dans la nouvelle liste.
    print(words) : Cette ligne de code imprime la liste de mots originale sur la console.
    print(sorted\_chars) : Cette ligne de code affiche la liste des caractères triés sur la console.
        \end{solution}
        

        \question
        Générer une liste de tuples contenant des nombres et leurs cubes

Exemple de sortie

[(1, 1), (2, 8), (3, 27), (4, 64), (5, 125), (6, 216), (7, 343), (8, 512), (9, 729), (10, 1000)]
        \par
        \begin{solution}
            \renewcommand{\nomfichier}{q543.py}
            \pythonfile{\chemincode \nomfichier}[][\nomfichier]
            Ce code Python utilise une compréhension de liste pour générer des paires de nombres et leurs cubes pour des valeurs de x allant de 1 à 10. Voici comment fonctionne le code :

    num\_cubes = [(x, x**3) for x in range(1, 11)] : Cette ligne de code initialise une variable nommée num\_cubes et lui affecte le résultat d'une compréhension de liste.
        for x in range(1, 11) : Cette partie met en place une boucle qui parcourt les nombres de 1 à 10 (inclus). Elle génère des valeurs pour x.
        (x, x**3) : Pour chaque valeur de x, cette expression crée un tuple contenant deux éléments : la valeur originale x et son cube, calculé comme x**3.
        [(x, x**3) for x in range(1, 11)] : Il s'agit de la compréhension de la liste elle-même. Elle parcourt les valeurs de x dans l'intervalle spécifié (1 à 10) et associe chaque valeur à son cube, en incluant ces paires (tuples) dans la nouvelle liste.
    print(num\_cubes) : Cette ligne de code imprime la liste num\_cubes sur la console.
        \end{solution}
        

        \question
        Créer une liste de voyelles minuscules à partir d'une chaîne de caractères

Exemple de résultat

Bonjour à tous !

['e', 'o', 'o']
        \par
        \begin{solution}
            \renewcommand{\nomfichier}{q544.py}
            \pythonfile{\chemincode \nomfichier}[][\nomfichier]
            Ce code Python prend une chaîne et crée une nouvelle liste appelée voyelles en utilisant une compréhension de liste. La nouvelle liste contient toutes les voyelles (minuscules et majuscules) de la chaîne originale. Voici comment fonctionne le code :

    string = "Hello, world !": Cette ligne initialise une variable nommée string et lui affecte une chaîne contenant le texte "Hello, world !"
    voyelles = [char for char in string if char.lower() in 'aeiou'] : Cette ligne de code initialise une variable nommée voyelles et lui affecte le résultat d'une compréhension de liste.
        for char in string : Cette partie met en place une boucle qui parcourt chaque caractère char de la chaîne.
        char.lower() in 'aeiou' : Pour chaque caractère de la chaîne, cette expression convertit d'abord le caractère en minuscules à l'aide de la méthode lower() afin de garantir l'insensibilité à la casse. Elle vérifie ensuite si le caractère minuscule se trouve dans la chaîne "aeiou", qui contient toutes les voyelles minuscules.
        [char for char in string if char.lower() in 'aeiou'] : Il s'agit de la compréhension de la liste elle-même. Elle parcourt les caractères de la chaîne, vérifie si chaque caractère est une voyelle minuscule et inclut les voyelles dans la nouvelle liste.
    print(string) : Cette ligne de code imprime la chaîne originale sur la console.
    print(voyelles) : Cette ligne de code imprime la liste des voyelles sur la console.
        \end{solution}
        

        \question
        Créer une liste de nombres avec leurs racines carrées

Exemple de résultat

[1, 4, 9, 16, 25]

[1.0, 2.0, 3.0, 4.0, 5.0]
        \par
        \begin{solution}
            \renewcommand{\nomfichier}{q545.py}
            \pythonfile{\chemincode \nomfichier}[][\nomfichier]
            Ce code Python calcule la racine carrée de chaque nombre d'une liste à l'aide de la fonction math.sqrt() et stocke les résultats dans une nouvelle liste. Voici comment fonctionne le code :

    import math : Cette ligne importe le module math, qui contient diverses fonctions et constantes mathématiques, dont la fonction sqrt() pour le calcul des racines carrées.
    numbers = [1, 4, 9, 16, 25] : Cette ligne initialise une variable nommée numbers et lui affecte une liste contenant cinq nombres.
    racines\_carrées = [math.sqrt(x) for x in numbers] : Cette ligne de code initialise une variable nommée racines\_carrées et lui affecte le résultat de la compréhension d'une liste.
        for x in numbers : Cette partie met en place une boucle qui parcourt chaque nombre x dans la liste des nombres.
        math.sqrt(x) : Pour chaque nombre de la liste, cette expression calcule la racine carrée de x à l'aide de la fonction math.sqrt() du module math.
        [math.sqrt(x) for x in numbers] : Il s'agit de la compréhension de la liste elle-même. Elle parcourt les nombres de la liste des nombres, calcule la racine carrée de chaque nombre et inclut ces racines carrées dans la nouvelle liste.
    print(nombres) : Cette ligne de code imprime la liste originale des nombres sur la console.
    print(racines\_carrées) : Cette ligne de code affiche la liste des racines carrées sur la console.
        \end{solution}
        

        \question
        Générer une liste de nombres palindromes de 1 à 100

Exemple de sortie

[1, 2, 3, 4, 5, 6, 7, 8, 9, 11, 22, 33, 44, 55, 66, 77, 88, 99]
        \par
        \begin{solution}
            \renewcommand{\nomfichier}{q546.py}
            \pythonfile{\chemincode \nomfichier}[][\nomfichier]
            Ce code Python génère une liste appelée palindromes en utilisant une compréhension de liste. La liste contient des nombres de 1 à 100 qui sont des palindromes lorsque leurs chiffres sont inversés. Voici comment fonctionne le code :

    palindromes = [x for x in range(1, 101) if str(x) == str(x)[::-1]] : Cette ligne de code initialise une variable nommée palindromes et lui affecte le résultat d'une compréhension de liste.
        for x in range(1, 101) : Cette partie met en place une boucle qui parcourt les nombres de 1 à 100 (inclus). Elle génère des valeurs pour x.
        str(x) == str(x)[::-1] : Pour chaque nombre de la plage, cette expression convertit x en chaîne de caractères à l'aide de str(x) , puis vérifie si la représentation de x sous forme de chaîne est égale à son inverse, obtenu par str(x)[::-1]. Cette comparaison détermine si x est un palindrome ou non.
        [x for x in range(1, 101) if str(x) == str(x)[::-1]] : Il s'agit de la compréhension de la liste elle-même. Elle parcourt les nombres de l'intervalle spécifié (1 à 100), vérifie si chaque nombre est un palindrome et inclut les nombres palindromiques dans la nouvelle liste.
    print(palindromes) : Cette ligne de code imprime la liste des palindromes sur la console.
        \end{solution}
        

        \question
        Créer une liste de nombres avec leurs valeurs factorielles

Exemple de résultat

[2, 3, 4, 5]

[2, 6, 24, 120]
        \par
        \begin{solution}
            \renewcommand{\nomfichier}{q547.py}
            \pythonfile{\chemincode \nomfichier}[][\nomfichier]
            Ce code Python calcule la factorielle de chaque nombre d'une liste à l'aide de la fonction math.factorial() du module math et stocke les résultats dans une nouvelle liste. Voici comment fonctionne le code :

    import math : Cette ligne importe le module math, qui contient diverses fonctions mathématiques, dont la fonction factorial() pour le calcul des factorielles.
    numbers = [2, 3, 4, 5] : Cette ligne initialise une variable nommée numbers et lui affecte une liste contenant quatre nombres.
    factorielles = [math.factorial(x) for x in numbers] : Cette ligne de code initialise une variable nommée factorials et lui affecte le résultat de la compréhension d'une liste.
        for x in numbers : Cette partie met en place une boucle qui parcourt chaque nombre x de la liste des nombres.
        math.factorial(x) : Pour chaque nombre de la liste, cette expression calcule sa factorielle à l'aide de la fonction math.factorial() du module math.
        [math.factorial(x) for x in numbers] : Il s'agit de la compréhension de la liste elle-même. Elle parcourt les nombres de la liste des nombres, calcule la factorielle de chaque nombre et inclut ces factorielles dans la nouvelle liste.
    print(nombres) : Cette ligne de code imprime la liste originale des nombres sur la console.
    print(factorials) : Cette ligne de code imprime la liste des factorielles sur la console.
        \end{solution}
        

        \question
        Générer une liste de chaînes de caractères dont les voyelles ont été supprimées d'une phrase

Exemple de sortie

Voici un exemple de phrase avec quelques voyelles.

['Ths', 's', '', 'smpl', 'sntnc', 'wth', 'sm', 'vwls'].
        \par
        \begin{solution}
            \renewcommand{\nomfichier}{q548.py}
            \pythonfile{\chemincode \nomfichier}[][\nomfichier]
            Ce code Python prend une phrase, la divise en mots et crée une nouvelle phrase dans laquelle toutes les voyelles (minuscules et majuscules) sont supprimées de chaque mot. Voici comment fonctionne le code :

    phrase = "Ceci est un exemple de phrase avec quelques voyelles" : Cette ligne initialise une variable nommée sentence et lui assigne une chaîne contenant la phrase d'entrée.
    no\_vowels = [''.join([char for char in word if char.lower() not in 'aeiou']) for word in sentence.split()] : Cette ligne de code initialise une variable nommée no\_vowels et lui affecte le résultat d'une compréhension de liste.
        for word in sentence.split() : Cette partie met en place une boucle qui parcourt chaque mot de la phrase. Elle divise la phrase en mots en utilisant sentence.split().
        [char for char in word if char.lower() not in 'aeiou'] : Pour chaque mot de la liste, cette expression parcourt chaque caractère char du mot et l'inclut dans une nouvelle liste uniquement s'il ne s'agit pas d'une voyelle. Elle vérifie que la version minuscule du caractère n'est pas dans la chaîne 'aeiou'.
        ''.join([char for char in word if char.lower() not in 'aeiou']) : Cette partie réunit les caractères de la liste (sans les voyelles) en une seule chaîne, formant ainsi un mot sans les voyelles.
        [''.join([char for char in word if char.lower() not in 'aeiou']) for word in sentence.split()] : Il s'agit de la compréhension de la liste elle-même. Elle parcourt les mots de la phrase, supprime les voyelles de chaque mot et inclut les mots modifiés dans la nouvelle liste.
    print(sentence) : Cette ligne de code imprime la phrase originale sur la console.
    print(no\_vowels) : Cette ligne de code imprime la liste no\_vowels (qui contient les mots modifiés) sur la console.
        \end{solution}
        

        \question
        Créer une liste de caractères qui sont des chiffres à partir d'une chaîne de caractères

Exemple de sortie

12345Bonjour67890

['1', '2', '3', '4', '5', '6', '7', '8', '9', '0']
        \par
        \begin{solution}
            \renewcommand{\nomfichier}{q549.py}
            \pythonfile{\chemincode \nomfichier}[][\nomfichier]
            Ce code Python prend une chaîne et crée une nouvelle liste appelée digits en utilisant une compréhension de liste. La nouvelle liste ne contient que les chiffres de la chaîne originale. Voici comment fonctionne le code :

    string = "12345Hello67890" : Cette ligne initialise une variable nommée chaîne et lui attribue une chaîne contenant un mélange de chiffres et de caractères autres que des chiffres.
    digits = [char for char in string if char.isdigit()] : Cette ligne de code initialise une variable nommée digits et lui affecte le résultat d'une compréhension de liste.
        for char in string : Cette partie met en place une boucle qui parcourt chaque caractère char de la chaîne.
        char.isdigit() : Pour chaque caractère de la chaîne, cette expression vérifie si le caractère est un chiffre en utilisant la méthode isdigit(), qui renvoie True si le caractère est un chiffre et False dans le cas contraire.
        [char for char in string if char.isdigit()] : Il s'agit de la compréhension de la liste elle-même. Elle parcourt les caractères de la chaîne et n'inclut dans la nouvelle liste que les caractères qui sont des chiffres.
    print(string) : Cette ligne de code imprime la chaîne originale sur la console.
    print(chiffres) : Cette ligne de code imprime la liste des chiffres (qui contient les caractères numériques) sur la console.
        \end{solution}
        

        \question
        Liste d'éléments avec leur fréquence dans une liste

Exemple de sortie

[1, 2, 2, 3, 4, 4, 4, 5]

\{1 : 1, 2 : 2, 3 : 1, 4 : 3, 5 : 1\}
        \par
        \begin{solution}
            \renewcommand{\nomfichier}{q550.py}
            \pythonfile{\chemincode \nomfichier}[][\nomfichier]
            Ce code Python calcule la fréquence de chaque élément d'une liste et stocke les résultats dans un dictionnaire appelé element\_frequencies. Voici comment fonctionne le code :

    nombres = [1, 2, 2, 3, 4, 4, 4, 5] : Cette ligne initialise une variable nommée numbers et lui assigne une liste contenant plusieurs nombres, y compris quelques doublons.
    element\_frequencies = {num : numbers.count(num) for num in set(numbers)} : Cette ligne de code initialise une variable nommée fréquences\_éléments et lui affecte le résultat de la compréhension d'un dictionnaire.
        set(numbers) : Cette partie convertit la liste des nombres en un ensemble, ce qui permet de supprimer les éléments en double et de ne conserver que les éléments uniques. Cette étape garantit que chaque élément unique n'est compté qu'une seule fois.
        {num : numbers.count(num) for num in set(numbers)} : Il s'agit de la compréhension du dictionnaire proprement dite. Elle parcourt les éléments uniques de l'ensemble et, pour chaque élément (num), compte le nombre de fois qu'il apparaît dans la liste originale numbers à l'aide de la méthode numbers.count(num). Le résultat est une paire clé-valeur dans le dictionnaire, où la clé est l'élément et la valeur est sa fréquence.
    print(numbers) : Cette ligne de code imprime la liste originale des nombres sur la console.
    print(element\_frequencies) : Cette ligne de code imprime le dictionnaire element\_frequencies (qui contient les fréquences des éléments) sur la console.
        \end{solution}
        

        \question
        Liste de mots dont la première et la dernière lettre ont été interverties

Exemple de sortie

['apple', 'banana', 'cherry', 'date']

['eppla', 'aananb', 'yherrc', 'eatd']
        \par
        \begin{solution}
            \renewcommand{\nomfichier}{q551.py}
            \pythonfile{\chemincode \nomfichier}[][\nomfichier]
            Ce code Python prend une liste de mots et crée une nouvelle liste appelée swapped\_words à l'aide d'une compréhension de liste. Dans la nouvelle liste, la première et la dernière lettre de chaque mot sont interverties. Voici comment fonctionne le code :

    mots = ["pomme", "banane", "cerise", "date"] : Cette ligne initialise une variable nommée words et lui affecte une liste contenant quatre mots.
    swapped\_words = [word[-1] + word[1:-1] + word[0] for word in words] : Cette ligne de code initialise une variable nommée swapped\_words et lui affecte le résultat de la compréhension d'une liste.
        for word in words : Cette partie met en place une boucle qui parcourt chaque mot de la liste des mots.
        word[-1] : Cette partie extrait le dernier caractère du mot en utilisant l'indexation négative (-1).
        mot[1:-1] : Cette partie extrait les caractères du mot à partir du deuxième caractère (index 1) jusqu'au dernier caractère (index -1).
        word[0] : Cette partie extrait le premier caractère du mot en utilisant l'indexation (0).
        mot[-1] + mot[1:-1] + mot[0] : Ces parties combinent le dernier caractère, les caractères du milieu et le premier caractère pour former un nouveau mot dont la première et la dernière lettre sont interverties.
        [mot[-1] + mot[1:-1] + mot[0] pour mot dans mots] : Il s'agit de la compréhension de la liste elle-même. Elle parcourt les mots de la liste des mots et crée de nouveaux mots dont la première et la dernière lettre sont interverties, en incluant ces mots modifiés dans la nouvelle liste.
    print(mots) : Cette ligne de code imprime la liste de mots originale sur la console.
    print(mots\_échangés) : Cette ligne de code imprime la liste swapped\_words (qui contient les mots modifiés) sur la console.
        \end{solution}
        

        \question
        Liste des nombres avec leurs diviseurs

Exemple de sortie

[10, 15, 20, 25]

\{10 : [1, 2, 5, 10], 15 : [1, 3, 5, 15], 20 : [1, 2, 4, 5, 10, 20], 25 : [1, 5, 25]\}
        \par
        \begin{solution}
            \renewcommand{\nomfichier}{q552.py}
            \pythonfile{\chemincode \nomfichier}[][\nomfichier]
            Ce code Python crée un dictionnaire appelé diviseurs où chaque paire clé-valeur représente un nombre de la liste des nombres et ses diviseurs. Voici comment fonctionne le code :

    nombres = [10, 15, 20, 25] : Cette ligne initialise une variable nommée numbers et lui affecte une liste contenant quatre nombres.
    diviseurs = \{num : [x for x in range(1, num+1) if num \% x == 0] for num in numbers\} : Cette ligne de code initialise une variable nommée diviseurs et lui affecte le résultat d'une compréhension du dictionnaire.
        for num in numbers : Cette partie met en place une boucle qui parcourt chaque nombre num dans la liste des nombres.
        range(1, num+1) : Cette partie crée une plage de nombres allant de 1 à num (inclus). Il s'agit des diviseurs potentiels de num.
        [x for x in range(1, num+1) if num \% x == 0] : Cette liste de compréhension parcourt les nombres de la plage et n'inclut que les nombres qui sont des diviseurs de num. Elle vérifie si num est divisible par x (c'est-à-dire si num \% x == 0).
        \{num : [x for x in range(1, num+1) if num \% x == 0] for num in numbers\} : Il s'agit de la compréhension du dictionnaire proprement dite. Il parcourt les nombres de la liste des nombres, calcule les diviseurs de chaque nombre et les stocke sous forme de paires clé-valeur dans le dictionnaire des diviseurs.
    print(nombres) : Cette ligne de code imprime la liste originale des nombres sur la console.
    print(diviseurs) : Cette ligne de code imprime le dictionnaire des diviseurs (qui contient les diviseurs de chaque nombre) sur la console.
        \end{solution}
        

        \question
        Liste des caractères qui sont des voyelles ou des consonnes

Exemple de sortie

['a', 'b', 'c', 'e', 'f', 'i', 'o']

['a', 'e', 'i', 'o']

['b', 'c', 'f']
        \par
        \begin{solution}
            \renewcommand{\nomfichier}{q553.py}
            \pythonfile{\chemincode \nomfichier}[][\nomfichier]
            Ce code Python prend une liste de caractères et les sépare en deux listes : l'une contenant les voyelles et l'autre les consonnes. Voici comment fonctionne le code :

    characters = ['a', 'b', 'c', 'e', 'f', 'i', 'o'] : Cette ligne initialise une variable nommée characters et lui affecte une liste contenant plusieurs caractères, dont des voyelles et des consonnes.
    voyelles = [char for char in characters if char.lower() in 'aeiou'] : Cette ligne de code initialise une variable nommée voyelles et lui affecte le résultat de la compréhension d'une liste.
        for char in characters : Cette partie met en place une boucle qui parcourt chaque caractère char dans la liste des caractères.
        char.lower() in 'aeiou' : Pour chaque caractère de la liste, cette expression convertit char en minuscules à l'aide de char.lower() pour garantir l'insensibilité à la casse et vérifie si le caractère minuscule se trouve dans la chaîne "aeiou", qui contient toutes les voyelles minuscules.
        [char for char in characters if char.lower() in 'aeiou'] : Il s'agit de la compréhension de la liste elle-même. Elle parcourt les caractères de la liste des caractères et n'inclut que les caractères qui sont des voyelles dans la nouvelle liste.
    consonnes = [char for char in characters if char.lower() not in 'aeiou'] : Cette ligne de code initialise une variable nommée consonnes et lui affecte le résultat d'une compréhension de liste.
        for char in characters : Cette partie met en place une boucle qui parcourt chaque caractère char de la liste des caractères.
        char.lower() not in 'aeiou' : Pour chaque caractère de la liste, cette expression convertit char en minuscules à l'aide de char.lower() pour garantir l'insensibilité à la casse et vérifie si le caractère minuscule ne se trouve pas dans la chaîne "aeiou", qui contient toutes les voyelles minuscules.
        [char for char in characters if char.lower() not in 'aeiou'] : Il s'agit de la compréhension de la liste elle-même. Elle parcourt les caractères de la liste characters et n'inclut dans la nouvelle liste que les caractères qui ne sont pas des voyelles.
    print(characters) : Cette ligne de code imprime la liste originale des caractères sur la console.
    print(voyelles) : Cette ligne de code imprime la liste des voyelles (qui contient les voyelles) sur la console.
    print(consonnes) : Cette ligne de code imprime la liste des consonnes (qui contient les consonnes) sur la console.
        \end{solution}
        

        \question
        Suppression des espaces dans les chaînes de caractères d'une liste

Exemple de sortie

[' hello ', ' world ', ' python ']

['hello', 'world', 'python']
        \par
        \begin{solution}
            \renewcommand{\nomfichier}{q554.py}
            \pythonfile{\chemincode \nomfichier}[][\nomfichier]
            Ce code Python prend une liste de chaînes et crée une nouvelle liste appelée trimmed, dans laquelle les espaces blancs de début et de fin (y compris les espaces, les tabulations et les caractères de retour à la ligne) sont supprimés de chaque chaîne. Voici comment fonctionne ce code :

    strings = [" hello ", " world ", " python "] : Cette ligne initialise une variable nommée strings et lui affecte une liste contenant trois chaînes, chacune d'entre elles comportant des espaces avant et arrière.
    trimmed = [string.strip() for string in strings] : Cette ligne de code initialise une variable nommée trimmed et lui affecte le résultat d'une compréhension de liste.
        for string in strings : Cette partie met en place une boucle qui parcourt chaque chaîne de la liste strings.
        string.strip() : Pour chaque chaîne de la liste, la méthode strip() est appelée pour supprimer les espaces blancs de début et de fin de la chaîne. Le résultat est une chaîne dont les espaces blancs ont été supprimés.
        [string.strip() for string in strings] : Il s'agit de la compréhension de la liste elle-même. Elle parcourt les chaînes de la liste strings, supprime les espaces de chaque chaîne et inclut les chaînes modifiées dans la nouvelle liste.
    print(strings) : Cette ligne de code imprime la liste originale des chaînes sur la console.
    print(trimmed) : Cette ligne de code imprime sur la console la liste élaguée (qui contient les chaînes modifiées avec les espaces blancs de début et de fin supprimés).
        \end{solution}
        

        \question
        Créer une liste de caractères qui ne sont pas des voyelles à partir d'une chaîne de caractères

Exemple de sortie

Bonjour à tous !

['H', 'l', 'l', ',', ' ', 'w', 'r', 'l', 'd', '!']
        \par
        \begin{solution}
            \renewcommand{\nomfichier}{q555.py}
            \pythonfile{\chemincode \nomfichier}[][\nomfichier]
            Ce code Python prend une chaîne et crée une nouvelle liste appelée non\_voyelles, qui contient tous les caractères de la chaîne originale qui ne sont pas des voyelles (les voyelles minuscules et majuscules sont prises en compte). Voici comment fonctionne le code :

    string = "Hello, world !": Cette ligne initialise une variable nommée string et lui assigne une chaîne contenant une phrase.
    non\_voyelles = [char for char in string if char.lower() not in 'aeiou'] : Cette ligne de code initialise une variable nommée non\_voyelles et lui affecte le résultat d'une compréhension de liste.
        for char in string : Cette partie met en place une boucle qui parcourt chaque caractère char de la chaîne.
        char.lower() not in 'aeiou' : Pour chaque caractère de la chaîne, cette expression convertit char en minuscules à l'aide de char.lower() pour garantir l'insensibilité à la casse et vérifie si le caractère minuscule ne se trouve pas dans la chaîne "aeiou", qui contient toutes les voyelles minuscules.
        [char for char in string if char.lower() not in 'aeiou'] : Il s'agit de la compréhension de la liste elle-même. Elle parcourt les caractères de la chaîne et n'inclut que les caractères qui ne sont pas des voyelles (minuscules et majuscules) dans la nouvelle liste.
    print(string) : Cette ligne de code imprime la chaîne originale sur la console.
    print(non\_voyelles) : Cette ligne de code imprime la liste des non\_voyelles (qui contient les caractères qui ne sont pas des voyelles) sur la console.
        \end{solution}
        

	
\end{questions}





\end{document}