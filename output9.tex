
%--------------------%
        \question
        Créer une liste de carrés de nombres de 1 à 10

        Exemple de sortie
        
        [1, 4, 9, 16, 25, 36, 49, 64, 81, 100]
        
        \begin{solution}
                \renewcommand{\nomfichier}{q500.py}
                \pythonfile{\chemincode \nomfichier}[][\nomfichier]
                Le code que vous avez fourni est écrit en Python et utilise une compréhension de liste pour créer une liste appelée squares qui contient les carrés des nombres de 1 à 10. Voici une explication pas à pas du code :

\begin{itemize}
	\item carrés = [x**2 for x in range(1, 11)] : Cette ligne de code initialise une variable nommée carrés et lui affecte le résultat d'une compréhension de liste.
\begin{itemize}
	\item for x in range(1, 11) : Cette partie met en place une boucle qui parcourt les nombres de 1 à 10 (inclus). La fonction range(1, 11) génère une séquence de nombres commençant par 1 et se terminant par 10.
	\item x**2 : pour chaque valeur de x dans la plage, cette expression calcule le carré de x.
	\item \[x**2 for x in range(1, 11)\] : Il s'agit de la compréhension de la liste elle-même. Elle parcourt les nombres de l'intervalle spécifié (1 à 10) et, pour chaque nombre, calcule son carré. Les carrés obtenus sont rassemblés dans une nouvelle liste.
\end{itemize}
	\item print(carrés) : Cette ligne de code imprime simplement la liste des carrés sur la console.
\end{itemize}
                 

        \end{solution}
%--------------------%
\question



\begin{solution}
        \renewcommand{\nomfichier}{q501.py}
        \pythonfile{\chemincode \nomfichier}[][\nomfichier]
\end{solution}
%--------------------%
\question


\begin{solution}
        \renewcommand{\nomfichier}{pascorrige.py}
        \pythonfile{\chemincode \nomfichier}[][\nomfichier]
\end{solution}
%--------------------%
\question


\begin{solution}
\renewcommand{\nomfichier}{pascorrige.py}
\pythonfile{\chemincode \nomfichier}[][\nomfichier]
\end{solution}
%--------------------%
\question


\begin{solution}
        \renewcommand{\nomfichier}{pascorrige.py}
        \pythonfile{\chemincode \nomfichier}[][\nomfichier]
\end{solution}
%--------------------%
\question


\begin{solution}
\renewcommand{\nomfichier}{pascorrige.py}
\pythonfile{\chemincode \nomfichier}[][\nomfichier]
\end{solution}
%--------------------%
\question


\begin{solution}
        \renewcommand{\nomfichier}{pascorrige.py}
        \pythonfile{\chemincode \nomfichier}[][\nomfichier]
\end{solution}
%--------------------%
\question


\begin{solution}
\renewcommand{\nomfichier}{pascorrige.py}
\pythonfile{\chemincode \nomfichier}[][\nomfichier]
\end{solution}
%--------------------%
\question


\begin{solution}
        \renewcommand{\nomfichier}{pascorrige.py}
        \pythonfile{\chemincode \nomfichier}[][\nomfichier]
\end{solution}
%--------------------%
\question


\begin{solution}
\renewcommand{\nomfichier}{pascorrige.py}
\pythonfile{\chemincode \nomfichier}[][\nomfichier]
\end{solution}
%--------------------%
\question


\begin{solution}
        \renewcommand{\nomfichier}{pascorrige.py}
        \pythonfile{\chemincode \nomfichier}[][\nomfichier]
\end{solution}
%--------------------%
\question


\begin{solution}
\renewcommand{\nomfichier}{pascorrige.py}
\pythonfile{\chemincode \nomfichier}[][\nomfichier]
\end{solution}
%--------------------%
\question


\begin{solution}
        \renewcommand{\nomfichier}{pascorrige.py}
        \pythonfile{\chemincode \nomfichier}[][\nomfichier]
\end{solution}
%--------------------%
\question


\begin{solution}
\renewcommand{\nomfichier}{pascorrige.py}
\pythonfile{\chemincode \nomfichier}[][\nomfichier]
\end{solution}
%--------------------%
\question


\begin{solution}
        \renewcommand{\nomfichier}{pascorrige.py}
        \pythonfile{\chemincode \nomfichier}[][\nomfichier]
\end{solution}
%--------------------%
\question


\begin{solution}
\renewcommand{\nomfichier}{pascorrige.py}
\pythonfile{\chemincode \nomfichier}[][\nomfichier]
\end{solution}
%--------------------%
\question


\begin{solution}
        \renewcommand{\nomfichier}{pascorrige.py}
        \pythonfile{\chemincode \nomfichier}[][\nomfichier]
\end{solution}
%--------------------%
\question


\begin{solution}
\renewcommand{\nomfichier}{pascorrige.py}
\pythonfile{\chemincode \nomfichier}[][\nomfichier]
\end{solution}
%--------------------%
\question


\begin{solution}
        \renewcommand{\nomfichier}{pascorrige.py}
        \pythonfile{\chemincode \nomfichier}[][\nomfichier]
\end{solution}
%--------------------%
\question


\begin{solution}
\renewcommand{\nomfichier}{pascorrige.py}
\pythonfile{\chemincode \nomfichier}[][\nomfichier]
\end{solution}
%--------------------%
\question


\begin{solution}
        \renewcommand{\nomfichier}{pascorrige.py}
        \pythonfile{\chemincode \nomfichier}[][\nomfichier]
\end{solution}
%--------------------%
\question


\begin{solution}
\renewcommand{\nomfichier}{pascorrige.py}
\pythonfile{\chemincode \nomfichier}[][\nomfichier]
\end{solution}