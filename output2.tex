
%Question 1
\question
	Écrivez une fonction \textbf{precedent\_suivant()} qui lit un numéro entier et renvoie ses numéros précédents et suivants.
	
	Exemple d'entrée:
	
	precedent\_suivant(179)
	
	Exemple de sortie:
	
	(178, 180)
  \par
  \renewcommand{\nomfichier}{q075.py}
  \begin{solution}
      \pythonfile{\chemincode \nomfichier}[][\nomfichier]
  \end{solution}       
%Question 2
  \question
  N étudiants prennent K pommes et les distribuent entre eux uniformément.La partie restante (indivisible) reste dans le panier.Combien de pommes aura chaque étudiante et combien resteront dans le panier ?
  
  La fonction lit les nombres n et k et renvoie les deux réponses pour les questions ci-dessus.

	Exemple d'entrée:
	
	Apple\_sharing(6, 50)
	
	Exemple de sortie:
	
	(8, 2)
  \par
  \renewcommand{\nomfichier}{q076.py}
  \begin{solution}
      \pythonfile{\chemincode \nomfichier}[][\nomfichier]
  \end{solution}
    
%Question 3
  \question
  Écrivez une fonction appelée \textbf{carre()} qui calcule la valeur du carré d'un nombre.

	Exemple d'entrée:
	
	carre(6)
	
	Exemple de sortie:
	
	36
  \par
  \renewcommand{\nomfichier}{q077.py}
  \begin{solution}
      \pythonfile{\chemincode \nomfichier}[][\nomfichier]
  \end{solution}
        
%Question 4
  \question
  Écrire la fonction \textbf{heures\_minutes()} pour transformer le nombre donné en secondes en heures et minutes.
	
	Exemple 1:
	
	heures\_minutes(3900)\newline
	sortie : (1, 5)
	
	Exemple 2:
	
	heures\_minutes(60)\newline
	sortie : (0, 1)
        \par
        \renewcommand{\nomfichier}{q078.py}
        \begin{solution}
            \pythonfile{\chemincode \nomfichier}[][\nomfichier]
        \end{solution}
        
%Question 5
    \question
    Étant donné deux horodatages du même jour.
    Chaque horodatage est représenté par un nombre :
    \begin{itemize}
    \item d'heures
    \item de minutes
    \item de secondes
    \end{itemize}
    
    L'instant du premier horodatage s'est produit avant l'instant du second. Calculez le nombre de secondes qui se sont écoulées entre les deux.

		Exemple 1:
		
		two\_timestamp(1,1,1,2,2,2)\newline
		Sortie : 3661
		
		Exemple 2:
		
		two\_timestamp(1,2,30,1,3,20)\newline
		Sortie : 50
        \par
        \renewcommand{\nomfichier}{q079.py}
        \begin{solution}
            \pythonfile{\chemincode \nomfichier}[][\nomfichier]
        \end{solution}
        
%Question 6
    \question
    Créez une fonction nommée two\_digits().
    
    Étant donné un entier à deux chiffres, two\_digits() renvoie son chiffre gauche (le chiffre des dizaines) puis son chiffre droit (le chiffre des unités).

		Exemple d'entrée:
		
		two\_digits(79)
		
		Exemple de sortie:
		
		(7, 9)
    \par
    \renewcommand{\nomfichier}{q080.py}
    \begin{solution}
        \pythonfile{\chemincode \nomfichier}[][\nomfichier]
    \end{solution}
        
%Question 7
    \question
    Écrire la fonction nommée swap\_digits().
    
    Étant donné un entier à deux chiffres, swap\_digits() échange ses chiffres et imprimez le résultat.

		Exemple d'entrée:
		
		swap\_digits(79)
		
		Exemple de sortie:
		
		97
    \par
    \renewcommand{\nomfichier}{q081.py}
    \begin{solution}
        \pythonfile{\chemincode \nomfichier}[][\nomfichier]
    \end{solution}
        
%Question 8
    \question
    Écrire la fonction last\_two\_digits().Étant donné un entier supérieur à 9, last\_two\_digits() imprime ses deux derniers chiffres.

		Exemple d'entrée:
		
		last\_two\_digits(1234)
		
		Exemple de sortie:
		
		34
    \par
    \renewcommand{\nomfichier}{q082.py}
    \begin{solution}
        \pythonfile{\chemincode \nomfichier}[][\nomfichier]
    \end{solution}
        
%Question 9
    \question
    Écrire la fonction tens\_digit().
    
    Étant donné un entier, tens\_digit() renvoie son chiffre de dizaines.

		Exemple 1:
		
		tens\_digit(1234)\newline
		Sortie : 3
		
		Exemple 2:
		
		tens\_digit(179)\newline
		Sortie : 7
    \par
    \renewcommand{\nomfichier}{q083.py}
    \begin{solution}
        \pythonfile{\chemincode \nomfichier}[][\nomfichier]
    \end{solution}
        
%Question 10
    \question
    Écrire la fonction digits\_sum().
    
    Étant donné un numéro à trois chiffres, digits\_sum() trouve la somme de ses chiffres.

		Exemple d'entrée:
		
		digits\_sum(123)
		
		Exemple de sortie:
		
		6
    \par
    \renewcommand{\nomfichier}{q084.py}
    \begin{solution}
        \pythonfile{\chemincode \nomfichier}[][\nomfichier]
    \end{solution}
        
%Question 11
    \question
    Écrire la fonction first\_digit(). Étant donné un nombre réel positif, first\_digit() renvoie son premier chiffre (à droite de la virgule).

		Exemple d'entrée:
		
		first\_digit(1.79)
		
		Exemple de sortie:
		
		7
    \par
    \renewcommand{\nomfichier}{q085.py}
    \begin{solution}
        \pythonfile{\chemincode \nomfichier}[][\nomfichier]
    \end{solution}
        
%Question 12
		\question
		Une voiture peut parcourir une distance de N kilomètres par jour. Combien de jours lui faudra-t-il pour parcourir un itinéraire d'une longueur de M kilomètres ?
		Instructions :
		
    Écrire une fonction car\_route() qui prend deux arguments :
    \begin{itemize}
	    \item la distance qu'elle peut parcourir en un jour
	    \item la distance à parcourir
    \end{itemize}
    
    Cette fonction calcule le nombre de jours qu'il faudra pour parcourir cette distance.
		
		Exemple d'entrée:
		
		car\_route(20, 40)
		
		Exemple de sortie:
		
		2
    \par
    \renewcommand{\nomfichier}{q086.py}
    \begin{solution}
        \pythonfile{\chemincode \nomfichier}[][\nomfichier]
    \end{solution}
    
%Question 13
		\question
		Écrivez une fonction century().
		Cette dernière prend une année en paramètre sous la forme d'un entier et renvoi le numéro du siècle.
		
		Exemple d'entrée:
		
		century(2001)
		
		Exemple de sortie:
		
		21
		\par
		\renewcommand{\nomfichier}{q087.py}
		\begin{solution}
		    \pythonfile{\chemincode \nomfichier}[][\nomfichier]
		\end{solution}
        
%Question 14
		\question
		Un petit gâteau coûte d euros et c centimes. Écrivez une fonction qui détermine le nombre d'euros et de centimes qu'une personne devrait payer pour n petits gâteaux. La fonction reçoit trois nombres : d, c, n et doit renvoyer deux nombres : le coût total en euros et en centimes.
		
		Exemple d'entrée:
		
		total\_cost(15, 22, 4)
		
		Sortie :
		
		(60, 88)
		\par
		\renewcommand{\nomfichier}{q088.py}
		\begin{solution}
		    \pythonfile{\chemincode \nomfichier}[][\nomfichier]
		\end{solution}
        
%Question 15
		\question
		Écrire une fonction day\_of\_week(). On lui fourni un entier k compris entre 1 et 365, la fonction day\_of\_week() trouve le numéro du jour de la semaine pour le k-ième jour de l'année, à condition que le 1er janvier de cette année soit un jeudi.
		
		Les jours de la semaine sont numérotés comme:
		
		\begin{enumerate}
		\item[0] Dimanche
		\item[1] Lundi
		\item[2] Mardi ...
		
		\item[6] Samedi
		\end{enumerate}
		
		Exemple d'entrée:
		
		day\_of\_week(1)
		
		Exemple de sortie:
		
		4
		\par
		\renewcommand{\nomfichier}{q089.py}
		\begin{solution}
		    \pythonfile{\chemincode \nomfichier}[][\nomfichier]
		\end{solution}
        
%Question 16
		\question
		Soit l'entier n - le nombre de minutes qui se sont écoulées depuis minuit, combien d'heures et de minutes sont affichées sur l'horloge numérique de 24 heures ? Écrivez une fonction digital\_clock() pour le calculer. La fonction doit afficher deux nombres : le nombre d'heures (entre 0 et 23) et le nombre de minutes (entre 0 et 59).
		
		Exemple d'entrée:
		
		digital\_clock(150)
		
		Exemple de sortie:
		
		(2, 30)
		\par
		\renewcommand{\nomfichier}{q090.py}
		\begin{solution}
		    \pythonfile{\chemincode \nomfichier}[][\nomfichier]
		\end{solution}
        
%Question 17 Supprimer Q2 site 1
\question
Question supprimée, reste la question 2 du site 1
%		\question
%		Créez une fonction nommée factorial (), qui reçoit un nombre en tant que paramètre et renvoie le factoriel de cette valeur.
%		
%		Exemple d'entrée:
%		
%		factorielle(8)
%		
%		Exemple de sortie:
%		
%		40320
%		\par
%		\renewcommand{\nomfichier}{q091.py}
%		\begin{solution}
%		    \pythonfile{\chemincode \nomfichier}[][\nomfichier]
%		\end{solution}
        
%Question 18
		\question
		Créez une fonction nommée racine(), qui reçoit un nombre en tant que paramètre et renvoie la racine carrée.
		
		Si le nombre résultant a des décimales, veuillez ne garder que les 2 premiers.
		
		Exemple d'entrée:
		
		racine(50)
		
		Exemple de sortie:
		
		7.07
		\par
		\renewcommand{\nomfichier}{q092.py}
		\begin{solution}
		    \pythonfile{\chemincode \nomfichier}[][\nomfichier]
		\end{solution}
        
%Question 19
		\question
		Créez une fonction appelée squares\_dictionary ().La fonction reçoit un nombre n et devrait générer un dictionnaire qui contient des paires de la forme (n: n * n) pour chaque nombre dans la plage de 1 à n, inclus.
		
		Imprimez le dictionnaire résultant.
		
		Exemple d'entrée:
		
		squares\_dictionary(8)
		
		Exemple de sortie:
		
		\{1: 1, 2: 4, 3: 9, 4: 16, 5: 25, 6: 36, 7: 49, 8: 64\}
		\par
		\renewcommand{\nomfichier}{q093.py}
		\begin{solution}
		    \pythonfile{\chemincode \nomfichier}[][\nomfichier]
		\end{solution}
        
%Question 20
		\question
		Créez une fonction appelée list\_and\_tuple(), qui prend en entrée n nombres et renvoie une liste et un tuple de ces nombres sous forme de chaîne.
		
		Imprimez la liste et le tuple sur deux lignes.
		
		Exemple d'entrée:
		
		list\_and\_tuple(34,67,55,33,12,98)
		
		Exemple de sortie:
		
		['34', '67', '55', '33', '12', '98']
		('34', '67', '55', '33', '12', '98')
		\par
		\renewcommand{\nomfichier}{q094.py}
		\begin{solution}
		    \pythonfile{\chemincode \nomfichier}[][\nomfichier]
		\end{solution}
        
%Question 21
		\question
Question POO
%		Définissez une classe appelée InputOutString qui a au moins deux méthodes:
%		
%		get\_string pour obtenir une chaîne à partir de l'entrée de la console.
%		print\_string pour imprimer la chaîne en majuscule.
%		
%		Testez les méthodes de votre classe.
%		\par
%		\renewcommand{\nomfichier}{q095.py}
%		\begin{solution}
%		  \pythonfile{\chemincode \nomfichier}[][\nomfichier]
%		\end{solution}

%Question 22
		\question
		Écrivez une fonction print\_formula(), avec un paramètre qui calcule et imprime la valeur en fonction de la formule donnée:
		
		Q = racine carrée de (2 * c * d) / h
		
		Voici les valeurs fixes de C et H:
		
		C est de 50.\newline
		H est 30.\newline
		D serait le paramètre de la fonction.
		
		Exemple d'entrée:
		
		print\_formula(150)
		
		Sortie:
		
		22
		\par
		\renewcommand{\nomfichier}{q096.py}
		\begin{solution}
		    \pythonfile{\chemincode \nomfichier}[][\nomfichier]
		\end{solution}
        
%Question 23
		\question
		Écrivez une fonction two\_dimensional\_list(), qui prend 2 chiffres (x, y) en entrée et génère une liste à 2 dimensions.
		
		La valeur de l'élément dans la ligne i et la colonne j doit être i * j.
		
		Exemple d'entrée:
		
		two\_dimensional\_list(3,5)
		
		Exemple de sortie:
		
		[[0, 0, 0, 0, 0], [0, 1, 2, 3, 4], [0, 2, 4, 6, 8]]
		\par
		\renewcommand{\nomfichier}{q097.py}
		\begin{solution}
		    \pythonfile{\chemincode \nomfichier}[][\nomfichier]
		\end{solution}
        
%Question 24
		\question
		Écrire une fonction sequence\_of\_words, qui accepte en entrée une séquence de mots séparés par des virgules (une chaîne).
		
    Imprimer les mots dans une séquence séparée par des virgules après les avoir triés par ordre alphabétique.
	
		Exemple d'entrée:
		
		sequence\_of\_words("sans, bonjour, sac, monde")
		
		Exemple de sortie:
		
		Sac, bonjour, sans, monde
		\par
		\renewcommand{\nomfichier}{q098.py}
		\begin{solution}
		    \pythonfile{\chemincode \nomfichier}[][\nomfichier]
		\end{solution}
        
%Question 25
		\question
		Écrire une fonction appelée remove\_duplicate\_words() qui accepte en entrée une séquence de mots séparés par des espaces et qui renvoie les mots après avoir supprimé tous les mots en double et les avoir triés par ordre alphanumérique.
		
		Exemple d'entrée:
		
		remove\_duplicate\_words("Hello World and Practice rend à nouveau parfait et bonjour le monde")
		
		Exemple de sortie:
		
		Encore une fois et bonjour fait un monde de pratique parfait
		\par
		\renewcommand{\nomfichier}{q099.py}
		\begin{solution}
		    \pythonfile{\chemincode \nomfichier}[][\nomfichier]
		\end{solution}
        
%Question 26
		\question
		Écrire une fonction divisible\_binary() qui prend en entrée une séquence de nombres binaires à 4 chiffres séparés par des virgules et vérifie s'ils sont divisibles par 5. Imprimer les nombres qui sont divisibles par 5 dans une séquence séparée par des virgules.
		
		Exemple d'entrée:
		
		divisible\_binary("1000,1100,1010,1111")
		
		Exemple de sortie:
		
		1010,1111
		\par
		\renewcommand{\nomfichier}{q100.py}
		\begin{solution}
		    \pythonfile{\chemincode \nomfichier}[][\nomfichier]
		\end{solution}
        
%Question 27
		\question
		Définir une fonction nommée all\_digits\_even() pour identifier et imprimer tous les nombres entre 1000 et 3000 (inclus) où chaque chiffre du nombre est un nombre pair. Affichez les nombres résultants dans une séquence séparée par des virgules sur une seule ligne.
		\par
		\renewcommand{\nomfichier}{q101.py}
		\begin{solution}
		    \pythonfile{\chemincode \nomfichier}[][\nomfichier]
		\end{solution}
        
%Question 28
			\question
			Écrire une fonction nommée letters\_and\_digits() qui prend une phrase en entrée et calcule le nombre de lettres et de chiffres qu'elle contient.
			
			Exemple d'entrée:
			
			letters\_and\_digits("Hello World! 123")
			
			Exemple de sortie:
			
			Lettres 10
			Chiffres 3
			\par
			\renewcommand{\nomfichier}{q102.py}
			\begin{solution}
			    \pythonfile{\chemincode \nomfichier}[][\nomfichier]
			\end{solution}
        
%Question 29
		\question
		Écrivez un programme number\_of\_uppercase() qui accepte une phrase et calcule le nombre de lettres majuscules et minuscules.
		
		Exemple d'entrée:
		
		number\_of\_uppercase("Hello World!")
		
		Exemple de sortie:
		
		Majuscule 1
		Minuscule 9
		\par
		\renewcommand{\nomfichier}{q103.py}
		\begin{solution}
		  \pythonfile{\chemincode \nomfichier}[][\nomfichier]
		\end{solution}
        
%Question 30
		\question
		Écrivez un programme computed\_value() pour calculer la somme d'un + aa + aaa + aaaa, où «a» est un chiffre donné.
		
		Exemple d'entrée:
		
		computed\_value(9)
		
		Exemple de sortie:
		
		11106
		\par
		\renewcommand{\nomfichier}{q104.py}
		\begin{solution}
		    \pythonfile{\chemincode \nomfichier}[][\nomfichier]
		\end{solution}
        
%Question 31
		\question
		Écrivez une fonction nommée square\_odd\_numbers() qui accepte en entrée une chaîne de nombres séparés par des virgules, ne met au carré que les nombres impairs et renvoie les résultats sous la forme d'une liste.
		
		Exemple d'entrée:
		
		square\_odd\_numbers("1,2,3,4,5,6,7,8,9")
		
		Exemple de sortie:
		
		[1, 9, 25, 49, 81]
		\par
		\renewcommand{\nomfichier}{q105.py}
		\begin{solution}
		  \pythonfile{\chemincode \nomfichier}[][\nomfichier]
		\end{solution}
        
%Question 32
		\question
		Écrire une fonction nommée net\_amount() qui calcule le montant net d'un compte bancaire sur la base d'un journal de transactions provenant de l'entrée. Le format du journal des transactions est le suivant :
		
		D 100\newline
		W 200
		
		D signifie dépôt tandis que w signifie le retrait.\newline
		Exemple d'entrée:
		
		net\_amount("D 300 D 300 W 200 D 100")
		
		Exemple de sortie:
		
		500
		\par
		\renewcommand{\nomfichier}{q106.py}
		\begin{solution}
		    \pythonfile{\chemincode \nomfichier}[][\nomfichier]
		\end{solution}
        
%Question 33
		\question
		Un site Web oblige les utilisateurs à saisir un nom d'utilisateur et un mot de passe pour s'inscrire.Écrivez une fonction nommée valid\_password() pour vérifier la validité de l'entrée de mot de passe par les utilisateurs.Voici les critères de vérification du mot de passe:
		
	\begin{itemize}
	\item 	Au moins 1 lettre entre [A-Z].
	\item 		Au moins 1 nombre entre [0-9].
	\item 		Au moins 1 lettre entre [A-Z].
	\item 		Au moins 1 caractère de [\$ \# @].
	\item 		Longueur minimale du mot de passe: 6.
	\item 		Longueur maximale du mot de passe: 12.
	\end{itemize}
		
		Votre programme doit accepter un mot de passe et le vérifier en fonction des critères précédents.Si le mot de passe est validé avec succès, la fonction renvoie la chaîne suivante "Mot de passe valide".Sinon, il renvoie "mot de passe non valide. Veuillez réessayer".
		Exemple d'entrée:
		
		valid\_password("ABD1234 @ 1")
		
		Exemple de sortie:
		
		"Mot de passe valide"
		\par
		\renewcommand{\nomfichier}{q107.py}
		\begin{solution}
		    \pythonfile{\chemincode \nomfichier}[][\nomfichier]
		\end{solution}
        
%Question 34
		\question
		Écrivez une fonction sort\_tuples\_ascending() pour trier les tuples (nom, âge, score) par ordre croissant, où nom, âge et score sont tous des chaînes de caractères. Les critères de tri sont :
		
		\begin{itemize}
		\item Trier basé sur le nom.
		\item Puis trier en fonction de l'âge.
		\item Puis trier par score.
		\end{itemize}
		
		La priorité est le nom> Age> Score.\newline
		Exemple d'entrée:
		
		sort\_tuples\_ascending([«Tom, 19,80», «John, 20,90», «Jony, 17,91», «Jony, 17,93», «Jason, 21,85»])
		
		Exemple de sortie:
		
		[('Jason', '21', '85'), ('John', '20', '90'), ('Jony', '17', '91'), ('Jony', '17',' 93 '), (' Tom ',' 19 ',' 80 ')]]
		\par
		\renewcommand{\nomfichier}{q108.py}
		\begin{solution}
		    \pythonfile{\chemincode \nomfichier}[][\nomfichier]
		\end{solution}
        
%Question 35
		\question
		Question POO
%		Définir une classe avec une fonction génératrice qui peut itérer les nombres qui sont divisibles par 7 entre un intervalle donné 0 et n.
%		\par
%		\renewcommand{\nomfichier}{q109.py}
%		\begin{solution}
%		    \pythonfile{\chemincode \nomfichier}[][\nomfichier]
%		\end{solution}
        
%Question 36
		\question
		Un robot se déplace dans un plan à partir du point d'origine (0,0). Le robot peut se déplacer vers le HAUT, le BAS, la GAUCHE et la DROITE avec des étapes données. La trace du mouvement du robot est présentée sous la forme d'une liste comme la suivante :
		
		["UP 5", "DOWN 3", "LEFT 3", "RIGHT 2"]
		
		Les nombres qui suivent la direction sont des pas. Veuillez écrire un programme nommé compute\_robot\_distance() pour calculer la distance finale après une séquence de mouvements à partir du point d'origine. Si la distance est un flotteur, il suffit d'imprimer l'entier le plus proche.
		Exemple d'entrée :
		
		compute\_robot\_distance(["UP 5", "DOWN 3", "LEFT 3", "RIGHT 2"])
		
		Exemple de sortie:
		
		2
		\par
		\renewcommand{\nomfichier}{q110.py}
		\begin{solution}
		    \pythonfile{\chemincode \nomfichier}[][\nomfichier]
		\end{solution}
        
%Question 37
		\question
		Écrivez une fonction appelée compute\_word\_frequency() pour calculer la fréquence des mots à partir d'une chaîne de caractères.
		
		  \begin{itemize}
		  \item Placez chaque mot séparé par un espace dans un dictionnaire et comptez sa fréquence.
		  \item Classez le dictionnaire par ordre alphanumérique et imprimez dans la console chaque clé sur une nouvelle ligne.
		  \end{itemize}
		
		Exemple d'entrée:
		
		compute\_word\_frequency("New to Python or choosing between Python 2 and Python 3? Read Python 2 or Python 3.")
		
		Exemple de sortie:
		
		2: 2\newline
		3.: 1\newline
		3?: 1\newline
		New: 1\newline
		Python: 5\newline
		Read: 1\newline
		and: 1\newline
		between: 1\newline
		choosing: 1\newline
		or: 2\newline
		to: 1
		\par
		\renewcommand{\nomfichier}{q111.py}
		\begin{solution}
		    \pythonfile{\chemincode \nomfichier}[][\nomfichier]
		\end{solution}
        
%Question 38
		\question
		Question POO
%		En Python, une classe est une structure qui permet d'organiser et d'encapsuler des données et des fonctionnalités connexes. Les classes sont une caractéristique fondamentale de la programmation orientée objet (POO), un paradigme de programmation qui utilise des objets pour modéliser et organiser le code.
%		
%		En termes simples, une classe est comme un plan ou un modèle pour créer des objets. Un objet est une instance spécifique d'une classe à laquelle sont associés des attributs (données) et des méthodes (fonctions). Les attributs représentent les caractéristiques de l'objet et les méthodes représentent les actions que l'objet peut effectuer.
%		Exemple :
%		\renewcommand{\nomfichier}{q112depart.py}
%		\pythonfile{\chemincode \nomfichier}[][\nomfichier]
%		
%		Dans ce code :
%		
%    \begin{itemize}
%    \item La classe Student possède une méthode \_\_init\_\_ pour initialiser les attributs nom, âge et classe de l'étudiant.
%    \item introduce est une méthode qui imprime un message de présentation de l'étudiant.
%    \item study est une méthode qui simule l'acte d'étudier et met à jour la note de l'étudiant.
%    \end{itemize}
%		
%		Instructions :
%		
%    Pour réaliser cet exercice, copiez le code fourni dans l'exemple et collez-le dans votre fichier. Exécutez le code et testez sa fonctionnalité. Essayez de modifier différents aspects du code pour observer son comportement. Cette approche pratique vous aidera à comprendre la structure et le comportement de la classe Étudiant. Une fois que vous serez familiarisé avec le code et ses effets, n'hésitez pas à passer à l'exercice suivant.
%		\par
%		\renewcommand{\nomfichier}{q112.py}
%		\begin{solution}
%		    \pythonfile{\chemincode \nomfichier}[][\nomfichier]
%		\end{solution}
        
%Question 39
        \question
        Question POO
%		\textbf{Méthodes \_\_init\_\_ et \_\_str\_\_}
%		
%		En général, lorsque vous travaillez avec des classes, vous rencontrez des méthodes de la forme \_\_<méthode>\_\_ ; ces méthodes sont appelées "méthodes magiques". Il en existe un grand nombre, chacune ayant un objectif spécifique. Cette fois-ci, nous nous concentrerons sur l'apprentissage de deux des méthodes les plus fondamentales.
%		
%		La méthode magique \_\_init\_\_ est essentielle pour l'initialisation des objets au sein d'une classe. Elle est automatiquement exécutée lorsqu'une nouvelle instance de la classe est créée, ce qui permet d'attribuer des valeurs initiales aux attributs de l'objet.
%		
%		La méthode \_\_str\_\_ est utilisée pour fournir une représentation sous forme de chaîne de caractères lisible de l'instance, ce qui permet de personnaliser la sortie lors de l'impression de l'objet. Cette méthode est particulièrement utile pour améliorer la lisibilité du code et faciliter le débogage, car elle définit une version conviviale des informations contenues dans l'objet.
%		
%		Exemple :
%		
%		\renewcommand{\nomfichier}{q113depart.py}
%		\pythonfile{\chemincode \nomfichier}[][\nomfichier]
%
%		Instructions:
%		
%    \begin{itemize}
%    \item Créez une classe appelée Book qui possède les méthodes \_\_init\_\_ et \_\_str\_\_.
%    \item La méthode \_\_init\_\_ doit initialiser les attributs title, author et year.
%    \item La méthode \_\_str\_\_ doit renvoyer une chaîne de caractères représentant les informations d'une instance du livre suivant de cette manière :\newline
%    book1 = ("The Great Gatsby", "F. Scott Fitzgerald", 1925)\newline
%    print(book1)\newline
%    \newline
%    \# Sortie :\newline
%    \#\newline
%    \# Title : Le Grand Gatsby\newline
%    \# Author : F. Scott Fitzgerald\newline
%    \# Year: 1925\newline
%    \end{itemize}
%    \par
%    \renewcommand{\nomfichier}{q113.py}
%    \begin{solution}
%        \pythonfile{\chemincode \nomfichier}[][\nomfichier]
%    \end{solution}
    
%Question 40
    \question
    Question POO
%		\textbf{Héritage et polymorphisme}
%		
%		Maintenant que nous avons compris ce qu'est une classe et certaines de ses caractéristiques, abordons deux nouveaux concepts liés aux classes : l'héritage et le polymorphisme. Prenons l'exemple suivant :
%		
%		\renewcommand{\nomfichier}{q114depart.py}
%		\pythonfile{\chemincode \nomfichier}[][\nomfichier]
%		
%
%		En supposant que la classe Student de l'exercice précédent soit codée juste au-dessus de la classe HighSchoolStudent, pour hériter de ses méthodes et attributs, il suffit d'inclure le nom de la classe dont nous voulons hériter (la classe mère) entre les parenthèses de la classe enfant (HighSchoolStudent). Comme vous pouvez le constater, nous pouvons maintenant utiliser la méthode introduce de la classe Student sans avoir à la coder à nouveau, ce qui rend notre code plus efficace. Il en va de même pour les attributs ; nous n'avons pas besoin de les redéfinir.
%		
%		En outre, nous avons la possibilité d'ajouter de nouvelles méthodes exclusivement pour cette classe ou même de remplacer une méthode héritée si nécessaire, comme le montre la méthode study, qui est légèrement modifiée à partir de la méthode Student ; c'est ce qu'on appelle le polymorphisme.
%		
%		\textbf{Instructions :}
%
%		\begin{itemize}
%		\item Créez une classe appelée CollegeStudent qui hérite de la classe Student déjà définie.
%		\item Ajoutez un nouvel attribut appelé major pour représenter la spécialité étudiée.
%		\item Modifiez la méthode introduce héritée pour qu'elle renvoie cette chaîne de caractères :\newline
%		"Bonjour ! Je m'appelle <nom> et je suis étudiant en <major>."
%		\item Ajoutez une nouvelle méthode appelée attend\_lecture qui renvoie la chaîne suivante :\newline
%		"<nom> assiste à une conférence pour les étudiants de <major>".
%		\item Créez une instance de votre nouvelle classe et appelez chacune de ses méthodes. 
%		\item Exécutez votre code pour vous assurer qu'il fonctionne.
%		\end{itemize}
%    \par
%    \renewcommand{\nomfichier}{q114.py}
%    \begin{solution}
%        \pythonfile{\chemincode \nomfichier}[][\nomfichier]
%    \end{solution}
    
%Question 41
		\question
		Question POO
%		\textbf{Méthodes statiques}
%		
%		Une méthode statique en Python est une méthode liée à une classe plutôt qu'à une instance de la classe. Contrairement aux méthodes ordinaires, les méthodes statiques n'ont pas accès à l'instance ou à la classe elle-même.
%		
%		Les méthodes statiques sont souvent utilisées lorsqu'une méthode particulière ne dépend pas de l'état de l'instance ou de la classe. Elles ressemblent davantage à des fonctions utilitaires associées à une classe.
%		
%		\renewcommand{\nomfichier}{q115depart.py}
%		\pythonfile{\chemincode \nomfichier}[][\nomfichier]
%		
%		Dans cet exemple:
%		
%    La méthode statique is\_adult vérifie si une personne est un adulte en fonction de son âge. Elle n'a pas accès directement aux variables d'instance ou de classe.
%
%		\textbf{Instructions :}
%
%    \begin{itemize}
%    \item Créez une classe appelée MathOperations.
%    \item Créez une méthode statique nommée add\_numbers qui prend deux nombres en paramètre et renvoie leur somme.
%    \item Créez une instance de la classe MathOperations.
%    \item Utilisez la méthode statique add\_numbers pour additionner deux nombres, par exemple 10 et 15.
%    \item Imprimez le résultat.
%    \end{itemize}
%		
%		Exemple d'entrée:
%		
%		math\_operations\_instance = MathOperations()
%		sum\_of\_numbers = MathOperations.add\_numbers(10, 15)
%		
%		Sortie:
%		
%		25
%		\par
%		\renewcommand{\nomfichier}{q115.py}
%		\begin{solution}
%		    \pythonfile{\chemincode \nomfichier}[][\nomfichier]
%		\end{solution}
        
%Question 42
        \question
        Question POO
%		\textbf{Méthodes de classe}
%		
%		Une méthode de classe est une méthode liée à la classe et non à l'instance de la classe. Elle prend comme premier paramètre la classe elle-même, souvent nommée "cls". Les méthodes de classe sont définies à l'aide du décorateur @classmethod.
%		
%		La principale caractéristique d'une méthode de classe est qu'elle peut accéder et modifier les attributs au niveau de la classe, mais qu'elle ne peut pas accéder ou modifier les attributs spécifiques à l'instance puisqu'elle n'a pas accès à une instance de la classe. Les méthodes de classe sont souvent utilisées pour des tâches qui impliquent la classe elle-même plutôt que des instances individuelles.
%		
%		\renewcommand{\nomfichier}{q116depart.py}
%		\pythonfile{\chemincode \nomfichier}[][\nomfichier]
%		
%		Dans cet exemple:
%		
%    La méthode de classe get\_total\_people renvoie le nombre total de personnes créées (instances de la classe Person).
%
%		\textbf{Instructions :}
%
%		\begin{itemize}
%		\item Créez une classe appelée MathOperations.
%		\item Dans cette classe, définissez les éléments suivants :
%			\begin{itemize}
%			\item Une variable de classe nommée pi avec une valeur de 3,14159.
%			\item Une méthode de classe nommée calculate\_circle\_area qui prend un rayon comme paramètre et renvoie l'aire d'un cercle à l'aide de la formule : $aire = \pi × rayon^2$.
%			\end{itemize}
%		\item Utilisez la méthode de classe calculate\_circle\_area pour calculer l'aire d'un cercle de rayon 5.
%		\item Imprimez le résultat. (Il n'est pas nécessaire de créer une instance)
%		\end{itemize}
%		
%		Exemple d'entrée:
%		
%		circle\_area = MathOperations.calculate\_circle\_area(5)
%		
%		Sortie:
%		
%		78.53975
%		\par
%		\renewcommand{\nomfichier}{q116.py}
%		\begin{solution}
%		    \pythonfile{\chemincode \nomfichier}[][\nomfichier]
%		\end{solution}
        
