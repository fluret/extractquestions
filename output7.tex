
%--------------------%
        \question
Nous avons mis en place un système d'amende pour les chasseurs de notre commune.
Chaque chasseur se voit pénaliser d'un certain nombre de point par faute.
Le barème des pénalités est le suivant :
\begin{itemize}
\item s'il tue une poule : 1 point.
\item s'il tue un chien : 3 points.
\item s'il tue une vache : 5 points.
\item s'il tue un ami : 10 points.
\end{itemize}
Un point à une valeur de 2€.

Écrire une fonction amende qui reçoit le nombre de victimes du chasseur et qui renvoie la somme due.

Utilisez cette fonction dans un programme principal qui demande le nombre de victimes et qui affiche la somme que le chasseur doit débourser.

        \begin{solution}
                \renewcommand{\nomfichier}{q133.py}
                \pythonfile{\chemincode \nomfichier}[][\nomfichier]
        \end{solution}
%--------------------%
\question
Soit des comptes bancaires d'individus définis par la liste :\par
\renewcommand{\nomfichier}{q134-comptes.py}
\pythonfile{\chemincode \nomfichier}[][\nomfichier]
On considère que les individus qui portent le même 'nom' sont de la même famille.
En cas d'absence de revenu attribué à un individu, nous considérerons que son épargne est nulle (cas de 'Bernard Gueux').

Écrire une fonction qui retourne le nom de la famille la plus pauvre et de la plus riche avec le montant de leur épargne respective.
Ici, ('Gueux', 1253) et ('Durois', 310000).

\begin{solution}
        \renewcommand{\nomfichier}{q134.py}
        \pythonfile{\chemincode \nomfichier}[][\nomfichier][breakable]
\end{solution}
%%--------------------%
%\question
%
%
%\begin{solution}
%        \renewcommand{\nomfichier}{pascorrige.py}
%        \pythonfile{\chemincode \nomfichier}[][\nomfichier]
%\end{solution}
%%--------------------%
%\question
%
%
%\begin{solution}
%\renewcommand{\nomfichier}{pascorrige.py}
%\pythonfile{\chemincode \nomfichier}[][\nomfichier]
%\end{solution}
%%--------------------%
%\question
%
%
%\begin{solution}
%        \renewcommand{\nomfichier}{pascorrige.py}
%        \pythonfile{\chemincode \nomfichier}[][\nomfichier]
%\end{solution}
%%--------------------%
%\question
%
%
%\begin{solution}
%\renewcommand{\nomfichier}{pascorrige.py}
%\pythonfile{\chemincode \nomfichier}[][\nomfichier]
%\end{solution}
%%--------------------%
%\question
%
%
%\begin{solution}
%        \renewcommand{\nomfichier}{pascorrige.py}
%        \pythonfile{\chemincode \nomfichier}[][\nomfichier]
%\end{solution}
%%--------------------%
%\question
%
%
%\begin{solution}
%\renewcommand{\nomfichier}{pascorrige.py}
%\pythonfile{\chemincode \nomfichier}[][\nomfichier]
%\end{solution}
%%--------------------%
%\question
%
%
%\begin{solution}
%        \renewcommand{\nomfichier}{pascorrige.py}
%        \pythonfile{\chemincode \nomfichier}[][\nomfichier]
%\end{solution}
%%--------------------%
%\question
%
%
%\begin{solution}
%\renewcommand{\nomfichier}{pascorrige.py}
%\pythonfile{\chemincode \nomfichier}[][\nomfichier]
%\end{solution}
%%--------------------%
%\question
%
%
%\begin{solution}
%        \renewcommand{\nomfichier}{pascorrige.py}
%        \pythonfile{\chemincode \nomfichier}[][\nomfichier]
%\end{solution}
%%--------------------%
%\question
%
%
%\begin{solution}
%\renewcommand{\nomfichier}{pascorrige.py}
%\pythonfile{\chemincode \nomfichier}[][\nomfichier]
%\end{solution}
%%--------------------%
%\question
%
%
%\begin{solution}
%        \renewcommand{\nomfichier}{pascorrige.py}
%        \pythonfile{\chemincode \nomfichier}[][\nomfichier]
%\end{solution}
%%--------------------%
%\question
%
%
%\begin{solution}
%\renewcommand{\nomfichier}{pascorrige.py}
%\pythonfile{\chemincode \nomfichier}[][\nomfichier]
%\end{solution}
%%--------------------%
%\question
%
%
%\begin{solution}
%        \renewcommand{\nomfichier}{pascorrige.py}
%        \pythonfile{\chemincode \nomfichier}[][\nomfichier]
%\end{solution}
%%--------------------%
%\question
%
%
%\begin{solution}
%\renewcommand{\nomfichier}{pascorrige.py}
%\pythonfile{\chemincode \nomfichier}[][\nomfichier]
%\end{solution}
%%--------------------%
%\question
%
%
%\begin{solution}
%        \renewcommand{\nomfichier}{pascorrige.py}
%        \pythonfile{\chemincode \nomfichier}[][\nomfichier]
%\end{solution}
%%--------------------%
%\question
%
%
%\begin{solution}
%\renewcommand{\nomfichier}{pascorrige.py}
%\pythonfile{\chemincode \nomfichier}[][\nomfichier]
%\end{solution}
%%--------------------%
%\question
%
%
%\begin{solution}
%        \renewcommand{\nomfichier}{pascorrige.py}
%        \pythonfile{\chemincode \nomfichier}[][\nomfichier]
%\end{solution}
%%--------------------%
%\question
%
%
%\begin{solution}
%\renewcommand{\nomfichier}{pascorrige.py}
%\pythonfile{\chemincode \nomfichier}[][\nomfichier]
%\end{solution}
%%--------------------%
%\question
%
%
%\begin{solution}
%        \renewcommand{\nomfichier}{pascorrige.py}
%        \pythonfile{\chemincode \nomfichier}[][\nomfichier]
%\end{solution}
%%--------------------%
%\question
%
%
%\begin{solution}
%\renewcommand{\nomfichier}{pascorrige.py}
%\pythonfile{\chemincode \nomfichier}[][\nomfichier]
%\end{solution}