
        \question
Projet d'inventaire de produits - Créer une application qui gère un inventaire de produits. Créez une classe de produits avec un prix, un identifiant et une quantité disponible. Créez ensuite une classe d'inventaire qui garde la trace des différents produits et peut résumer la valeur de l'inventaire.
        \par
        
        \begin{solution}
        voir les fichiers q300 et q300-01
%        		\renewcommand{\nomfichier}{q300.py}
%            \pythonfile{\chemincode \nomfichier}[][\nomfichier]
%            \renewcommand{\nomfichier}{q300-01.py}
%            \pythonfile{\chemincode \nomfichier}[][\nomfichier]
            
        \end{solution}

\question
Système de réservation de billets d'avion ou de chambres d'hôtel - Créer un système de réservation de billets d'avion ou de chambres d'hôtel. Il applique différents tarifs pour des sections particulières de l'avion ou de l'hôtel. Par exemple, la première classe coûtera plus cher que la première classe. Les chambres d'hôtel ont des suites penthouse qui coûtent plus cher. Gardez une trace de la disponibilité des chambres et de leur programmation.

\question
Gestionnaire de compte bancaire - Créez une classe appelée Compte qui sera une classe abstraite pour trois autres classes appelées CompteChèque, CompteÉpargne et CompteAffaires. Gérez les crédits et les débits de ces comptes à l'aide d'un programme de type distributeur automatique de billets.

       \begin{solution}
       Voir le fichier q301
%        		\renewcommand{\nomfichier}{q301.py}
%            \pythonfile{\chemincode \nomfichier}[][\nomfichier]
        \end{solution}
        
\question
Classes de surface et de périmètre des formes - Créez une classe abstraite appelée Forme et héritez-en d'autres formes comme le diamant, le rectangle, le cercle, le triangle, etc. Ensuite, chaque classe doit surcharger les fonctionnalités de surface et de périmètre pour gérer chaque type de forme.

       \begin{solution}
       		Voir le fichier q302
%        		\renewcommand{\nomfichier}{q302.py}
%            \pythonfile{\chemincode \nomfichier}[][\nomfichier]
        \end{solution}
       