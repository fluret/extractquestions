\renewcommand{\chemincode}{../../code/}
%--------------------%
        \question
        Trouvez tous les nombres compris entre 1 et 1000 qui sont divisibles par 7.

        \begin{solution}
                \renewcommand{\nomfichier}{q135.py}
                \pythonfile{\chemincode \nomfichier}[][\nomfichier]
        \end{solution}
%--------------------%
\question
Trouvez tous les nombres de 1-1000 qui contiennent un 3.

\begin{solution}
        \renewcommand{\nomfichier}{q136.py}
        \pythonfile{\chemincode \nomfichier}[][\nomfichier]
\end{solution}
%--------------------%
\question
Compter le nombre d'espaces dans une chaine.

\begin{solution}
        \renewcommand{\nomfichier}{q137.py}
        \pythonfile{\chemincode \nomfichier}[][\nomfichier]
\end{solution}
%--------------------%
\question
Créer une liste de toutes les consonnes de la chaîne "Les Yaks jaunes aiment crier et bailler et hier ils ont jodlé en mangeant des ignames yuky".

\begin{solution}
\renewcommand{\nomfichier}{q138.py}
\pythonfile{\chemincode \nomfichier}[][\nomfichier][breakable]
\end{solution}
%--------------------%
\question
Obtenir l'indice et la valeur sous forme de tuple pour les éléments de la liste ["hi", 4, 8.99, 'apple', ('t,b','n')].  Le résultat ressemblerait à [(index, valeur), (index, valeur)].

\begin{solution}
        \renewcommand{\nomfichier}{q139.py}
        \pythonfile{\chemincode \nomfichier}[][\nomfichier]
\end{solution}
%--------------------%
\question
Trouver les nombres communs à deux listes (sans utiliser de tuple ou d'ensemble) list\_a = [1, 2, 3, 4], list\_b = [2, 3, 4, 5]

\begin{solution}
\renewcommand{\nomfichier}{q140.py}
\pythonfile{\chemincode \nomfichier}[][\nomfichier]
\end{solution}
%--------------------%
\question
Dans une phrase comme "En 1984, il y a eu 13 cas de manifestations ayant rassemblé plus de 1 000 personnes", il ne faut retenir que les chiffres.  Le résultat est une liste de nombres comme [3,4,5].

\begin{solution}
        \renewcommand{\nomfichier}{q141.py}
        \pythonfile{\chemincode \nomfichier}[][\nomfichier]
\end{solution}
%--------------------%
\question
Étant donné numbers = range(20), produisez une liste contenant le mot "even" si un des nombres est pair, et le mot "odd" si le nombre est impair.  Le résultat ressemblerait à ['odd', 'odd', 'even'].

\begin{solution}
\renewcommand{\nomfichier}{q142.py}
\pythonfile{\chemincode \nomfichier}[][\nomfichier]
\end{solution}
%--------------------%
\question
Produisez une liste de tuples composée uniquement des nombres correspondants dans ces listes list\_a = [1, 2, 3, 4, 5, 6, 7, 8, 9], list\_b = [2, 7, 1, 12].  Le résultat ressemblerait à (4,4), (12,12)

\begin{solution}
        \renewcommand{\nomfichier}{q143.py}
        \pythonfile{\chemincode \nomfichier}[][\nomfichier]
\end{solution}
%--------------------%
\question
Trouver tous les mots d'une chaîne de moins de 4 lettres

\begin{solution}
\renewcommand{\nomfichier}{q144.py}
\pythonfile{\chemincode \nomfichier}[][\nomfichier]
\end{solution}
%--------------------%
\question
Utilisez la compréhension d'une liste imbriquée pour trouver tous les nombres de 1 à 100 qui sont divisibles par n'importe quel chiffre à part 1 (2-9).

\begin{solution}
        \renewcommand{\nomfichier}{q145.py}
        \pythonfile{\chemincode \nomfichier}[][\nomfichier]
\end{solution}
