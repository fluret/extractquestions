
        \question
        Créer une classe Véhicule avec les attributs d'instance vitesse\_max et kilométrage.
        \par
        \renewcommand{\nomfichier}{q238.py}
        \begin{solution}
            \pythonfile{\chemincode \nomfichier}[][\nomfichier]
        \end{solution}
        

        \question
        Créer une classe Véhicule sans variables ni méthodes
        \par
        \renewcommand{\nomfichier}{q239.py}
        \begin{solution}
            \pythonfile{\chemincode \nomfichier}[][\nomfichier]
        \end{solution}
        

        \question
        Créer une classe enfant Bus qui héritera de toutes les variables et méthodes de la classe Véhicule.\newline
        Vous partirez du code suivant :
        \renewcommand{\nomfichier}{q240depart.py}
        \pythonfile{\chemincode \nomfichier}[][\nomfichier]
				Créer un objet Bus qui héritera de toutes les variables et méthodes de la classe parente "Véhicule" et l'afficher.\newline
				Sortie attendue :\newline
				Nom du véhicule : School Volvo Vitesse : 180 Kilométrage : 12
        \par
        \renewcommand{\nomfichier}{q240.py}
        \begin{solution}
            \pythonfile{\chemincode \nomfichier}[][\nomfichier]
        \end{solution}
        

        \question
        Héritage des classes

				Créez une classe Bus qui hérite de la classe Véhicule. Donnez à l'argument capacité de \textbf{Bus.seating\_capacity()} une valeur par défaut de 50.\newline
				
				Utilisez le code suivant pour votre classe mère Vehicle.
        \renewcommand{\nomfichier}{q241depart.py}
        \pythonfile{\chemincode \nomfichier}[][\nomfichier]				
				Sortie attendue :\newline
				La capacité d'accueil d'un bus est de 50 passagers.
        \par
        \textbf{Indices : }Tout d'abord, utilisez la surcharge de méthode.\newline
   Ensuite, utilisez l'argument de méthode par défaut dans la définition de la méthode seating\_capacity() d'une classe de bus.
        \renewcommand{\nomfichier}{q241.py}
        \begin{solution}
            \pythonfile{\chemincode \nomfichier}[][\nomfichier]
        \end{solution}
        

        \question
        Définir une propriété qui doit avoir la même valeur pour chaque instance de classe (objet)\newline

Définir un attribut de classe "color" dont la valeur par défaut est white.\newline

Utilisez le code suivant pour cet exercice.\newline
        \renewcommand{\nomfichier}{q242depart.py}
        \pythonfile{\chemincode \nomfichier}[][\nomfichier]		

Résultat attendu :\newline

Couleur : Blanc, Nom du véhicule : School Volvo, Vitesse : 180, Kilométrage : 12\newline
Couleur : Blanc, Nom du véhicule : Audi Q5, Vitesse : 240, Kilométrage : 18


        \par
        \textbf{Indices : }Définir une couleur comme variable de classe dans une classe de véhicule
        \renewcommand{\nomfichier}{q242.py}
        \begin{solution}
            \pythonfile{\chemincode \nomfichier}[][\nomfichier]
            Les variables créées dans .\_\_init\_\_() sont appelées variables d'instance. La valeur d'une variable d'instance est spécifique à une instance particulière de la classe. Par exemple, dans la solution, tous les objets Véhicule ont un nom et une vitesse maximale, mais les valeurs des variables nom et vitesse maximale varient en fonction de l'instance de Véhicule.
            
            En revanche, la variable de classe est partagée par toutes les instances de la classe. Vous pouvez définir un attribut de classe en attribuant une valeur à un nom de variable en dehors de .\_\_init\_\_().
        \end{solution}
        

        \question
        Héritage des classes

Créez une classe enfant Bus qui hérite de la classe Véhicule. Le tarif par défaut de tout véhicule est égal au \textbf{nombre de places * 100}. Si le véhicule est une instance de bus, nous devons ajouter 10 \% au tarif total à titre de frais de maintenance. Ainsi, le tarif total pour l'instance de bus deviendra le \textbf{montant final = tarif total + 10 \% du tarif total}.

Remarque : le nombre de places assises dans le bus est de 50, le montant final du tarif devrait donc être de 5500. Vous devez surcharger la méthode fare() de la classe Vehicle dans la classe Bus.

Utilisez le code suivant pour votre classe de véhicule parent. Nous devons accéder à la classe mère à partir d'une méthode d'une classe enfant.

        \renewcommand{\nomfichier}{q243depart.py}
        \pythonfile{\chemincode \nomfichier}[][\nomfichier]	

Résultat attendu :\newline

Le prix total du billet d'autobus est de 5500.0
        \par
        \renewcommand{\nomfichier}{q243.py}
        \begin{solution}
            \pythonfile{\chemincode \nomfichier}[][\nomfichier]
        \end{solution}
        

        \question
        Vérifier le type d'un objet

Écrire un programme permettant de déterminer à quelle classe appartient un objet Bus donné.
Vous partirez du code suivant :
        \renewcommand{\nomfichier}{q244depart.py}
        \pythonfile{\chemincode \nomfichier}[][\nomfichier]	
        \par
        \textbf{Indices : }Utilisez la fonction intégrée type() de Python.
        \renewcommand{\nomfichier}{q244.py}
        \begin{solution}
            \pythonfile{\chemincode \nomfichier}[][\nomfichier]
        \end{solution}
        

        \question
        Déterminer si School\_bus est également une instance de la classe Vehicle
        Vous partirez du code suivant :
        \renewcommand{\nomfichier}{q245depart.py}
        \pythonfile{\chemincode \nomfichier}[][\nomfichier]	        
        \par
        \textbf{Indices : }Utiliser la fonction isinstance()
        \renewcommand{\nomfichier}{q245.py}
        \begin{solution}
            \pythonfile{\chemincode \nomfichier}[][\nomfichier]
        \end{solution}
        

        
