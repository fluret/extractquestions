%Question 1
\renewcommand{\chemincode}{../../code/}

\question
Écrivez un programme qui trouve tous les nombres multiples de 7 mais pas de 5,
entre 2000 et 3200 (les deux inclus).
Les nombres obtenus doivent être imprimés dans une séquence séparée par des virgules sur une seule ligne.
\par
\textbf{Indices : }Utilisez la méthode range(début, fin)
\renewcommand{\nomfichier}{q001.py}
\begin{solution}
    \pythonfile{\chemincode \nomfichier}[][\nomfichier]
\end{solution}

\renewcommand{\nomfichier}{q001-01.py}
\begin{solution}
    \pythonfile{\chemincode \nomfichier}[][\nomfichier]
\end{solution}

%Question 2
\question
Écrivez un programme qui peut calculer la factorielle d'un nombre donné.

Supposons que l'entrée suivante soit fournie au programme:\par

8\newline
Ensuite, la sortie doit être:\newline
40320

\renewcommand{\nomfichier}{q002.py}
\begin{solution}
    \pythonfile{\chemincode \nomfichier}[][\nomfichier]
\end{solution}

\renewcommand{\nomfichier}{q002-01.py}
\begin{solution}
    \pythonfile{\chemincode \nomfichier}[][\nomfichier]
\end{solution}

\renewcommand{\nomfichier}{q002-02.py}
\begin{solution}
    \pythonfile{\chemincode \nomfichier}[][\nomfichier]
\end{solution}

\renewcommand{\nomfichier}{q002-03.py}
\begin{solution}
    \pythonfile{\chemincode \nomfichier}[][\nomfichier]
\end{solution}

\renewcommand{\nomfichier}{q002-04.py}
\begin{solution}
    \pythonfile{\chemincode \nomfichier}[][\nomfichier]
\end{solution}
%Question 3
\question
Avec un nombre entier \textbf{n} donné, écrivez un programme pour générer un dictionnaire qui contient \textbf{(i, i*i)} tel que \textbf{i} est un nombre entier entre \textbf{1 et n} (les deux inclus). et ensuite le programme doit imprimer le dictionnaire.

Supposons que l'entrée suivante soit fournie au programme :\newline
8\newline
La sortie devrait alors être :\newline
\{1: 1, 2: 4, 3: 9, 4: 16, 5: 25, 6: 36, 7: 49, 8: 64\}

\renewcommand{\nomfichier}{q003.py}
\begin{solution}
    \pythonfile{\chemincode \nomfichier}[][\nomfichier]
\end{solution}
\renewcommand{\nomfichier}{q003-01.py}
\begin{solution}
    \pythonfile{\chemincode \nomfichier}[][\nomfichier]
\end{solution}

%Question 4
\question
Écrire un programme qui accepte une séquence de nombres séparés par des virgules à partir de la console et qui génère une liste et un tuple contenant chaque nombre.

Supposons que l'entrée suivante soit fournie au programme :\newline
34,67,55,33,12,98\newline
Ensuite, la sortie doit être:\newline
['34', '67', '55', '33', '12', '98']\newline
('34', '67', '55', '33', '12', '98')
\renewcommand{\nomfichier}{q004.py}
\begin{solution}
    \pythonfile{\chemincode \nomfichier}[][\nomfichier]
\end{solution}
\renewcommand{\nomfichier}{q004-01.py}
\begin{solution}
    \pythonfile{\chemincode \nomfichier}[][\nomfichier]
\end{solution}

%Question 5
\question
Question POO
%Définir une classe qui possède au moins deux méthodes :\newline
%\textbf{getString} : pour obtenir une chaîne de caractères à partir de l'entrée de la console\newline
%\textbf{printString} : pour imprimer la chaîne en majuscules.\newline
%Veuillez également inclure une fonction de test simple pour tester les méthodes de la classe.
%\par
%\textbf{Indices : }Utilisez la méthode \_\_init\_\_ pour construire certains paramètres
%\renewcommand{\nomfichier}{q005.py}
%\begin{solution}
%    \pythonfile{\chemincode \nomfichier}[][\nomfichier]
%\end{solution}

%Question 6
\question
Écrivez un programme qui calcule et imprime la valeur selon la formule donnée :

Q = Racine carrée de [(2 * C * D)/H]\newline

Voici les valeurs fixes de C et H :\newline
C est 50. H est égal à 30.\newline
D est la variable dont les valeurs doivent être introduites dans votre programme dans une séquence séparée par des virgules.\newline
Exemple\newline
Supposons que le programme reçoive la séquence d'entrée suivante, séparée par des virgules :\newline
100,150,180\newline
La sortie du programme devrait être :\newline
18,22,24
\par
\textbf{Indices : }Si la sortie reçue est sous forme décimale, elle doit être arrondi à sa valeur la plus proche (par exemple, si la sortie reçue est de 26,0, elle doit être imprimée comme 26)
\renewcommand{\nomfichier}{q006.py}
\begin{solution}
    \pythonfile{\chemincode \nomfichier}[][\nomfichier]
\end{solution}

\renewcommand{\nomfichier}{q006-01.py}
\begin{solution}
    \pythonfile{\chemincode \nomfichier}[][\nomfichier]
\end{solution}

\renewcommand{\nomfichier}{q006-02.py}
\begin{solution}
    \pythonfile{\chemincode \nomfichier}[][\nomfichier]
\end{solution}

\renewcommand{\nomfichier}{q006-03.py}
\begin{solution}
    \pythonfile{\chemincode \nomfichier}[][\nomfichier]
\end{solution}

\renewcommand{\nomfichier}{q006-04.py}
\begin{solution}
    \pythonfile{\chemincode \nomfichier}[][\nomfichier]
\end{solution}
%Question 7
\question
Écrivez un programme qui prend 2 chiffres, X,Y en entrée et génère un tableau à 2 dimensions. La valeur de l'élément dans la i-ième ligne et la j-ième colonne du tableau doit être i*j.\newline
Remarque : i = 0,1.., X-1 ; j = 0,1,¡Y-1.\newline
Exemple\newline
Supposons que les entrées suivantes soient données au programme :\newline
3,5\newline
La sortie du programme devrait alors être la suivante :\newline
[[0, 0, 0, 0, 0], [0, 1, 2, 3, 4], [0, 2, 4, 6, 8]]\newline
Puis un affichage sous la forme d'un tableau :\newline
0 0 0 0 0\newline
0 1 2 3 4\newline
0 2 4 6 8
\renewcommand{\nomfichier}{q007.py}
\begin{solution}
    \pythonfile{\chemincode \nomfichier}[][\nomfichier]
\end{solution}
\renewcommand{\nomfichier}{q007-01.py}
\begin{solution}
    \pythonfile{\chemincode \nomfichier}[][\nomfichier]
\end{solution}
\renewcommand{\nomfichier}{q007-02.py}
\begin{solution}
    \pythonfile{\chemincode \nomfichier}[][\nomfichier]
\end{solution}
\renewcommand{\nomfichier}{q007-03.py}
\begin{solution}
    \pythonfile{\chemincode \nomfichier}[][\nomfichier]
\end{solution}

%Question 8
\question
Écrivez un programme qui accepte une séquence de mots séparée par des virgules en entrée et imprime les mots dans une séquence séparée par des virgules après les avoir triés de manière alphabétique.\newline
Supposons que l'entrée suivante soit fournie au programme:\newline
sans, bonjour, sac, monde\newline
Ensuite, la sortie doit être:\newline
Sac, bonjour, sans, monde

\renewcommand{\nomfichier}{q008.py}
\begin{solution}
    \pythonfile{\chemincode \nomfichier}[][\nomfichier]
\end{solution}

%Question 9
\question
Écrivez un programme qui accepte une séquence de lignes en entrée et imprime les lignes après avoir mis en majuscules tous les caractères de la phrase. La saisie d'une ligne vide lance votre traitement.\newline
Supposons que l'entrée suivante soit fournie au programme :\newline
Bonjour le monde\newline
C'est en forgeant qu'on devient forgeron\newline
La sortie devrait alors être :\newline
BONJOUR AU MONDE\newline
C'EST EN FORGEANT QU'ON DEVIENT FORGERON
\renewcommand{\nomfichier}{q009.py}
\begin{solution}
    \pythonfile{\chemincode \nomfichier}[][\nomfichier]
\end{solution}

\renewcommand{\nomfichier}{q009-01.py}
\begin{solution}
    \pythonfile{\chemincode \nomfichier}[][\nomfichier]
\end{solution}

%Question 10
\question
Écrivez un programme qui accepte une séquence de mots séparés dans l'espace en entrée et imprime les mots après avoir retiré tous les mots en double et les tris de manière alphanumérique.\newline
Supposons que l'entrée suivante soit fournie au programme:\newline
Bonjour le monde et la pratique rend à nouveau le monde parfait et bonjour\newline
La sortie doit être:\newline
Bonjour bonjour et la le monde nouveau parfait pratique rend à
\par
\textbf{Indices : }
Nous utilisons le conteneur \textbf{set} pour supprimer automatiquement les données dupliqués.
\renewcommand{\nomfichier}{q010.py}
\begin{solution}
    \pythonfile{\chemincode \nomfichier}[][\nomfichier]
\end{solution}
\renewcommand{\nomfichier}{q010-01.py}
\begin{solution}
    \pythonfile{\chemincode \nomfichier}[][\nomfichier]
\end{solution}
\renewcommand{\nomfichier}{q010-02.py}
\begin{solution}
    \pythonfile{\chemincode \nomfichier}[][\nomfichier]
\end{solution}

%Question 11
\question
Écrivez un programme qui accepte une séquence de nombres binaires à 4 chiffres séparés par des virgules comme entrée, puis vérifiez s'ils sont divisibles par 5 ou non.Les nombres divisibles par 5 doivent être imprimés dans une séquence séparée par des virgules.\newline
Exemple:\newline
0100,0011,1010,1001\newline
Qui correspondent respectivement à 4, 3, 10 et 9.\newline
Alors la sortie doit être:\newline
1010
\renewcommand{\nomfichier}{q011.py}
\begin{solution}
    \pythonfile{\chemincode \nomfichier}[][\nomfichier]
\end{solution}

\renewcommand{\nomfichier}{q011-01.py}
\begin{solution}
    \pythonfile{\chemincode \nomfichier}[][\nomfichier]
\end{solution}

\renewcommand{\nomfichier}{q011-02.py}
\begin{solution}
    \pythonfile{\chemincode \nomfichier}[][\nomfichier]
\end{solution}
\renewcommand{\nomfichier}{q011-03.py}
\begin{solution}
    \pythonfile{\chemincode \nomfichier}[][\nomfichier]
\end{solution}

%Question 12
\question
Écrivez un programme, qui trouvera tous les chiffres entre 1000 et 3000 (tous deux inclus) pour lesquels chaque chiffre du nombre est pair.
Les nombres obtenus doivent être imprimés dans une séquence séparée par des virgules sur une seule ligne.

\renewcommand{\nomfichier}{q012.py}
\begin{solution}
    \pythonfile{\chemincode \nomfichier}[][\nomfichier]
\end{solution}
\renewcommand{\nomfichier}{q012-01.py}
\begin{solution}
    \pythonfile{\chemincode \nomfichier}[][\nomfichier]
\end{solution}
\renewcommand{\nomfichier}{q012-02.py}
\begin{solution}
    \pythonfile{\chemincode \nomfichier}[][\nomfichier]
\end{solution}
\renewcommand{\nomfichier}{q012-03.py}
\begin{solution}
    \pythonfile{\chemincode \nomfichier}[][\nomfichier]
\end{solution}
\renewcommand{\nomfichier}{q012-04.py}
\begin{solution}
    \pythonfile{\chemincode \nomfichier}[][\nomfichier]
\end{solution}
%Question 13
\question
Écrivez un programme qui accepte une phrase et qui calcule le nombre de lettres et de chiffres.\newline
Supposons que l'entrée suivante soit fournie au programme:\newline
Bonjour le monde!123\newline
Ensuite, la sortie doit être:\newline
Lettres 14\newline
Chiffres 3
\renewcommand{\nomfichier}{q013.py}
\begin{solution}
    \pythonfile{\chemincode \nomfichier}[][\nomfichier]
\end{solution}
\renewcommand{\nomfichier}{q013-01.py}
\begin{solution}
    \pythonfile{\chemincode \nomfichier}[][\nomfichier]
\end{solution}

\renewcommand{\nomfichier}{q013-02.py}
\begin{solution}
    \pythonfile{\chemincode \nomfichier}[][\nomfichier]
\end{solution}

%Question 14
\question
Écrivez un programme qui accepte une phrase et calculez le nombre de lettres en majuscules et de lettres minuscules.\newline
Supposons que l'entrée suivante soit fournie au programme:\newline
BonJour le Monde!\newline
Ensuite, la sortie doit être:\newline
Majuscules 3\newline
Minuscules 11
\renewcommand{\nomfichier}{q014.py}
\begin{solution}
    \pythonfile{\chemincode \nomfichier}[][\nomfichier]
\end{solution}
\renewcommand{\nomfichier}{q014-01.py}
\begin{solution}
    \pythonfile{\chemincode \nomfichier}[][\nomfichier]
\end{solution}
\renewcommand{\nomfichier}{q014-02.py}
\begin{solution}
    \pythonfile{\chemincode \nomfichier}[][\nomfichier]
\end{solution}
\renewcommand{\nomfichier}{q014-03.py}
\begin{solution}
    \pythonfile{\chemincode \nomfichier}[][\nomfichier]
\end{solution}
\renewcommand{\nomfichier}{q014-04.py}
\begin{solution}
    \pythonfile{\chemincode \nomfichier}[][\nomfichier]
\end{solution}
%Question 15
\question
Écrivez un programme qui calcule la valeur d'un a + aa + aaa + aaaa avec un chiffre donné comme valeur de a.\newline
Supposons que l'entrée suivante soit fournie au programme:\newline
9\newline
Ensuite, la sortie doit être:\newline
Le résultat de : 9 + 99 + 999 + 9999\newline
est : 11106

\renewcommand{\nomfichier}{q015.py}
\begin{solution}
    \pythonfile{\chemincode \nomfichier}[][\nomfichier]
\end{solution}

\renewcommand{\nomfichier}{q015-01.py}
\begin{solution}
    \pythonfile{\chemincode \nomfichier}[][\nomfichier]
\end{solution}
\renewcommand{\nomfichier}{q015-02.py}
\begin{solution}
    \pythonfile{\chemincode \nomfichier}[][\nomfichier]
\end{solution}

%Question 16
\question
Utilisez une compréhension de liste pour élever au carré chaque nombre impair d'une liste. La liste est introduite par une séquence de nombres séparés par des virgules.\newline
Supposons que l'entrée suivante soit fournie au programme :\newline
1,2,3,4,5,6,7,8,9\newline
La sortie devrait alors être :\newline
1,9,25,49,81

\renewcommand{\nomfichier}{q016.py}
\begin{solution}
    \pythonfile{\chemincode \nomfichier}[][\nomfichier]
\end{solution}

%Question 17
\question
Écrivez un programme qui calcule le montant net d'un compte bancaire basé sur un journal de transaction à partir de l'entrée de la console

Le format de journal des transactions est affiché comme suit:\newline
D 100\newline
W 200\newline

D signifie dépôt et w retrait.\newline
Supposons que l'entrée suivante soit fournie au programme:\newline
D 300\newline
D 300\newline
W 200\newline
D 100\newline
Ensuite, la sortie doit être:\newline
500
\renewcommand{\nomfichier}{q017.py}
\begin{solution}
    \pythonfile{\chemincode \nomfichier}[][\nomfichier]
\end{solution}
\renewcommand{\nomfichier}{q017-01.py}
\begin{solution}
    \pythonfile{\chemincode \nomfichier}[][\nomfichier]
\end{solution}
\renewcommand{\nomfichier}{q017-02.py}
\begin{solution}
    \pythonfile{\chemincode \nomfichier}[][\nomfichier]
\end{solution}

%Question 18
\question
Un site Web oblige les utilisateurs à saisir le nom d'utilisateur et le mot de passe pour s'inscrire.Écrivez un programme pour vérifier la validité de la saisie du mot de passe par les utilisateurs.\newline
Voici les critères de vérification du mot de passe:
\begin{enumerate}
	\item Au moins 1 lettre entre [a-z]
	\item Au moins 1 nombre entre [0-9]
	\item Au moins 1 lettre entre [A-Z]
	\item Au moins 1 personnage de [\$ \# @]
	\item Longueur minimal : 6
	\item Longueur maximale : 12
	\item Ne doit pas contenir d'espace
\end{enumerate}
Votre programme doit accepter une séquence de mots de passe séparés par des virgules et les vérifiera conformément aux critères ci-dessus.Les mots de passe qui correspondent aux critères doivent être imprimés, chacun séparé par une virgule.\newline
Exemple\newline
Si les mots de passe suivants sont donnés en entrée au programme:\newline
ABd1234@1,a F1\#,2w3E*,2We3345\newline
Ensuite, la sortie du programme doit être:\newline
AbD1234@1

\renewcommand{\nomfichier}{q018.py}
\begin{solution}
    \pythonfile{\chemincode \nomfichier}[][\nomfichier]
\end{solution}
\renewcommand{\nomfichier}{q018-01.py}
\begin{solution}
    \pythonfile{\chemincode \nomfichier}[][\nomfichier]
\end{solution}
\renewcommand{\nomfichier}{q018-02.py}
\begin{solution}
    \pythonfile{\chemincode \nomfichier}[][\nomfichier]
\end{solution}
\renewcommand{\nomfichier}{q018-03.py}
\begin{solution}
    \pythonfile{\chemincode \nomfichier}[][\nomfichier]
\end{solution}
\renewcommand{\nomfichier}{q018-04.py}
\begin{solution}
    \pythonfile{\chemincode \nomfichier}[][\nomfichier]
\end{solution}
\renewcommand{\nomfichier}{q018-05.py}
\begin{solution}
    \pythonfile{\chemincode \nomfichier}[][\nomfichier]
\end{solution}

%Question 19
\question
Vous devez rédiger un programme pour trier les tuples (nom, âge, hauteur) par ordre croissant où le nom est une chaîne, l'âge et la taille sont des entiers.Les tuples sont entrés par console.\newline

Les critères de tri sont:\newline
\begin{enumerate}
\item Trier basé sur le nom;
\item puis trier en fonction de l'âge;
\item Puis triez par la taille.
\end{enumerate}

Si les tuples suivants sont donnés comme entrée au programme:\newline
Tom,19,80\newline
John,20,90\newline
Jony,17,91\newline
Jony,17,93\newline
Json,21,85
Ensuite, la sortie du programme doit être:\newline
[('John', '20', '90'), ('Jony', '17', '91'), ('Jony', '17', '93'), ('Json', '21', '85'), ('Tom', '19', '80')]

\renewcommand{\nomfichier}{q019.py}
\begin{solution}
    \pythonfile{\chemincode \nomfichier}[][\nomfichier]
\end{solution}

\renewcommand{\nomfichier}{q019-01.py}
\begin{solution}
    \pythonfile{\chemincode \nomfichier}[][\nomfichier]
\end{solution}

%Question 20
\question
Question POO
%Définissez une classe avec un générateur qui peut itérer les nombres, qui sont divisibles par 7, entre une plage donnée 0 et n.
%Par exemple l'entrée suivante :\newline
%18\newline
%donne la sortie :\newline
%0\newline
%7\newline
%14\newline
%
%\renewcommand{\nomfichier}{q020.py}
%\begin{solution}
%    \pythonfile{\chemincode \nomfichier}[][\nomfichier]
%\end{solution}
%\renewcommand{\nomfichier}{q020-01.py}
%\begin{solution}
%    \pythonfile{\chemincode \nomfichier}[][\nomfichier]
%\end{solution}
%\renewcommand{\nomfichier}{q020-02.py}
%\begin{solution}
%    \pythonfile{\chemincode \nomfichier}[][\nomfichier]
%\end{solution}
%Question 21
\question
Un robot se déplace dans un avion à partir du point d'origine (0,0).Le robot peut se déplacer vers le haut, le bas, la gauche et la droite.\newline
La trace du mouvement du robot est indiquée comme suit:\newline
UP 5\newline
DOWN 3\newline
LEFT 3\newline
RIGHT 2

Les nombres qui suivent la direction sont des pas.\newline
Veuillez écrire un programme pour calculer la distance entre la position actuelle après une séquence de mouvements et le point d'origine. Si la distance est un flottant, il suffit d'imprimer l'entier le plus proche.\newline
Exemple:\newline
Si les tuples suivants sont donnés comme entrée au programme:
UP 5\newline
DOWN 3\newline
LEFT 3\newline
RIGHT 2\newline
Ensuite, la sortie du programme doit être:\newline
2

\renewcommand{\nomfichier}{q021.py}
\begin{solution}
    \pythonfile{\chemincode \nomfichier}[][\nomfichier]
\end{solution}

\renewcommand{\nomfichier}{q021-01.py}
\begin{solution}
    \pythonfile{\chemincode \nomfichier}[][\nomfichier]
\end{solution}

%Question 22
\question
Écrivez un programme pour calculer la fréquence des mots à partir de l'entrée.La sortie doit sortir après le tri de la clé de manière alphanumérique.\newline
Supposons que l'entrée suivante soit fournie au programme:\newline
Nouveau sur Python ou choisir entre Python 2 et Python 3 ? Lisez Python 2 ou Python 3.\newline
Ensuite, la sortie doit être:\newline
2:2\newline
3:1\newline
3.:1\newline
?:1\newline
Lisez:1\newline
Nouveau:1\newline
Python:5\newline
choisir:1\newline
entre:1\newline
et:1\newline
ou:2\newline
sur:1

\renewcommand{\nomfichier}{q022.py}
\begin{solution}
    \pythonfile{\chemincode \nomfichier}[][\nomfichier]
\end{solution}
\renewcommand{\nomfichier}{q022-01.py}
\begin{solution}
    \pythonfile{\chemincode \nomfichier}[][\nomfichier]
\end{solution}
\renewcommand{\nomfichier}{q022-02.py}
\begin{solution}
    \pythonfile{\chemincode \nomfichier}[][\nomfichier]
\end{solution}
\renewcommand{\nomfichier}{q022-03.py}
\begin{solution}
    \pythonfile{\chemincode \nomfichier}[][\nomfichier]
\end{solution}
\renewcommand{\nomfichier}{q022-04.py}
\begin{solution}
    \pythonfile{\chemincode \nomfichier}[][\nomfichier]
\end{solution}




%Question 23
\question
Écrire une fonction qui peut calculer la valeur carrée d'un nombre.

\renewcommand{\nomfichier}{q023.py}
\begin{solution}
    \pythonfile{\chemincode \nomfichier}[][\nomfichier]
\end{solution}

%Question 24
\question
Python possède de nombreuses fonctions intégrées, il a une fonction de documentation intégrée pour toutes ses fonctions.
Veuillez écrire un programme pour imprimer la documentation des fonctions suivantes :
\begin{itemize}
\item abs()
\item int()
\item input ()
\end{itemize}
Puis écrire une fonction qui peut calculer la valeur carrée d'un nombre
et lui ajouter une documentation.

\renewcommand{\nomfichier}{q024.py}
\begin{solution}
    \pythonfile{\chemincode \nomfichier}[][q024.py]
\end{solution}

%Question 25
\question
Question POO
%    Définir une classe qui a un paramètre de classe et un même paramètre d'instance.
%    
%\par
%\textbf{Indices : }
%\begin{itemize}
%\item Pour définir un paramètre d'instance, il faut l'ajouter dans la méthode \_\_init\_\_.
%\item Vous pouvez initialiser un objet avec un paramètre de construction ou en définir la valeur ultérieurement.
%\end{itemize}
%\renewcommand{\nomfichier}{q025.py}
%\begin{solution}
%    \pythonfile{\chemincode \nomfichier}[][\nomfichier]
%\end{solution}
%\renewcommand{\nomfichier}{q025-01.py}
%\begin{solution}
%    \pythonfile{\chemincode \nomfichier}[][\nomfichier]
%\end{solution}
%\renewcommand{\nomfichier}{q025-02.py}
%\begin{solution}
%    \pythonfile{\chemincode \nomfichier}[][\nomfichier]
%\end{solution}

%Question 26
\question
Définissez une fonction qui peut calculer la somme de deux nombres.
\par
\textbf{Indices : }Définissez une fonction avec deux nombres comme arguments.Vous pouvez calculer la somme dans la fonction et renvoyer la valeur.
\renewcommand{\nomfichier}{q026.py}
\begin{solution}
    \pythonfile{\chemincode \nomfichier}[][\nomfichier]
\end{solution}

%Question 27
\question
Définissez une fonction qui peut convertir un entier en une chaîne et l'imprimer dans la console.
\par
\textbf{Indices : }Utilisez STR () pour convertir un nombre en chaîne.
\renewcommand{\nomfichier}{q027.py}
\begin{solution}
    \pythonfile{\chemincode \nomfichier}[][\nomfichier]
\end{solution}

%Question 28
\question
Définir une fonction qui peut recevoir deux nombres entiers sous forme de chaîne de caractères et calculer leur somme, puis l'imprimer dans la console.
\par
\textbf{Indices : }Utilisez int() pour convertir une chaîne en entier.
\renewcommand{\nomfichier}{q028.py}
\begin{solution}
    \pythonfile{\chemincode \nomfichier}[][\nomfichier]
\end{solution}

%Question 29
\question
Définissez une fonction qui peut accepter deux chaînes en entrée et les concaténer, puis l'imprimer dans la console.
\par
\textbf{Indices : }Utiliser + pour concaténer les chaines
\renewcommand{\nomfichier}{q029.py}
\begin{solution}
    \pythonfile{\chemincode \nomfichier}[][\nomfichier]
\end{solution}

%Question 30
\question
Définir une fonction capable d'accepter deux chaînes de caractères en entrée et d'imprimer la chaîne de caractères de longueur maximale dans la console. Si les deux chaînes ont la même longueur, la fonction doit imprimer les deux une par ligne.
\par
\textbf{Indices : }Utilisez la fonction Len() pour obtenir la longueur d'une chaîne
\renewcommand{\nomfichier}{q030.py}
\begin{solution}
    \pythonfile{\chemincode \nomfichier}[][\nomfichier]
\end{solution}

%Question 31
\question
Définir une fonction qui accepte un nombre entier en entrée et qui imprime "C'est un nombre pair" si le nombre est pair, sinon "C'est un nombre impair".
\par
\textbf{Indices : }Utilisez un opérateur \% pour vérifier si un nombre est pair ou impair.
\renewcommand{\nomfichier}{q031.py}
\begin{solution}
    \pythonfile{\chemincode \nomfichier}[][\nomfichier]
\end{solution}

%Question 32
\question
Définir une fonction capable d'imprimer un dictionnaire dont les clés sont des nombres compris entre 1 et 3 (les deux inclus) et dont les valeurs sont des carrés des clés.
\par
\textbf{Indices : }
\begin{itemize}
	\item Utiliser le modèle dict[key]=value pour placer une entrée dans un dictionnaire.
	\item Utiliser l'opérateur ** pour obtenir la puissance d'un nombre.
\end{itemize}
\renewcommand{\nomfichier}{q032.py}
\begin{solution}
    \pythonfile{\chemincode \nomfichier}[][\nomfichier]
\end{solution}
\renewcommand{\nomfichier}{q032-01.py}
\begin{solution}
    \pythonfile{\chemincode \nomfichier}[][\nomfichier]
\end{solution}
\renewcommand{\nomfichier}{q032-02.py}
\begin{solution}
    \pythonfile{\chemincode \nomfichier}[][\nomfichier]
\end{solution}

%Question 33
\question
Définir une fonction capable d'imprimer un dictionnaire dont les clés sont des nombres compris entre 1 et 20 (les deux inclus) et dont les valeurs sont des carrés de clés.
\par
\textbf{Indices : }
\begin{itemize}
\item Utiliser le modèle dict[key]=value pour placer une entrée dans un dictionnaire.
\item Utiliser l'opérateur ** pour obtenir la puissance d'un nombre.
\item Utiliser range() pour les boucles.
\end{itemize}
\renewcommand{\nomfichier}{q033.py}
\begin{solution}
    \pythonfile{\chemincode \nomfichier}[][\nomfichier]
\end{solution}
\renewcommand{\nomfichier}{q033-01.py}
\begin{solution}
    \pythonfile{\chemincode \nomfichier}[][\nomfichier]
\end{solution}

%Question 34
\question
Définir une fonction capable de générer un dictionnaire dont les clés sont des nombres compris entre 1 et 20 (les deux inclus) et dont les valeurs sont des carrés de clés. La fonction ne doit imprimer que les valeurs.
\par
\textbf{Indices : }
\begin{itemize}
	\item Utiliser le modèle dict[key]=value pour placer une entrée dans un dictionnaire.
	\item Utiliser l'opérateur ** pour obtenir la puissance d'un nombre.
	\item Utiliser range() pour les boucles.
	\item Utiliser values() pour itérer les clés dans le dictionnaire. Nous pouvons également utiliser items() pour obtenir des paires clé/valeur.
\end{itemize}

\renewcommand{\nomfichier}{q034.py}
\begin{solution}
    \pythonfile{\chemincode \nomfichier}[][\nomfichier]
\end{solution}

%Question 35
\question
Définir une fonction capable de générer un dictionnaire dont les clés sont des nombres compris entre 1 et 20 (les deux inclus) et dont les valeurs sont des carrés de clés. La fonction ne doit imprimer que les clés.
\par
\textbf{Indices : }
\begin{itemize}
	\item Utiliser le modèle dict[key]=value pour placer une entrée dans un dictionnaire.
	\item Utiliser l'opérateur ** pour obtenir la puissance d'un nombre.
	\item Utiliser range() pour les boucles.
	\item Utiliser keys() pour itérer les clés dans le dictionnaire. Nous pouvons également utiliser items() pour obtenir des paires clé/valeur.
\end{itemize}
\renewcommand{\nomfichier}{q035.py}
\begin{solution}
    \pythonfile{\chemincode \nomfichier}[][\nomfichier]
\end{solution}

%Question 36
\question
Définir une fonction capable de générer et d'imprimer une liste dont les valeurs sont des carrés de nombres compris entre 1 et 20 (les deux inclus).
\par
\textbf{Indices : }
\begin{itemize}
	\item Utilisez ** Opérateur pour obtenir la puissance d'un nombre.
	\item Utilisez la range() pour les boucles.
	\item Utilisez list.append() pour ajouter des valeurs dans une liste.
\end{itemize}
\renewcommand{\nomfichier}{q036.py}
\begin{solution}
    \pythonfile{\chemincode \nomfichier}[][\nomfichier]
\end{solution}

%Question 37
\question
Définir une fonction capable de générer une liste dont les valeurs sont des carrés de nombres compris entre 1 et 20 (les deux inclus). La fonction doit ensuite imprimer les 5 derniers éléments de la liste.
\par
\textbf{Indices : }
\begin{itemize}
	\item Utilisez ** Opérateur pour obtenir la puissance d'un nombre.
	\item Utilisez la range() pour les boucles.
	\item Utilisez list.append() pour ajouter des valeurs dans une liste.
	\item Utilisez [N1: N2] pour slicer une liste
\end{itemize}
\renewcommand{\nomfichier}{q037.py}
\begin{solution}
    \pythonfile{\chemincode \nomfichier}[][\nomfichier]
\end{solution}

%Question 38
\question
Définir une fonction capable de générer et d'imprimer un tuple dont les valeurs sont des carrés de nombres compris entre 1 et 20 (les deux inclus).
\par
\textbf{Indices : }
\begin{itemize}
\item Utilisez ** Opérateur pour obtenir la puissance d'un nombre.
\item Utilisez la range() pour les boucles.
\item Utilisez list.append() pour ajouter des valeurs dans une liste.
\item Utilisez tuple() pour obtenir un tuple d'une liste.
\end{itemize}
\renewcommand{\nomfichier}{q038.py}
\begin{solution}
    \pythonfile{\chemincode \nomfichier}[][\nomfichier]
\end{solution}

\renewcommand{\nomfichier}{q038-01.py}
\begin{solution}
    \pythonfile{\chemincode \nomfichier}[][\nomfichier]
\end{solution}

%Question 39
\question
Ecrivez un programme pour générer et imprimer un autre tuple dont les valeurs sont des nombres pairs dans le tuple donné (1,2,3,4,5,6,7,8,9,10).
\par
\textbf{Indices : }
\begin{itemize}
\item Utilisez "for" pour itérer le tuple
\item Utilisez Tuple() pour générer un tuple à partir d'une liste.
\end{itemize}
\renewcommand{\nomfichier}{q039.py}
\begin{solution}
    \pythonfile{\chemincode \nomfichier}[][\nomfichier]
\end{solution}
\renewcommand{\nomfichier}{q039-01.py}
\begin{solution}
    \pythonfile{\chemincode \nomfichier}[][\nomfichier]
\end{solution}
\renewcommand{\nomfichier}{q039-02.py}
\begin{solution}
    \pythonfile{\chemincode \nomfichier}[][\nomfichier]
\end{solution}

%Question 40
\question
Écrire un programme qui accepte une chaîne de caractères en entrée pour imprimer "Oui" si la chaîne est "oui" ou "OUI" ou "Oui", sinon imprimer "Non".

\renewcommand{\nomfichier}{q040.py}
\begin{solution}
    \pythonfile{\chemincode \nomfichier}[][\nomfichier]
\end{solution}

%Question 41
\question
Écrivez un programme qui peut filtrer les nombres pairs dans une liste en utilisant la fonction filter. 

La liste est la suivante : [1,2,3,4,5,6,7,8,9,10].
\par
\textbf{Indices : }
\begin{itemize}
\item Utilisez filter() pour filtrer certains éléments dans une liste.
\item Utilisez lambda pour définir des fonctions anonymes.
\end{itemize}
\renewcommand{\nomfichier}{q041.py}
\begin{solution}
    \pythonfile{\chemincode \nomfichier}[][\nomfichier]
\end{solution}

%Question 42
\question
Écrivez un programme qui peut utiliser map() pour créer une liste dont les éléments sont le carré des éléments de [1,2,3,4,5,6,7,8,9,10].
\par
\textbf{Indices : }
\begin{itemize}
\item Utilisez map() pour générer une liste.
\item Utilisez lambda pour définir des fonctions anonymes.
\end{itemize}
\renewcommand{\nomfichier}{q042.py}
\begin{solution}
    \pythonfile{\chemincode \nomfichier}[][\nomfichier]
\end{solution}

%Question 43
\question
Écrivez un programme qui peut utiliser map() et filter() pour créer une liste dont les éléments sont les carrés des nombres pairs de la liste :

[1,2,3,4,5,6,7,8,9,10].
\par
\textbf{Indices : }
\begin{itemize}
\item Utilisez map() pour générer une liste.
\item Utilisez filter() pour filtrer les éléments d'une liste.
\item Utilisez lambda pour définir des fonctions anonymes.
\end{itemize}
\renewcommand{\nomfichier}{q043.py}
\begin{solution}
    \pythonfile{\chemincode \nomfichier}[][\nomfichier]
\end{solution}

%Question 44
\question
Écrivez un programme qui peut filtrer() pour faire une liste dont les éléments sont des nombres pairs entre 1 et 20 (les deux inclus).
\par
\textbf{Indices : }
Utilisez filter() pour filtrer les éléments d'une liste.
Utilisez lambda pour définir des fonctions anonymes.
\renewcommand{\nomfichier}{q044.py}
\begin{solution}
    \pythonfile{\chemincode \nomfichier}[][\nomfichier]
\end{solution}

\renewcommand{\nomfichier}{q044-01.py}
\begin{solution}
    \pythonfile{\chemincode \nomfichier}[][\nomfichier]
\end{solution}

%Question 45
\question
Écrivez un programme qui peut utiliser map() pour créer une liste dont les éléments sont des carrés de nombres compris entre 1 et 20 (les deux inclus).
\par
\textbf{Indices : }
Utilisez map() pour générer une liste.
Utilisez lambda pour définir des fonctions anonymes.
\renewcommand{\nomfichier}{q045.py}
\begin{solution}
    \pythonfile{\chemincode \nomfichier}[][\nomfichier]
\end{solution}

%Question 46
\question
Question POO
%Définissez une classe nommée American qui possède une méthode statique appelée printNationality.
%\par
%\textbf{Indices : }
%Utilisez @StaticMethod Decorator pour définir la méthode statique de classe.
%\renewcommand{\nomfichier}{q046.py}
%\begin{solution}
%    \pythonfile{\chemincode \nomfichier}[][\nomfichier]
%\end{solution}

%Question 47
\question
Question POO
%Définissez une classe nommée American et sa sous-classe Newyorker.
%
%\renewcommand{\nomfichier}{q047.py}
%\begin{solution}
%    \pythonfile{\chemincode \nomfichier}[][\nomfichier]
%\end{solution}
%
%\renewcommand{\nomfichier}{q047-01.py}
%\begin{solution}
%    \pythonfile{\chemincode \nomfichier}[][\nomfichier]
%\end{solution}

%Question 48
\question
Question POO
%Définir une classe nommée Cercle qui peut être construite par un rayon. La classe Cercle possède une méthode qui permet de calculer la surface.
%
%
%Puis définir une classe rectangle qui peut être construit par une longueur et une largeur. La classe Rectangle possède une méthode qui permet de calculer la surface.
%
%
%
%\par
%\textbf{Indices : }
%Utilisez Def nom\_de\_le\_methode(Self) pour définir une méthode.
%\renewcommand{\nomfichier}{q048.py}
%\begin{solution}
%    \pythonfile{\chemincode \nomfichier}[][\nomfichier]
%\end{solution}
%
%\renewcommand{\nomfichier}{q048-01.py}
%\begin{solution}
%    \pythonfile{\chemincode \nomfichier}[][\nomfichier]
%\end{solution}

%Question 49
\question
Question POO
%Définissez une classe nommée Shape et sa sous-classe Square. La classe Square possède une fonction init qui prend une longueur en argument. Les deux classes disposent d'une fonction area qui permet d'imprimer l'aire de la forme, l'aire de Shape étant égale à 0 par défaut.
%\par
%\textbf{Indices : }Pour remplacer une méthode dans une super-classe, nous pouvons définir une méthode portant le même nom dans la super-classe.
%\renewcommand{\nomfichier}{q048bis.py}
%\begin{solution}
%    \pythonfile{\chemincode \nomfichier}[][\nomfichier]
%\end{solution}


%Question 50
\question
En supposant que nous avons des adresses e-mail au format \textbf{username@companyname.com}, veuillez écrire un programme pour imprimer le nom d'utilisateur d'une adresse e-mail donnée.Les noms d'utilisateurs et les noms d'entreprise sont composés de lettres uniquement.\newline

Exemple:\newline
Si l'adresse e-mail suivante est donnée comme entrée au programme:\newline

John@google.com\newline

Ensuite, la sortie du programme doit être:\newline

John\newline

\par
\textbf{Indices : }aidez vous du package "re"
\renewcommand{\nomfichier}{q049.py}
\begin{solution}
    \pythonfile{\chemincode \nomfichier}[][\nomfichier]
\end{solution}
\renewcommand{\nomfichier}{q049-01.py}
\begin{solution}
    \pythonfile{\chemincode \nomfichier}[][\nomfichier]
\end{solution}

%Question 51
\question
Écrivez un programme qui accepte une séquence de mots séparés par des espaces comme entrée et qui génère une liste contenant toutes les valeurs numériques de cette entrée.\newline
Exemple:\newline
Si les mots suivants sont donnés en entrée au programme:\newline

2 chats et 3 chiens.\newline

Ensuite, la sortie du programme doit être:\newline

['2', '3']\newline

\par
\textbf{Indices : }Utilisez re.findall() pour trouver tous les sous-chaînes à l'aide de regex.
\renewcommand{\nomfichier}{q050.py}
\begin{solution}
    \pythonfile{\chemincode \nomfichier}[][\nomfichier]
\end{solution}
\renewcommand{\nomfichier}{q050-01.py}
\begin{solution}
    \pythonfile{\chemincode \nomfichier}[][\nomfichier]
\end{solution}
\renewcommand{\nomfichier}{q050-02.py}
\begin{solution}
    \pythonfile{\chemincode \nomfichier}[][\nomfichier]
\end{solution}

%%Question 51
%\question
%Écrivez un commentaire spécial pour indiquer qu'un fichier de code source Python est dans Unicode.
%\par
%\textbf{Indices : }
%\renewcommand{\nomfichier}{q051.py}
%\begin{solution}
%    \pythonfile{\chemincode \nomfichier}[][\nomfichier]
%\end{solution}

%Question 52
\question
Écrivez un programme pour calculer:

f (n) = f (n - 1) +100 quand n> 0\newline
et f (0) = 1\newline

avec une entrée n donnée par console (n> 0).

Exemple:\newline
Si le n suivant est donné en entrée au programme:\newline

5

Ensuite, la sortie du programme doit être:\newline

500

\par
\textbf{Indices : }Nous pouvons définir une fonction récursive dans Python.
\renewcommand{\nomfichier}{q052.py}
\begin{solution}
    \pythonfile{\chemincode \nomfichier}[][\nomfichier]
\end{solution}

\renewcommand{\nomfichier}{q052-01.py}
\begin{solution}
    \pythonfile{\chemincode \nomfichier}[][\nomfichier]
\end{solution}

%Question 53
\question
La séquence Fibonacci est calculée en fonction de la formule suivante:

$f(n)=0 \textrm{ si } n=0$\newline
$f(n)=1 \textrm{ si } n=1$\newline
$f(n)=f (n - 1) + f(n - 2) \textrm{ si } n> 1$\newline

Veuillez écrire un programme pour calculer la valeur de F (n) avec une entrée n donnée par console.\newline

Exemple:\newline
Si le n suivant est donné en entrée au programme:\newline

7

Ensuite, la sortie du programme doit être:\newline

13

\par
\textbf{Indices : }Nous pouvons définir une fonction récursive dans Python.
\renewcommand{\nomfichier}{q053.py}
\begin{solution}
    \pythonfile{\chemincode \nomfichier}[][\nomfichier]
\end{solution}
\renewcommand{\nomfichier}{q053-01.py}
\begin{solution}
    \pythonfile{\chemincode \nomfichier}[][\nomfichier]
\end{solution}
\renewcommand{\nomfichier}{q053-02.py}
\begin{solution}
    \pythonfile{\chemincode \nomfichier}[][\nomfichier]
\end{solution}
\renewcommand{\nomfichier}{q053-03.py}
\begin{solution}
    \pythonfile{\chemincode \nomfichier}[][\nomfichier]
\end{solution}
%Question 54
\question
La séquence Fibonacci est calculée en fonction de la formule suivante:


$f(n)=0 \textrm{ si } n=0$\newline
$f(n)=1 \textrm{ si } n=1$\newline
$f(n)=f (n - 1) + f(n - 2) \textrm{ si } n> 1$\newline

Veuillez écrire un programme en utilisant la compréhension de la liste pour imprimer la séquence Fibonacci sous forme de virgule séparée avec une entrée N donnée par console.\newline

Exemple:\newline
Si le n suivant est donné en entrée au programme:\newline

7

Ensuite, la sortie du programme doit être:\newline

0,1,1,2,3,5,8,13
\par
\textbf{Indices : }
\begin{itemize}
\item Nous pouvons définir une fonction récursive dans Python.
\item Utilisez la compréhension de la liste pour générer une liste à partir d'une liste existante.
\item Utilisez <string>.Join() pour concaténer une liste de chaînes.
\end{itemize}
\renewcommand{\nomfichier}{q054.py}
\begin{solution}
    \pythonfile{\chemincode \nomfichier}[][\nomfichier]
\end{solution}
\renewcommand{\nomfichier}{q054-01.py}
\begin{solution}
    \pythonfile{\chemincode \nomfichier}[][\nomfichier]
\end{solution}
\renewcommand{\nomfichier}{q054-02.py}
\begin{solution}
    \pythonfile{\chemincode \nomfichier}[][\nomfichier]
\end{solution}
\renewcommand{\nomfichier}{q054-03.py}
\begin{solution}
    \pythonfile{\chemincode \nomfichier}[][\nomfichier]
\end{solution}



%Question 55
\question
Écrire un programme à l'aide du générateur pour imprimer les nombres pair entre 0 et N sous forme d'une suite de valeur séparées par des virgules. La valeur N est fournie par l'utilisateur.\newline

Exemple:\newline
Si la valeur de N est :\newline

10

La sortie du programme doit être:\newline

0,2,4,6,8,10
\par
\textbf{Indices : }Utilisez yield pour produire la valeur suivante dans le générateur.

\renewcommand{\nomfichier}{q055.py}
\begin{solution}
    \pythonfile{\chemincode \nomfichier}[][\nomfichier]
\end{solution}
\renewcommand{\nomfichier}{q055-01.py}
\begin{solution}
    \pythonfile{\chemincode \nomfichier}[][\nomfichier]
\end{solution}

%Question 56
\question
Veuillez écrire un programme utilisant un générateur pour imprimer les nombres divisibles par 5 et 7 entre 0 et n sous la forme d'une liste séparée par des virgules. La valeur n est fournie par l'utilisateur.\newline

Exemple:\newline
Si le n suivant est donné en entrée au programme:\newline

100

Ensuite, la sortie du programme doit être:\newline

0,35,70
\par
\textbf{Indices : }Utilisez le yield pour produire la valeur suivante dans le générateur.
\renewcommand{\nomfichier}{q056.py}
\begin{solution}
    \pythonfile{\chemincode \nomfichier}[][\nomfichier]
\end{solution}
\renewcommand{\nomfichier}{q056-01.py}
\begin{solution}
    \pythonfile{\chemincode \nomfichier}[][\nomfichier]
\end{solution}


%Question 57
\question
Écrire un code pour vérifier que tous les nombres de la liste [2,4,6,8] sont pairs.
\par
\textbf{Indices : }Utilisez "assert expression" pour effectuer l'opération.
\renewcommand{\nomfichier}{q057.py}
\begin{solution}
    \pythonfile{\chemincode \nomfichier}[][\nomfichier]
\end{solution}

%Question 58
\question
Veuillez écrire une fonction de recherche binaire qui recherche un élément dans une liste triée. La fonction doit renvoyer l'index de l'élément à rechercher dans la liste.
\par
\textbf{Indices : }Utilisez if / elif pour gérer les conditions.
\renewcommand{\nomfichier}{q058.py}
\begin{solution}
    \pythonfile{\chemincode \nomfichier}[][\nomfichier]
\end{solution}
\renewcommand{\nomfichier}{q058-01.py}
\begin{solution}
    \pythonfile{\chemincode \nomfichier}[][\nomfichier]
\end{solution}
\renewcommand{\nomfichier}{q058-02.py}
\begin{solution}
    \pythonfile{\chemincode \nomfichier}[][\nomfichier]
\end{solution}
\renewcommand{\nomfichier}{q058-03.py}
\begin{solution}
    \pythonfile{\chemincode \nomfichier}[][\nomfichier]
\end{solution}

%Question 59
\question
Veuillez générer un flottant aléatoire où la valeur se situe entre 10 et 100 à l'aide du module math.
\par
\textbf{Indices : }Utilisez random.random () pour générer un flottant aléatoire dans [0,1].
\renewcommand{\nomfichier}{q059.py}
\begin{solution}
    \pythonfile{\chemincode \nomfichier}[][\nomfichier]
\end{solution}
\renewcommand{\nomfichier}{q059-01.py}
\begin{solution}
    \pythonfile{\chemincode \nomfichier}[][\nomfichier]
\end{solution}

%Question 60
\question
Veuillez écrire un programme pour produire un nombre pair aléatoire entre 0 et 10 inclus en utilisant le module aléatoire et la compréhension de la liste.
\par
\textbf{Indices : }Utilisez random.choice() à un élément aléatoire d'une liste.
\renewcommand{\nomfichier}{q060.py}
\begin{solution}
    \pythonfile{\chemincode \nomfichier}[][\nomfichier]
\end{solution}
\renewcommand{\nomfichier}{q060-01.py}
\begin{solution}
    \pythonfile{\chemincode \nomfichier}[][\nomfichier]
\end{solution}


%Question 61
\question
Veuillez rédiger un programme pour générer une liste avec 5 nombres aléatoires entre 100 et 200 inclusifs.
\par
\textbf{Indices : }Utilisez random.sample() pour générer une liste de valeurs aléatoires.
\renewcommand{\nomfichier}{q061.py}
\begin{solution}
    \pythonfile{\chemincode \nomfichier}[][\nomfichier]
\end{solution}

%Question 62
\question
Veuillez écrire un programme pour générer de manière aléatoire une liste avec 5 nombres, qui sont divisibles par 5 et 7, entre 1 et 1000 inclusifs.
\par
\textbf{Indices : }Utilisez random.sample() pour générer une liste de valeurs aléatoires.
\renewcommand{\nomfichier}{q062.py}
\begin{solution}
    \pythonfile{\chemincode \nomfichier}[][\nomfichier]
\end{solution}

%Question 63
\question
Veuillez écrire un programme pour imprimer au hasard un numéro entier entre 7 et 15 inclusif.
\par
\textbf{Indices : }Utilisez random.randrange()
\renewcommand{\nomfichier}{q063.py}
\begin{solution}
    \pythonfile{\chemincode \nomfichier}[][\nomfichier]
\end{solution}

%Question 64
\question
Veuillez écrire un programme pour comprimer et décompresser la chaîne "Hello World! Hello World! Hello World! Hello World!".
\par
\textbf{Indices : }Utilisez zlib.compress () et zlib.decompress () pour compresser et décompresser une chaîne.
\renewcommand{\nomfichier}{q064.py}
\begin{solution}
    \pythonfile{\chemincode \nomfichier}[][\nomfichier]
\end{solution}

%Question 65
\question
Rédiger un programme pour mélanger et imprimer la liste [3,6,7,8].
\par
\textbf{Indices : }Utilisez la fonction Shuffle() pour mélanger une liste.
\renewcommand{\nomfichier}{q065.py}
\begin{solution}
    \pythonfile{\chemincode \nomfichier}[][\nomfichier]
\end{solution}
\renewcommand{\nomfichier}{q065-01.py}
\begin{solution}
    \pythonfile{\chemincode \nomfichier}[][\nomfichier]
\end{solution}

%Question 66
\question
Écrire un programme pour générer toutes les phrases où le sujet est dans ["I", "You"] et le verbe est dans ["Play", "Love"] et l'objet est dans ["Hockey", "Football"].

\renewcommand{\nomfichier}{q066.py}
\begin{solution}
    \pythonfile{\chemincode \nomfichier}[][\nomfichier]
\end{solution}
\renewcommand{\nomfichier}{q066-01.py}
\begin{solution}
    \pythonfile{\chemincode \nomfichier}[][\nomfichier]
\end{solution}

%Question 67
\question
En utilisant la compréhension de liste, veuillez écrire un programme pour imprimer la liste après avoir supprimé les nombres divisibles par 5 et 7 dans [12,24,35,70,88,120,155].

\renewcommand{\nomfichier}{q067.py}
\begin{solution}
    \pythonfile{\chemincode \nomfichier}[][\nomfichier]
\end{solution}

%Question 68
\question
En utilisant la compréhension de liste, écrivez un programme qui génère un tableau 3D 3*5*8 dont chaque élément est 0.
.
\renewcommand{\nomfichier}{q068.py}
\begin{solution}
    \pythonfile{\chemincode \nomfichier}[][\nomfichier]
\end{solution}

%Question 69
\question
En utilisant la compréhension de liste, veuillez écrire un programme pour imprimer la liste après avoir enlevé la valeur 24 dans [12,24,35,24,88,120,155].
\par
\textbf{Indices : }Utilisez la méthode de suppression de la liste pour supprimer une valeur.
\renewcommand{\nomfichier}{q069.py}
\begin{solution}
    \pythonfile{\chemincode \nomfichier}[][\nomfichier]
\end{solution}
\renewcommand{\nomfichier}{q069-01.py}
\begin{solution}
    \pythonfile{\chemincode \nomfichier}[][\nomfichier]
\end{solution}

%Question 70
\question
Question POO
%Définissez une classe Personne et ses deux classes enfants : Homme et Femme. Toutes les classes ont une méthode "getGenre" qui peut afficher "Homme" pour la classe Homme et "Femme" pour la classe Femme.
%\par
%\textbf{Indices : }Utilisez la subclass(parentClass) pour définir une classe d'enfants.
%\renewcommand{\nomfichier}{q070.py}
%\begin{solution}
%    \pythonfile{\chemincode \nomfichier}[][\nomfichier]
%\end{solution}
%\renewcommand{\nomfichier}{q070-01.py}
%\begin{solution}
%    \pythonfile{\chemincode \nomfichier}[][\nomfichier]
%\end{solution}

%Question 71
\question
Veuillez écrire un programme qui accepte une chaîne de la console et l'imprimez dans l'ordre inverse.\newline

Exemple:\newline
Si la chaîne suivante est donnée en entrée au programme:\newline

Rise pour voter Sir\newline

Ensuite, la sortie du programme doit être:\newline

riS retov ruop esiR

\renewcommand{\nomfichier}{q071.py}
\begin{solution}
    \pythonfile{\chemincode \nomfichier}[][\nomfichier]
\end{solution}

%Question 72
\question
Veuillez écrire un programme qui accepte une chaîne de caractères de la console et qui imprime les caractères qui ont des index pairs.\newline

Exemple:\newline
Si la chaîne suivante est donnée en entrée au programme:\newline

H1E2L3L4O5W6O7R8L9D

Ensuite, la sortie du programme doit être:\newline

HELLOWORLD
\par
\textbf{Indices : }Utilisez la liste [:: 2] pour itérer une liste par étape 2.
\renewcommand{\nomfichier}{q072.py}
\begin{solution}
    \pythonfile{\chemincode \nomfichier}[][\nomfichier]
\end{solution}

\renewcommand{\nomfichier}{q072-01.py}
\begin{solution}
    \pythonfile{\chemincode \nomfichier}[][\nomfichier]
\end{solution}

\renewcommand{\nomfichier}{q072-02.py}
\begin{solution}
    \pythonfile{\chemincode \nomfichier}[][\nomfichier]
\end{solution}

%Question 73
\question
Veuillez écrire un programme qui imprime toutes les permutations de [1,2,3]
\par
\textbf{Indices : }Utilisez itertools.permutations() pour obtenir des permutations de liste.
\renewcommand{\nomfichier}{q073.py}
\begin{solution}
    \pythonfile{\chemincode \nomfichier}[][\nomfichier]
\end{solution}

\renewcommand{\nomfichier}{q073-01.py}
\begin{solution}
    \pythonfile{\chemincode \nomfichier}[][\nomfichier]
\end{solution}

%Question 73
\question
Écrire un programme pour résoudre un casse-tête classique de la Chine ancienne :
Nous comptons 35 têtes et 94 pattes parmi les poulets et les lapins d'une ferme. Combien de lapins et de poulets avons-nous ?
\par
\textbf{Indices : }Utilisez pour la boucle pour itérer toutes les solutions possibles.
\renewcommand{\nomfichier}{q074.py}
\begin{solution}
    \pythonfile{\chemincode \nomfichier}[][\nomfichier]
\end{solution}

%Question 74
\question
Écrivez une fonction pour calculer 5/0 et utilisez try/except pour attraper les exceptions.

\renewcommand{\nomfichier}{q117.py}
\begin{solution}
    \pythonfile{\chemincode \nomfichier}[][\nomfichier]
\end{solution}
%Question 75
\question
Définir une classe d'exception personnalisée qui prend un message sous forme de chaîne comme attribut.

\renewcommand{\nomfichier}{q118.py}
\begin{solution}
    \pythonfile{\chemincode \nomfichier}[][\nomfichier]
\end{solution}

%Question 76
\question
Écrire un programme pour calculer $1/2+2/3+3/4+...+n/n+1$ avec une entrée n.
Avec la valeur suivante :\newline
5\newline
La sortie sera :\newline
3.55
\renewcommand{\nomfichier}{q119.py}
\begin{solution}
    \pythonfile{\chemincode \nomfichier}[][\nomfichier]
\end{solution}

\renewcommand{\nomfichier}{q119-01.py}
\begin{solution}
    \pythonfile{\chemincode \nomfichier}[][\nomfichier]
\end{solution}


%Question 77
\question
Veuillez écrire un programme pour imprimer la liste après avoir enlevé les nombres pairs dans [5,6,77,45,22,12,24].
\renewcommand{\nomfichier}{q120.py}
\begin{solution}
    \pythonfile{\chemincode \nomfichier}[][\nomfichier]
\end{solution}

\renewcommand{\nomfichier}{q120-01.py}
\begin{solution}
    \pythonfile{\chemincode \nomfichier}[][\nomfichier]
\end{solution}

%Question 78
\question
En utilisant la compréhension de liste, veuillez écrire un programme pour imprimer la liste après avoir enlevé les 0ème, 2ème, 4ème, 6ème nombres dans [12,24,35,70,88,120,155].
\renewcommand{\nomfichier}{q121.py}
\begin{solution}
    \pythonfile{\chemincode \nomfichier}[][\nomfichier]
\end{solution}

\renewcommand{\nomfichier}{q121-01.py}
\begin{solution}
    \pythonfile{\chemincode \nomfichier}[][\nomfichier]
\end{solution}

%Question 79
\question
En utilisant la compréhension de liste, veuillez écrire un programme pour imprimer la liste après avoir enlevé les 2ème à 4ème nombres dans [12,24,35,70,88,120,155].
\renewcommand{\nomfichier}{q122.py}
\begin{solution}
    \pythonfile{\chemincode \nomfichier}[][\nomfichier]
\end{solution}

\renewcommand{\nomfichier}{q122-01.py}
\begin{solution}
    \pythonfile{\chemincode \nomfichier}[][\nomfichier]
\end{solution}

%Question 80
\question
En utilisant la compréhension de liste, veuillez écrire un programme pour imprimer la liste après avoir enlevé les 0ème, 4ème et 5ème nombres dans [12,24,35,70,88,120,155].
\renewcommand{\nomfichier}{q123.py}
\begin{solution}
    \pythonfile{\chemincode \nomfichier}[][\nomfichier]
\end{solution}

\renewcommand{\nomfichier}{q123-01.py}
\begin{solution}
    \pythonfile{\chemincode \nomfichier}[][\nomfichier]
\end{solution}

%Question 81
\question
Avec deux listes données [1,3,6,78,35,55] et [12,24,35,24,88,120,155], écrivez un programme pour créer une liste dont les éléments sont l'intersection des listes données ci-dessus.
\renewcommand{\nomfichier}{q124.py}
\begin{solution}
    \pythonfile{\chemincode \nomfichier}[][\nomfichier]
\end{solution}

\renewcommand{\nomfichier}{q124-01.py}
\begin{solution}
    \pythonfile{\chemincode \nomfichier}[][\nomfichier]
\end{solution}

%Question 82
\question
Avec une liste donnée [12,24,35,24,88,120,155,88,120,155], écrivez un programme pour imprimer cette liste après avoir supprimé toutes les valeurs en double, en conservant l'ordre original.
\renewcommand{\nomfichier}{q125.py}
\begin{solution}
    \pythonfile{\chemincode \nomfichier}[][\nomfichier]
\end{solution}

\renewcommand{\nomfichier}{q125-01.py}
\begin{solution}
    \pythonfile{\chemincode \nomfichier}[][\nomfichier]
\end{solution}

%Question 83
\question
Veuillez écrire un programme qui compte et imprime les numéros de chaque caractère dans une chaîne de caractères saisie par la console.
Par exemple, avec l'entrée suivante :\newline
abcdefgabc\newline
La sortie est :\newline
a,2\newline
b,2\newline
c,2\newline
d,1\newline
e,1\newline
f,1\newline
g,1
\renewcommand{\nomfichier}{q126.py}
\begin{solution}
    \pythonfile{\chemincode \nomfichier}[][\nomfichier]
\end{solution}

\renewcommand{\nomfichier}{q126-01.py}
\begin{solution}
    \pythonfile{\chemincode \nomfichier}[][\nomfichier]
\end{solution}

\renewcommand{\nomfichier}{q126-02.py}
\begin{solution}
    \pythonfile{\chemincode \nomfichier}[][\nomfichier]
\end{solution}

\renewcommand{\nomfichier}{q126-03.py}
\begin{solution}
    \pythonfile{\chemincode \nomfichier}[][\nomfichier]
\end{solution}

%Question 84
\question
A partir de la feuille de résultats des participants à la journée sportive de votre université, vous devez trouver le score du deuxième. On vous donne les scores. Classez-les dans une liste et trouvez le score du deuxième.

Si la chaîne suivante est donnée en entrée au programme :\newline
5\newline
2 3 6 6 5\newline
La sortie est :\newline
5
\renewcommand{\nomfichier}{q127.py}
\begin{solution}
    \pythonfile{\chemincode \nomfichier}[][\nomfichier]
\end{solution}

\renewcommand{\nomfichier}{q127-01.py}
\begin{solution}
    \pythonfile{\chemincode \nomfichier}[][\nomfichier]
\end{solution}

\renewcommand{\nomfichier}{q127-02.py}
\begin{solution}
    \pythonfile{\chemincode \nomfichier}[][\nomfichier]
\end{solution}

%Question 85
\question
On vous donne une chaîne de caractères S et une largeur W. Votre tâche consiste à envelopper la chaîne de caractères dans un paragraphe de largeur.

Si la chaîne suivante est donnée en entrée au programme :\newline
ABCDEFGHIJKLIMNOQRSTUVWXYZ\newline
4\newline
La sortie est :\newline
ABCD\newline
EFGH\newline
IJKL\newline
IMNO\newline
QRST\newline
UVWX\newline
YZ
\renewcommand{\nomfichier}{q128.py}
\begin{solution}
    \pythonfile{\chemincode \nomfichier}[][\nomfichier]
\end{solution}

\renewcommand{\nomfichier}{q128-01.py}
\begin{solution}
    \pythonfile{\chemincode \nomfichier}[][\nomfichier]
\end{solution}

\renewcommand{\nomfichier}{q128-02.py}
\begin{solution}
    \pythonfile{\chemincode \nomfichier}[][\nomfichier]
\end{solution}

\renewcommand{\nomfichier}{q128-03.py}
\begin{solution}
    \pythonfile{\chemincode \nomfichier}[][\nomfichier]
\end{solution}

\renewcommand{\nomfichier}{q128-04.py}
\begin{solution}
    \pythonfile{\chemincode \nomfichier}[][\nomfichier]
\end{solution}

%Question 86
\question
On vous donne un nombre entier, N. Votre tâche consiste à imprimer un rangoli alphabétique de taille N. (Le rangoli est une forme d'art populaire indien basé sur la création de motifs).

Différentes tailles de rangoli alphabétique sont présentées ci-dessous :\newline
size 3\newline\newline

----c----\newline
--c-b-c--\newline
c-b-a-b-c\newline
--c-b-c--\newline
----c----\newline

size 5\newline

--------e--------\newline
------e-d-e------\newline
----e-d-c-d-e----\newline
--e-d-c-b-c-d-e--\newline
e-d-c-b-a-b-c-d-e\newline
--e-d-c-b-c-d-e--\newline
----e-d-c-d-e----\newline
------e-d-e------\newline
--------e--------
\renewcommand{\nomfichier}{q129.py}
\begin{solution}
    \pythonfile{\chemincode \nomfichier}[][\nomfichier]
\end{solution}

\renewcommand{\nomfichier}{q129-01.py}
\begin{solution}
    \pythonfile{\chemincode \nomfichier}[][\nomfichier]
\end{solution}

%Question 87
\question
Etant donné 2 ensembles d'entiers, M et N, imprimez leur différence symétrique par ordre croissant. Le terme "différence symétrique" indique les valeurs qui existent dans M ou N mais qui n'existent pas dans les deux.

La première ligne d'entrée contient un entier, M. La deuxième ligne contient M entiers séparés par des espaces.La troisième ligne contient un entier, N.La quatrième ligne contient N entiers séparés par des espaces.\newline
4\newline
2 4 5 9\newline
4\newline
2 4 11 12\newline

La sortie est :\newline
5\newline
9\newline
11\newline
12
\renewcommand{\nomfichier}{q130.py}
\begin{solution}
    \pythonfile{\chemincode \nomfichier}[][\nomfichier]
\end{solution}

%Question 88
\question
On vous donne des mots. Certains mots peuvent se répéter. Pour chaque mot, indiquez le nombre d'occurrences. L'ordre de sortie doit correspondre à l'ordre d'apparition du mot en entrée. 

Voir l'exemple d'entrée/sortie pour plus de précisions.\newline

4\newline
bcdef\newline
abcdefg\newline
bcde\newline
bcdef

La sortie est :\newline
3\newline
2 1 1
\renewcommand{\nomfichier}{q131.py}
\begin{solution}
    \pythonfile{\chemincode \nomfichier}[][\nomfichier]
\end{solution}


%Question 89
\question
Votre tâche consiste à compter la fréquence des lettres de la chaîne et à imprimer les lettres par ordre décroissant de fréquence.

Si la chaîne suivante est donnée en entrée du programme :\newline

aabbbccde

La sortie est :\newline
b 3\newline
a 2\newline
c 2\newline
d 1\newline
e 1
\renewcommand{\nomfichier}{q131.py}
\begin{solution}
    \pythonfile{\chemincode \nomfichier}[][\nomfichier]
\end{solution}

\renewcommand{\nomfichier}{q131-01.py}
\begin{solution}
    \pythonfile{\chemincode \nomfichier}[][\nomfichier]
\end{solution}

\renewcommand{\nomfichier}{q131-02.py}
\begin{solution}
    \pythonfile{\chemincode \nomfichier}[][\nomfichier]
\end{solution}
