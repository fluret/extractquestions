%Question 1
\question
Définir une fonction nommée is\_two. Elle doit accepter une entrée et retourner True si l'entrée passée est soit le nombre, soit la chaîne 2, False sinon.

\renewcommand{\nomfichier}{q200.py}
\begin{solution}
    \pythonfile{\chemincode \nomfichier}[][\nomfichier]
\end{solution}

%Question 2
\question
Définir une fonction nommée is\_vowel. Elle doit renvoyer True si la chaîne passée est une voyelle, False sinon.

\renewcommand{\nomfichier}{q201.py}
\begin{solution}
    \pythonfile{\chemincode \nomfichier}[][\nomfichier]
\end{solution}

%Question 3
\question
Définir une fonction nommée is\_consonant. Elle doit retourner True si la chaîne passée est une consonne, False sinon. Utilisez votre fonction is\_vowel

\renewcommand{\nomfichier}{q202.py}
\begin{solution}
    \pythonfile{\chemincode \nomfichier}[][\nomfichier]
\end{solution}

%Question 4
\question
Définir une fonction qui accepte une chaîne de caractères qui est un mot. La fonction doit mettre en majuscule la première lettre du mot si celui-ci commence par une consonne.


\renewcommand{\nomfichier}{q203.py}
\begin{solution}
    \pythonfile{\chemincode \nomfichier}[][\nomfichier]
\end{solution}


%Question 5
\question
Définissez une fonction nommée calculate\_tip. Elle doit accepter un pourcentage de pourboire (un nombre entre 0 et 1) et le total de l'addition, et renvoyer le montant du pourboire.


\renewcommand{\nomfichier}{q204.py}
\begin{solution}
    \pythonfile{\chemincode \nomfichier}[][\nomfichier]
\end{solution}

%Question 6
\question
Définissez une fonction nommée apply\_discount. Elle doit accepter un prix d'origine et un pourcentage de remise, et renvoyer le prix après la remise.


\renewcommand{\nomfichier}{q205.py}
\begin{solution}
    \pythonfile{\chemincode \nomfichier}[][\nomfichier]
\end{solution}

%Question 7
\question
Définissez une fonction nommée handle\_commas. Elle doit accepter en entrée une chaîne de caractères qui est un nombre contenant des virgules, et retourner un nombre en sortie.

\renewcommand{\nomfichier}{q206.py}
\begin{solution}
    \pythonfile{\chemincode \nomfichier}[][\nomfichier]
\end{solution}

%Question 8
\question
Définissez une fonction nommée get\_letter\_grade. Elle doit accepter un nombre et retourner la lettre associée à ce nombre (A-F).

\renewcommand{\nomfichier}{q207.py}
\begin{solution}
    \pythonfile{\chemincode \nomfichier}[][\nomfichier]
\end{solution}

%Question 9
\question
Définissez une fonction nommée remove\_vowels qui accepte une chaîne et renvoie une chaîne dont toutes les voyelles ont été supprimées.

\renewcommand{\nomfichier}{q208.py}
\begin{solution}
    \pythonfile{\chemincode \nomfichier}[][\nomfichier]
\end{solution}

%Question 10
\question
Définissez une fonction nommée normalize\_name. Elle doit accepter une chaîne et retourner un identifiant python valide, c'est-à-dire :\newline
\begin{itemize}
	\item tout ce qui n'est pas un identifiant python valide doit être supprimé
	\item les espaces blancs de début et de fin doivent être supprimés
	\item tout doit être en minuscules
	\item les espaces doivent être remplacés par des traits de soulignement
\end{itemize}

par exemple :\newline
\begin{itemize}
	\item Nom deviendra nom
	\item Prénom deviendra prénom
	\item Completed deviendra completed
\end{itemize}

\renewcommand{\nomfichier}{q209.py}
\begin{solution}
    \pythonfile{\chemincode \nomfichier}[][\nomfichier]
\end{solution}

%Question 11
\question
Écrivez une fonction nommée cumulative\_sum qui accepte une liste de nombres et renvoie une liste qui est la somme cumulative des nombres de la liste.
\begin{itemize}
\item cumulative\_sum([1, 1, 1]) renvoie [1, 2, 3]
\item cumulative\_sum([1, 2, 3, 4]) renvoie [1, 3, 6, 10]
\end{itemize}

\renewcommand{\nomfichier}{q210.py}
\begin{solution}
    \pythonfile{\chemincode \nomfichier}[][\nomfichier]
\end{solution}

%Question 12
\question
Soit les deux listes suivantes :\newline
fruits = ['mango', 'kiwi', 'strawberry', 'guava', 'pineapple', 'mandarin orange']\newline
numbers = [2, 3, 4, 5, 6, 7, 8, 9, 10, 11, 13, 17, 19, 23, 256, -8, -4, -2, 5, -9]\newline
Réécrire l'exemple de code ci-dessus en utilisant la syntaxe de compréhension de liste. Créez une variable nommée uppercased\_fruits pour contenir la sortie de la compréhension de liste. La sortie devrait être ['MANGO', 'KIWI', etc...].
\renewcommand{\nomfichier}{q211.py}
\begin{solution}
    \pythonfile{\chemincode \nomfichier}[][\nomfichier]
\end{solution}

%Question 13
\question
Créer une variable nommée capitalized\_fruits et utiliser la syntaxe de compréhension de liste pour produire des résultats comme ['Mango', 'Kiwi', 'Strawberry', etc...].

\renewcommand{\nomfichier}{q212.py}
\begin{solution}
    \pythonfile{\chemincode \nomfichier}[][\nomfichier]
\end{solution}

%Question 14
\question
Utilisez une compréhension de liste pour créer une variable nommée fruits\_avec\_plus\_de\_deux\_voyelles.

Astuce : Vous aurez besoin d'un moyen de vérifier si quelque chose est une voyelle.


\renewcommand{\nomfichier}{q213.py}
\begin{solution}
    \pythonfile{\chemincode \nomfichier}[][\nomfichier]
\end{solution}

%Question 15
\question
Créer une variable nommée fruits\_avec\_seulement\_deux\_voyelles.

Le résultat devrait être ['mangue', 'kiwi', 'fraise'].

\renewcommand{\nomfichier}{q214.py}
\begin{solution}
    \pythonfile{\chemincode \nomfichier}[][\nomfichier]
\end{solution}

%Question 16
\question
Faire une liste qui contient chaque fruit avec plus de 5 caractères


\renewcommand{\nomfichier}{q215.py}
\begin{solution}
    \pythonfile{\chemincode \nomfichier}[][\nomfichier]
\end{solution}

%Question 17
\question
Faire une liste qui contient chaque fruit avec exactement 5 caractères

\renewcommand{\nomfichier}{q216.py}
\begin{solution}
    \pythonfile{\chemincode \nomfichier}[][\nomfichier]
\end{solution}

%Question 18
\question
Faire une liste qui contient des fruits qui ont moins de 5 caractères

\renewcommand{\nomfichier}{q217.py}
\begin{solution}
    \pythonfile{\chemincode \nomfichier}[][\nomfichier]
\end{solution}

%Question 19
\question
Faites une liste contenant le nombre de caractères de chaque fruit. Les résultats seraient 5, 4, 10, etc...

\renewcommand{\nomfichier}{q218.py}
\begin{solution}
    \pythonfile{\chemincode \nomfichier}[][\nomfichier]
\end{solution}

%Question 20
\question
Créez une variable nommée fruits\_avec\_lettre\_a qui contient une liste des seuls fruits contenant la lettre "a"

\renewcommand{\nomfichier}{q219.py}
\begin{solution}
    \pythonfile{\chemincode \nomfichier}[][\nomfichier]
\end{solution}

%Question 21
\question
Créer une variable nommée even\_numbers qui ne contiendra que les nombres pairs

\renewcommand{\nomfichier}{q220.py}
\begin{solution}
    \pythonfile{\chemincode \nomfichier}[][\nomfichier]
\end{solution}

%Question 22
\question
Créer une variable nommée nombres\_impairs qui ne contient que les nombres impairs


\renewcommand{\nomfichier}{q221.py}
\begin{solution}
    \pythonfile{\chemincode \nomfichier}[][\nomfichier]
\end{solution}

%Question 23
\question
Créer une variable nommée nombres\_positifs qui ne contient que les nombres positifs


\renewcommand{\nomfichier}{q222.py}
\begin{solution}
    \pythonfile{\chemincode \nomfichier}[][\nomfichier]
\end{solution}

%Question 24
\question
Créer une variable nommée nombres\_négatifs qui ne contient que les nombres négatifs

\renewcommand{\nomfichier}{q223.py}
\begin{solution}
    \pythonfile{\chemincode \nomfichier}[][\nomfichier]
\end{solution}

%Question 25
\question
Utiliser une compréhension de liste avec un conditionnel afin de produire une liste de nombres avec 2 chiffres ou plus

\renewcommand{\nomfichier}{q224.py}
\begin{solution}
    \pythonfile{\chemincode \nomfichier}[][\nomfichier]
\end{solution}

%Question 26
\question
Créez une variable nommée numbers\_squared qui contient la liste des nombres avec chaque élément au carré. La sortie est [4, 9, 16, etc...]

\renewcommand{\nomfichier}{q225.py}
\begin{solution}
    \pythonfile{\chemincode \nomfichier}[][\nomfichier]
\end{solution}

%Question 27
\question
Créez une variable nommée nombres\_impairs\_négatifs qui ne contient que les nombres qui sont à la fois impairs et négatifs.

\renewcommand{\nomfichier}{q226.py}
\begin{solution}
    \pythonfile{\chemincode \nomfichier}[][\nomfichier]
\end{solution}

%Question 28
\question
Créez une variable nommée nombres\_plus\_5. Dans cette variable, renvoyez une liste contenant chaque nombre plus cinq.

\renewcommand{\nomfichier}{q227.py}
\begin{solution}
    \pythonfile{\chemincode \nomfichier}[][\nomfichier]
\end{solution}

%Question 29
\question
Créez une variable nommée "primes" qui est une liste contenant les nombres premiers de la liste des nombres. *Astuce : vous pouvez créer ou trouver une fonction d'aide qui détermine si un nombre donné est premier ou non.

\renewcommand{\nomfichier}{q228.py}
\begin{solution}
    \pythonfile{\chemincode \nomfichier}[][\nomfichier]
\end{solution}
